\documentclass{template_practica}

\begin{document}

\practiceheader{Práctica 3: Relaciones}{Comisión: Rodrigo Cossio-Pérez y Leonardo Lattenero}

\begin{enumerate}

	\exercise Analizar las siguientes relaciones indicando dominio e imagen, graficandolas con el método indicado cuando se indique y estudiando sus propiedades.
	\begin{enumcols}
		\item $R=\{(x,y) \in A \times B ~|~ x+y=4\}$ con $A=\{1,2,3\}$ y $B=\{1,2,3,4,5\}$. Graficar mediante una tabla.
        \item $R=\{(x,y) \in A \times B ~|~ y>x\}$ con $A=\{1,2,3\}$ y $B=\{1,2,3,4,5\}$. Graficar en el plano cartesiano.
        \item $R=\{(X,Y) \in \power{C} \times \power{C} ~|~ X \intersec Y = \emptyset\}$ con $A=\{1,2,3\}$ y $B=\{1,2,3,4,5\}$ con $C=\{a,b\}$. Graficar mediante un grafico
        \item $R=\{(x,y) \in \R^2 ~|~ x \in \N \}$. Graficar en un plano cartesiano.
        \item $R=\{(x,y) \in \R^2 ~|~ y=x+1\}$. Graficar en un plano cartesiano.
        \item La relación definida en $\R$ de forma que $R=\{(x,y) ~|~ y>x+1\}$. Graficar en un plano cartesiano.
        \item $R=\{(x,y) \in \R^2 ~|~ x \geq 1 ~\land~ y \geq 2 \}$. Graficar en un plano cartesiano.
        \item La relación $R \subseteq \R^2$ definida por $R=\{(x,y) ~|~ x>3 \land y>3\}$. Graficar en un plano cartesiano.
        \item $R=\{(x,y) \in \R^2 ~|~ 1<x+y<4\}$. Graficar mediante un grafico
        \item $R=\{(x,y) \in \R^2 ~|~ x\neq y\}$. Graficar en un plano cartesiano.
        \item La relación definida en $A=\{a,b,c,d\}$ como $R=\{(a,a), (a,b), (b,a), (c,d)\}$. Graficar mediante un grafo.
        \item La relación $R=\{(a,a),(b,b),(b,c),(c,d).(c,c),(d,d)\}$ definida en $A=\{a,b,c,d\}$. Graficar mediante un grafo.
        \item $R=\{(a,a),(a,b),(b,a),(b,c),(c,b),(b,c)\}$ definida en $A=\{a,b,c,d\}$. Graficar mediante un grafo.
        \item $R=\{(a,b) ~|~ a \text{ vive con } b\}$, definida en el conjunto de personas de Bernal.
        \item $R=\{(a,b) ~|~ a \text{ conoce presencialmente a } b\}$, definida en el conjunto de personas de Argentina.
        \item $R=\{(a,b) ~|~ a \text{ sabe de la existencia de } b\}$, definida en el conjunto de personas.
        \item $R=\{(a,b) ~|~ a \text{ es descensiente de } b\}$, definida en el conjunto de personas.
        \item $R=\{(a,b) ~|~ a \text{ es padre/madre de } b\}$, definida en el conjunto de personas.
        \item $R=\{(a,b) ~|~ a \text{ sigue en las redes a } b\}$, definida en el conjunto de personas.
        \item $R=\{ (x,y) ~|~ x \text{ le gana a } y \}$ definida en $A\{\text{piedra}, \text{papel}, \text{tijera}\}$. Graficar mediante un grafo.
        \item $R=\{ (x,y) ~|~ x \text{ le gana o le empata a } y \}$ definida en $A\{\text{piedra}, \text{papel}, \text{tijera}\}$. Graficar mediante un grafo.

	\end{enumcols}


\end{enumerate}

\end{document}