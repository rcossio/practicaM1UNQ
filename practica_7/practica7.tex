\documentclass{template_practica}

\begin{document}

\practiceheader{Práctica 7: Sucesiones e Inducción}{Comisión: Rodrigo Cossio-Pérez y Leonardo Lattenero}

\begin{enumerate}

	\exercise Escribir una fórmula para el término general de las siguientes sucesiones para $n\in\N$.
	\begin{enumcols}[3]
		\item $1, 3, 5, 7, 9, \cdots$
		\answer $a_n=2n-1$
		\item $1, -1, 1, -1, 1, -1,\cdots$
		\answer $a_n=(-1)^{n+1}$
		\item $-1, 1, -1, 1, -1, \cdots$
		\answer $a_n=(-1)^n$
		\item $0, 2, 0, 2, 0, 2, 0, 2,\cdots$
		\answer $a_n=1+(-1)^n$
		\item $-3, -1, 1, 3, 5, 7, 9,\cdots$
		\answer $a_n=2n-5$
		\item $0, 5, 10, 15, 20,\cdots$
		\answer $a_n=5n$
		\item $3, -5, 7, -9, \cdots$
		\answer $a_n=(-1)^{n+1}.(2n-1)$. Resolución por \href{https://youtu.be/Ikjq2KZVNy0?t=136}{Mathematics with Grajeda}
		\item $1, \f{1}{2}, \f{1}{3}, \f{1}{4}, \f{1}{5}, \cdots$
		\answer $a_n=\f{1}{n}$
		\item $4, 9, 14, 19, 24, 29,\cdots$
		\answer $a_n=5n-1$
		\item $1, \f{1}{2}, \f{1}{4}, \f{1}{8}, \f{1}{16}, \cdots$
		\answer $a_n=\f{1}{2^{n-1}}$
		\item $3,6,12,24,48,\cdots$
		\answer $a_n=3.2^{n-1}$
		\item $4, 1, 1/4, 1/16, 1/64,\cdots$
		\answer $a_n=4.\left(\f{1}{4}\right)^{n-1}$
		\item $2,\f{4}{3}, \f{6}{5}, \f{8}{7}, \cdots$
		\answer $a_n=\f{2n}{2n-1}$. Resolución por \href{https://youtu.be/Ikjq2KZVNy0}{Mathematics with Grajeda}
		\item $0, 3, 8, 15, 24, 35,\cdots$
		\answer $a_n=n^2-1$
		\item $-1, \f{1}{2}, -\f{1}{3}, \f{1}{4}, -\f{1}{5},\cdots$
		\answer $a_n=(-1)^{n+1}.\f{1}{n}$
		\item $-\f{3}{5}, \f{4}{25}, -\f{5}{125}, \f{6}{625}, \cdots$
		\answer $a_n=(-1)^n.\f{n+2}{5^n}$. Resolución por \href{https://youtu.be/Ikjq2KZVNy0?t=353}{Mathematics with Grajeda}
		\item $2,5,10,17,26,37,50,\cdots$
		\answer $a_n=n^2+1$
		\item $5, 13, 25, 41, 61, 85, \cdots$
		\answer $a_n=2n^2+2n+1$
		\item $7,21,45,79,\cdots$
		\answer $a_n=5n^2-n+3$. Resolución por \href{https://youtu.be/koxHXYAtltg}{math2me}
		\item $1, 8, 27, 64, 125, \cdots$
		\answer $a_n=n^3$
	\end{enumcols}

	\exercise Carcular el resultado de las siguientes sumas mediante su definición y propiedades
	\begin{enumcols}
		\item $\Sum_{k=1}^7 \left(2k-4\right)$
		\answer $\Sum_{k=1}^7 \left(2k-4\right)=2\left(\Sum_{k=1}^7 k\right)-4\left(\Sum_{k=1}^7 1\right)=2\f{7(7+1)}{2}-4.7=28$
		\item $\Sum_{t=1}^5 \left(3^t+t^2\right)$
		\answer Considerando la suma de la serie geométrica y la sumatoria de los cuadrados de un número: \\ $\Sum_{t=1}^5 \left(3^t+t^2\right)=\Sum_{t=1}^5 3^t+\Sum_{t=1}^5 t^2=\f{1.(1-3^{10+1})}{1-3}+\f{10(10+1)(2.10+1)}{6}=88958$
		\item $\Sum_{h=5}^9 \left(2-\f{4}{h}\right)$
		\answer $\Sum_{h=5}^9 \left(2-\f{4}{h}\right)=\Sum_{h=5}^9 2 - \Sum_{h=5}^9 \f{4}{h}=2 \left(\Sum_{h=5}^9 1\right) - 4 \left(\Sum_{h=5}^9 \f{1}{h}\right)=2.5-4\left(\f{1}{5}+\f{1}{6}+\f{1}{7}+\f{1}{8}+\f{1}{9}\right)=\f{4421}{630}$

	\end{enumcols}

	\exercise Carcular el resultado de los siguientes productos mediante su definición y propiedades
	\begin{enumcols}
		\item Item
		\answer Respuesta

	\end{enumcols}

	\exercise Escribir las siguientes expresiones como sumatorias o productorias
	\begin{enumcols}
		\item Item
		\answer Respuesta

	\end{enumcols}

	\exercise Utilizando el principio de inducción matemática o una de sus variantes, demuestrar las siguientes afirmaciones.
	\begin{enumcols}
		\item Item
		\answer Respuesta

	\end{enumcols}

	\exercise Dadas las siguientes sucesiones recursivas, proponer una formula general y demostrarla por principio de inducción.
	\begin{enumcols}
		\item Item
		\answer Respuesta

	\end{enumcols}

\end{enumerate}

\end{document}