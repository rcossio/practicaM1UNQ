\documentclass{template_practica}

\begin{document}

\practiceheader{Práctica 7: Sucesiones e Inducción}{Comisión: Rodrigo Cossio-Pérez y Leonardo Lattenero}

\begin{enumerate}

	\exercise Escribir una fórmula para el término general de las siguientes sucesiones para $n\in\N$.
	\begin{enumcols}[3]
		\item $1, 3, 5, 7, 9, \cdots$
		\answer $a_n=2n-1$
		\item $1, -1, 1, -1, 1, -1,\cdots$
		\answer $a_n=(-1)^{n+1}$
		\item $-1, 1, -1, 1, -1, \cdots$
		\answer $a_n=(-1)^n$
		\item $0, 2, 0, 2, 0, 2, 0, 2,\cdots$
		\answer $a_n=1+(-1)^n$
		\item $-3, -1, 1, 3, 5, 7, 9,\cdots$
		\answer $a_n=2n-5$
		\item $0, 5, 10, 15, 20,\cdots$
		\answer $a_n=5n$
		\item $3, -5, 7, -9, \cdots$
		\answer $a_n=(-1)^{n+1}.(2n-1)$. Resolución por \href{https://youtu.be/Ikjq2KZVNy0?t=136}{Mathematics with Grajeda}
		\item $1, \f{1}{2}, \f{1}{3}, \f{1}{4}, \f{1}{5}, \cdots$
		\answer $a_n=\f{1}{n}$
		\item $4, 9, 14, 19, 24, 29,\cdots$
		\answer $a_n=5n-1$
		\item $1, \f{1}{2}, \f{1}{4}, \f{1}{8}, \f{1}{16}, \cdots$
		\answer $a_n=\f{1}{2^{n-1}}$
		\item $3,6,12,24,48,\cdots$
		\answer $a_n=3.2^{n-1}$
		\item $4, 1, 1/4, 1/16, 1/64,\cdots$
		\answer $a_n=4.\left(\f{1}{4}\right)^{n-1}$
		\item $2,\f{4}{3}, \f{6}{5}, \f{8}{7}, \cdots$
		\answer $a_n=\f{2n}{2n-1}$. Resolución por \href{https://youtu.be/Ikjq2KZVNy0}{Mathematics with Grajeda}
		\item $0, 3, 8, 15, 24, 35,\cdots$
		\answer $a_n=n^2-1$
		\item $-1, \f{1}{2}, -\f{1}{3}, \f{1}{4}, -\f{1}{5},\cdots$
		\answer $a_n=(-1)^{n+1}.\f{1}{n}$
		\item $-\f{3}{5}, \f{4}{25}, -\f{5}{125}, \f{6}{625}, \cdots$
		\answer $a_n=(-1)^n.\f{n+2}{5^n}$. Resolución por \href{https://youtu.be/Ikjq2KZVNy0?t=353}{Mathematics with Grajeda}
		\item $2,5,10,17,26,37,50,\cdots$
		\answer $a_n=n^2+1$
		\item $5, 13, 25, 41, 61, 85, \cdots$
		\answer $a_n=2n^2+2n+1$
		\item $7,21,45,79,\cdots$
		\answer $a_n=5n^2-n+3$. Resolución por \href{https://youtu.be/koxHXYAtltg}{math2me}
		\item $1, 8, 27, 64, 125, \cdots$
		\answer $a_n=n^3$
	\end{enumcols}

	\exercise Carcular el resultado de las siguientes sumas mediante su definición y propiedades
	\begin{enumcols}[2]
		\item $\Sum_{k=1}^7 \left(2k-4\right)$
		\answer $\Sum_{k=1}^7 \left(2k-4\right)=2\left(\Sum_{k=1}^7 k\right)-4\left(\Sum_{k=1}^7 1\right)=2\f{7(7+1)}{2}-4.7=28$
		\item $\Sum_{t=1}^5 \left(3^t+t^2\right)$
		\answer Considerando la suma de la serie geométrica y la sumatoria de los cuadrados de un número: \\ $\Sum_{t=1}^5 \left(3^t+t^2\right)=\Sum_{t=1}^5 3^t+\Sum_{t=1}^5 t^2=\f{1.(1-3^{10+1})}{1-3}+\f{10(10+1)(2.10+1)}{6}=88958$
		\item $\Sum_{h=5}^9 \left(2-\f{4}{h}\right)$
		\answer $\Sum_{h=5}^9 \left(2-\f{4}{h}\right)=\Sum_{h=5}^9 2 - \Sum_{h=5}^9 \f{4}{h}=2 \left(\Sum_{h=5}^9 1\right) - 4 \left(\Sum_{h=5}^9 \f{1}{h}\right)=2.5-4\left(\f{1}{5}+\f{1}{6}+\f{1}{7}+\f{1}{8}+\f{1}{9}\right)=\f{4421}{630}$
		\item $\left(\Sum_{j=1}^{7} 2j\right) + \left(\Sum_{i=1}^{15} 2i\right) + \left(\Sum_{j=8}^{15} 2j\right)$
		\answer Cambiamos las sumas a variable $i$: \\ $\left(\Sum_{j=1}^{7} 2j\right) + \left(\Sum_{i=1}^{15} 2i\right) + \left(\Sum_{j=8}^{15} 2j\right)= \left(\Sum_{i=1}^{7} 2i\right) + \left(\Sum_{i=1}^{15} 2i\right) + \left(\Sum_{i=8}^{15} 2i\right)=\\ \left(\Sum_{i=1}^{15} 2i\right) + \left(\Sum_{i=1}^{15} 2i\right)=2\left(\Sum_{i=1}^{15} 2i\right)=4\left(\Sum_{i=1}^{15} i\right)=4\f{15.(15+1)}{2}=480$
		\item $\Sum_{k=-5}^4 (5k)$
		\answer Sustituimos por $i=k+6$: $\Sum_{k=-5}^4 (5k) = \Sum_{i=1}^{10} [5(i-6)] = 5\left(\Sum_{i=1}^{10} i\right) -6 \left(\Sum_{i=1}^{10} 1 \right) = 5\f{10.(10+1)}{2} -6.10= 215 $
		\item $\Sum_{t=1}^{4} (4t^2+5)$
		\item $\Sum_{j=3}^{6} \left(\f{j-1}{j-2}\right)$
		\item $\Sum_{k=3}^{8} [k(k-1)]$
		\item $\Sum_{i=1}^{200} 10$
		\item $\Sum_{t=5}^{9} [1+(-1)^t]$
		\item $\Sum_{j=8}^{70} 20$
		\item $\Sum_{i=1}^{30} (4i+5)$
		\item $\Sum_{j=1}^{33} [-3(j-1)+2]$
		\item $\Sum_{k=1}^{45} [4 + 5(k-1)]$
		\item $\Sum_{t=1}^{h} (3t+1)$

	\end{enumcols}

	\exercise Calcular el resultado de los siguientes productos mediante su definición y propiedades
	\begin{enumcols}[3]
		\item $\Prod_{k=4}^{6} (k+1)$
		\answer $\Prod_{k=4}^{6} (k+1)=5.6.7=210$
		\item $\Prod_{j=1}^{n} j$
		\answer $\Prod_{j=1}^{n} j= n!$
		\item $\Prod_{i=17}^{50} i$
		\answer $\Prod_{i=17}^{50} i = \f{\prod_{i=1}^{50} i}{\prod_{i=1}^{16} i} = \f{50!}{16!}= 1,4536347~.10^{51}$
		\item $\f{\prod_{h=1}^{10} (2h^2+1)}{\prod_{h=1}^{9} (2h^2+1)}$
		\answer $\f{\prod_{h=1}^{10} (2h^2+1)}{\prod_{h=1}^{9} (2h^2+1)}=2.10^2+1=201$
		\item $\Prod_{m=1}^{20} (-1m)$
		\answer $\Prod_{m=1}^{20} (-1m) = (-1)^{20} \left(\Prod_{m=1}^{20} m\right) = 1.20!= 2,432902~.10^{18} $
		\item $\Prod_{k=3}^{10} (2k)$
		\answer $\Prod_{k=3}^{10} (2k)=\f{\prod_{k=1}^{10} (2k)}{\prod_{k=1}^{2} (2k)}=\f{2^{10}. 10!}{2.4}=464486400$

	\end{enumcols}

	\exercise Escribir las siguientes expresiones como sumatorias o productorias
	\begin{enumcols} [2]
		\item $1+2+3+4+\cdots+100$
		\item $1+2+4+8+16+\cdots+1024$
		\item $1 + 4 + 9 + 16 + 25 + \cdots + 81$
		\item $1 - 1 + 1 - 1 + 1 - 1 + 1$
		\item $1 + 2 + 3 + 4 + 5 + \cdots +46$
		\item $4 + 5 + 6 + 7 + 8 + \cdots +34$
		\item $1+(-4)+9+(-16)+25+\cdots+(-144)$
		\item $1+9+25+49+\cdots+441$
		\item $13 + 20 + 27 + 34 + 41 + \cdots + (13 + 7n)$
		\item $8 + \f{2}{3} + \f{1}{18} + \f{1}{216} + \cdots + 8 \left(\f{1}{12}\right)^k$
		\item $2 + 4 + 6 + 8 + \cdots + 2t$
	

	\end{enumcols}

	\exercise Utilizando el principio de inducción matemática o una de sus variantes, demuestrar las siguientes afirmaciones sobre sumas y productos
	\begin{enumcols}
		\item $2+4+6+\cdots+2n=n(n+1)$ para todo $n \in \N$
		\answer Resolución por \href{https://youtu.be/s9pAlcuP6vY?t=965}{Paola Ring}

		\item $1+3+5+\cdots+(2n-1) = n^2 $ para todo $n \in \N$
		\answer Resolución por \href{https://youtu.be/zO2aAq-rQJk}{Mate316}

		\item $\Sum_{i=1}^{n} i = \f{n(n+1)}{2}$ para todo $n \in \N$
		\answer Resolución por \href{https://youtu.be/sMaz5UU9xl0}{MateFacil}

		\item $\Sum_{i=1}^{n} i^2 = \f{n(n+1)(2n+1)}{6}$ para todo $n \in \N$
		\answer Resolución por \href{https://youtu.be/RpF9Z3I1cv0}{MateFacil}

		\item $\Sum_{i=1}^{n} i^3 = \f{n^2(n+1)^2}{4}$ para todo $n \in \N$
		\answer Resolución por \href{https://youtu.be/ZTledCV8bQs}{MateFacil}

		\item $\f{1}{1.2}+\f{1}{2.3}+\f{1}{3.4}+\cdots+\f{1}{n(n+1)}=\f{n}{n+1}$ para todo $n$ natural
		\answer Resolución por \href{https://youtu.be/S0gA9cwM_Y8?t=777}{Ricardo Jara}

		\item $\left(1+\frac{1}{1}\right).\left(1+\frac{1}{2}\right).\left(1+\frac{1}{3}\right)\cdots\left(1+\frac{1}{n}\right)=n+1$ para todo $n\in\N$
		\answer Resolución por \href{https://youtu.be/s9pAlcuP6vY?t=1627}{Paola Ring}

		\item $1+a+a^2+a^{n-1}=\f{a^n-1}{a-1}$ para $a\neq1$
		\answer Resolución por \href{https://youtu.be/QQbmattbUHk}{MateFacil}

		\item $\Sum_{i=1}^{n} (i.i!) = (n+1)! -1$ para todo $n$ natural
		\answer Resolución por \href{https://youtu.be/zAIi_viXCgc}{La Prof Lina M3}

		\item $\Sum_{i=1}^{n} [i(i+1)] = \f{n(n+1)(n+2)}{3}$
		\answer Resolución por \href{https://youtu.be/l5gkAwEWCfw}{Mate316}

		\item $\Sum_{j=0}^n (2j+1) = (n+1)^2$
		\answer Resolución por \href{https://youtu.be/kVLLbEIQamk}{La Prof Lina M3}

		\item $\forall n\in \N: \Sum_{i=1}^n (5i-3) = \f{n(5n-1)}{2}$
		\answer Resolución por \href{https://youtu.be/eSmC1fSGTyM}{María Alicia Piñeiro}

		\item $\Sum_{k=1}^{n} [k(k+2)]=\f{n(n+1)(2n+7)}{6}$
		\answer Resolución por \href{https://youtu.be/gRGOjTK48XU?t=941}{MathLogic}
		\item $\Sum_{k=1}^{n} \left(\f{2}{3}\right)^k=2-\f{2^{n+1}}{3^n}$
		\item $\Sum_{k=1}^{n} \f{k}{2^k} = 2- \f{n+2}{2^n}$
		\answer Resolución por \href{https://youtu.be/2ybajZqIjqg}{Ricardo Jara}

		\item $\Sum_{k=1}^{n} 2^{k}. 3^{k+1} = \f{18}{5}(6^n-1)$
		\item $\forall n\in \N: \Sum_{i=1}^n i.2^{i} = (n-1).2^{n+1}+2$
		\answer Resolución por \href{https://youtu.be/s9pAlcuP6vY?t=1420}{Paola Ring}

		\item $\forall n\in \N: \Sum_{i=1}^n 3^i=\f{3(3^n-1)}{2}$
		\answer Resolución por \href{https://youtu.be/s9pAlcuP6vY}{Paola Ring}

		\item $\forall n\in \N: \Sum_{k=1}^n \frac{1}{(2k-1)(2k+1)}=\frac{n}{2n+1}$
		\answer Resolución por \href{https://youtu.be/s9pAlcuP6vY?t=1168}{Paola Ring}

	\end{enumcols}

	\exercise Utilizando el principio de inducción matemática o una de sus variantes, demuestrar las siguientes afirmaciones sobre desigualdades.
	\begin{enumcols}
		\item $1+2^n \leq 3^n$ para todo $n \in \N$
		\answer Resolución por \href{https://youtu.be/41KonBuMXBs}{MateFacil}

		\item $2n+1 < 2^n$ para $n= 3,4,5,6, \cdots$
		\item $\forall n\in \N: 3^n > 2^n$
		\item $\forall n\in \N: n^2+n > 2n-1$
		\answer Resolución por \href{https://youtu.be/s3Ro7v6CV2w}{Mateclips}
		\item $\forall n\in \N: n^2+3n > n+3$
		\item $\forall n\in \Z_{\geq 3}: n! > 3^{n-2}$
		\answer Resolución por \href{https://youtu.be/daEODv_Zx7E&t=878}{Jonathan Castro}
		\item $\forall n\in \Z_{\geq 4}: 2^n < n!$
		\answer Resolución por \href{https://youtu.be/DAW-UKkcv1A}{María Alicia Piñeiro}
		\item $\forall n\in \Z_{\geq 5}: n^4 < 4^n$
		\answer Resolución por \href{https://youtu.be/Rb1xCOK7cxw}{María Alicia Piñeiro}
		\item $\Sum_{k=1}^{n} \f{1}{k^2} \leq 2-\f{1}{n}$ para $n\in\N$
		\answer Resolución por \href{https://youtu.be/S2tkLx019QI}{Eddie Woo}(en inglés)
		\item $\Sum_{j=1}^{n} \f{1}{\sqrt{j}} > \sqrt{n}$ para $n\in\Z_{\geq 2}$
		\answer Resolución por \href{https://youtu.be/dHGNvrXk5Tg}{Sheldon Collier} (en inglés)

	\end{enumcols}

	\exercise Utilizando el principio de inducción matemática o una de sus variantes, demuestrar las siguientes propiedades sobre divisibilidad.
	\begin{enumcols}
		\item $5^n-1$ es divisible por $4$ para $n\in \N$
		\answer Resolución por \href{https://youtu.be/z_rUB9_D0qM}{Juliana la Profe}

		\item $6~|~7^n-1$ para $n\in \N$
		\answer Resolución por \href{https://youtu.be/70gVrG1hn3Y}{EjerciciosMFC}

		\item $2^{2n-1}+3^{2n-1}$ es divisible por 5 para $n\in \N$
		\answer Resolución por \href{https://youtu.be/Pi1QnP0XBts}{lasmatematicas.es}

		\item $6.7^n-2.3^n$ es divisible por $4$ para $n \in \N$
		\item $11^{n+1}+12^{2n-1}$ es múltiplo de $19$, para $n\in\N$
		\answer Resolución por \href{https://youtu.be/kGS6Vtm9xQo}{MathPures}

		\item $\forall n\in \N: 2 ~|~ n^2+n$
		\answer Resolución por \href{https://youtu.be/L47ENIt6Xqw}{MateFacil}

		\item $\forall n\in \N: 3 ~|~ n^3+2n$
		\answer Resolución por \href{https://youtu.be/9I9NSzolQIM}{Mate316}

		\item $\forall n\in \N: 15 ~|~ 2^{4n} -1$
		\answer Resolución por \href{https://youtu.be/BLpIbJAg07w}{Juliana la Profe}

		\item $\forall n\in \N: 5 ~|~ 8^n -3^n$
		\answer Resolución por \href{https://youtu.be/CfQ-SyXPYjU}{MateFacil}

		\item $\forall n\in \N: 7 ~|~ 8^n +1$
		\answer Resolución por \href{https://youtu.be/mTPW6ORkF-s}{Maths Godserved} (en inglés)
		\item $\forall n\in \N: 7^{2n}+16n-1$ es múltiplo de $64$

		\item $\forall n\in \N: n^3+6n$ es divisible por $6$
		\answer Resolución por \href{https://youtu.be/F5OinC_8PQU}{María Alicia Piñeiro}
		
		\item $\forall n\in\N: n \text{ es impar } \then 8~|~n^2-1$
		\answer Resolución por \href{https://youtu.be/QZiFMtV9x_s}{UCAM}

		\item $\forall n\in \N: 3^n$ es impar
		\item $\forall n\in \N: 2^n$ no es múltiplo de $3$
		\item $\forall n\in \N$: Si $n$ es par $2^n \text{ mod } 3=1$, y si $n$ es impar $2^n \text{ mod } 3=2$

	\end{enumcols}

	\exercise Utilizando el principio de inducción matemática o una de sus variantes, demuestrar las siguientes propiedades sobre conjuntos.
	\begin{enumcols}[2]
		\item $|A|=n \then |\power{A}|=2^n$ para $n\in\N_0$
		\answer Resolución por \href{https://youtu.be/sP7GCqQf3Os}{María Alicia Piñeiro}

		\item $\left(\displaystyle\bigcup_{i=1}^n A_i\right) \cap B = \displaystyle\bigcup_{i=1}^n (A_i \cap B)$ para $n\in\N$
		\item $\compl{\displaystyle\bigcup_{i=1}^n A_i} \union B = \displaystyle\bigcap_{i=1}^n (\compl{A_i} \cup B)$ para $n\in\N$
	\end{enumcols}

	\exercise Utilizando el principio de inducción matemática o una de sus variantes, demuestrar las siguientes propiedades sobre el número combinatorio para $n\in\N$
	\begin{enumcols}[2]
		\item $\Sum_{k=0}^n \binom{n}{k} = 2^n$
		\answer Resolución por \href{https://youtu.be/nABscJsDW6c}{MateFacil}
		
		\item $(1+x)^n = \Sum_{r=0}^n \binom{n}{r} x^r$

		\item $(a+b)^n = \Sum_{r=0}^n \binom{n}{r} a^{n-r} b^r$
		\answer Resolución por \href{https://youtu.be/rp-CnvR-X7I}{MateFacil}
	\end{enumcols}

	\exercise Utilizando el principio de inducción matemática o una de sus variantes, demuestrar las siguientes propiedades sobre sucesiones escritas en forma recursiva
	\begin{enumcols}
		\item Dada $(a_n): \SEL{a_1=2 \\ a_2=8 \\ a_n=4( a_{n-1}-a_{n-2})}$ demostrar que $\forall n\in\N: a_n= n.2^n$
		\answer Resolución por \href{https://youtu.be/-6HGsZXRido}{Carlos Fuentes}

		\item Dada $(a_n): \SEL{a_1=2 \\ a_2=2 \\ a_n=2a_{n-1}+a_{n-2}}$ demostrar que $a_n\leq \left(\f{5}{2}\right)^{n-1}$
		\answer Resolución por \href{https://youtu.be/5EeiDlZf95w}{Pandágoras}
	
	\end{enumcols}

	\exercise Dadas las siguientes sucesiones recursivas, proponer una formula general y demostrarla por principio de inducción.
	\begin{enumcols}
		\item $(a_h): \SEL{a_1=2 \\ a_{h+1}=a_h+4}$ ~~~~ con $h\in\N$
		\item $(a_h): \SEL{a_1=1 \\ a_{h}=2a_{h-1}+1}$ ~~~~ con $h\in\Z_{\geq 2}$
		\item $(a_k): \SEL{a_1=1 \\ a_{k+1}=\left(1+\sqrt{a_k}\right)^2}$ ~~~~ con $k\in\N$
		\item $(a_k): \SEL{a_1=3 \\ a_{k+1}=k a_k}$ ~~~~ con $k\in\N$
		\item $(a_i): \SEL{a_1=1 \\ a_{i+1}=a_i + 2i+1}$ ~~~~ con $i\in\N$
		\item $(a_j): \SEL{a_1=2 \\ a_{j+1}=2 -\f{1}{a_j}}$ ~~~~ con $j\in\N$
		\item $(a_n): \SEL{a_0=4 \\ a_{n}=2a_{n-1} +3}$ ~~~~ con $n\in\N_0$
		\item $(a_n): \SEL{a_0=1 \\ a_1=2 \\ a_{n+1}=3a_n -a_{n-1} +1}$ ~~~~ con $n\in\N_0$
		\item $(a_n): \SEL{a_0=1 \\ a_{n+1}= a_n  + n}$ ~~~~ con $n\in\N_0$
		\item $(a_n): \SEL{a_1=1 \\ a_2=1 \\ a_{n+2}= a_{n+1}+a_{n}}$ ~~~~ con $n\in\N$

	\end{enumcols}


\end{enumerate}

\end{document}