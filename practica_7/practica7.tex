\documentclass{template_practica}

\begin{document}

\practiceheader{Práctica 7: Sucesiones e Inducción}{Comisión: Rodrigo Cossio-Pérez y Leonardo Lattenero}

\begin{enumerate}

	\exercise Escribir una fórmula para el término general de las siguientes sucesiones para $n\in\N$.
	\begin{enumcols}[2]
		\item $1, 3, 5, 7, 9, \cdots$
		\item $1, -1, 1, -1, 1, -1,\cdots$
		\item $-1, 1, -1, 1, -1, \cdots$
		\item $0, 2, 0, 2, 0, 2, 0, 2,\cdots$
		\item $-3, -1, 1, 3, 5, 7, 9,\cdots$
		\item $0, 5, 10, 15, 20,\cdots$
		\item $3, -5, 7, -9, \cdots$
		\answer $a_n=(-1)^(n+1).(2n-1)$. Resolución por \href{https://youtu.be/Ikjq2KZVNy0?t=136}{Mathematics with Grajeda}
		\item $1, \f{1}{2}, \f{1}{3}, \f{1}{4}, \f{1}{5}, \cdots$
		\item $2, 5, 7, 12, 19, 31, 50, \cdots$
		\item $4, 9, 14, 19, 24, 29,\cdots$
		\item $1, \f{1}{2}, \f{1}{4}, \f{1}{8}, \f{1}{16}, \cdots$
		\item $0, 3, 8, 15,24, 35,\cdots$
		\item $2,\f{4}{3}, ,\f{6}{5}, \f{8}{7}, \cdots$
		\answer $a_n=\f{2n}{2n-1}$. Resolución por \href{https://youtu.be/Ikjq2KZVNy0}{Mathematics with Grajeda}
		\item $-1, \f{1}{2}, -\f{1}{3}, \f{1}{4}, -\f{1}{5},\cdots$
		\item $-\f{3}{5}, \f{4}{25}, -\f{5}{125}, \f{6}{625}, -\f{7}{3125}, \cdots$
		\answer $a_n=(-1)^n.\f{n+2}{5^n}$. Resolución por \href{https://youtu.be/Ikjq2KZVNy0?t=353}{Mathematics with Grajeda}
		\item $1, 8, 27, 64, 125, \cdots$


	\end{enumcols}

	\exercise Escribir una fórmula cerrada para las siguientes sucesiones.
	\begin{enumcols}
		\item Item
		\answer Respuesta

	\end{enumcols}

	\exercise Carcular el resultado de las siguientes sumas mediante su definición y propiedades
	\begin{enumcols}
		\item Item
		\answer Respuesta

	\end{enumcols}

	\exercise Carcular el resultado de los siguientes productos mediante su definición y propiedades
	\begin{enumcols}
		\item Item
		\answer Respuesta

	\end{enumcols}

	\exercise Escribir las siguientes expresiones como sumatorias o productorias
	\begin{enumcols}
		\item Item
		\answer Respuesta

	\end{enumcols}

	\exercise Utilizando el principio de inducción matemática o una de sus variantes, demuestrar las siguientes afirmaciones.
	\begin{enumcols}
		\item Item
		\answer Respuesta

	\end{enumcols}

	\exercise Dadas las siguientes sucesiones recursivas, proponer una formula general y demostrarla por principio de inducción.
	\begin{enumcols}
		\item Item
		\answer Respuesta

	\end{enumcols}

\end{enumerate}

\end{document}