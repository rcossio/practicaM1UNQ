\documentclass[a4paper]{article}
\usepackage[margin=1.5cm]{geometry}

%\documentclass[11pt]{article}
%\usepackage[paperwidth=9cm,paperheight=60cm,margin=0.4cm]{geometry}

\usepackage{multicol}
\usepackage{enumitem}
\usepackage{graphicx}

%Links
\usepackage[colorlinks = true,
            linkcolor = blue,
            urlcolor  = blue,
            citecolor = blue,
            anchorcolor = blue]{hyperref}

%Simbolos matemáticos
\usepackage{amsmath}
\usepackage{amssymb}

%Enumeracion
\usepackage{enumitem}

%Páginas sin numeración
\pagestyle{empty}

%Interlineado
\renewcommand{\baselinestretch}{1.5}

%Arreglar comillas
\usepackage [autostyle]{csquotes}
\MakeOuterQuote{"}

%Macros
\newcommand{\Item}{\item[\stepcounter{enumii}$\blacktriangleright$\textbf{(\alph{enumii})}]} %Negrita en algunos items
\newcommand{\answer}{\item[**]}
\newcommand{\exercise}{\item}

%Logic macros
\newcommand{\then}{\to}
\newcommand{\eq}{\leftrightarrow}
\newcommand{\xor}{\veebar}
\newcommand{\nor}{\downarrow}
\newcommand{\nimply}{\nrightarrow}
\newcommand{\nand}{\uparrow}
\newcommand{\Then}{\Rightarrow}
\newcommand{\Eq}{\Leftrightarrow}


\begin{document}

\noindent \hrulefill 
\vspace{-7pt}
\begin{center} 
	\textbf{ Práctica 5: Funciones} \\
	Comisión: Rodrigo Cossio-Pérez y Leonardo Lattenero
\end{center}
\vspace{-10pt}
\hrulefill


\begin{enumerate}

	\exercise Indicar cuáles de las siguientes relaciones son funciones y en caso de que si, demostrarlo o justificarlo.
	%\begin{multicols}{2}
	\begin{enumerate} [label=(\alph*)]
		\item $R=\{(x,y)\in\mathbb{R}^2 ~|~ |x|=|y|\}$

		\item $R=\{(x,y)\in\mathbb{R}^2 ~|~ 2y+5=x^2 \}$

		\item $R=\{(x,y)\in\mathbb{R}^2 ~|~ y=\sqrt{x} \}$

		\item $R=\{(x,y)\in [0,+\infty)\times\mathbb{R} ~|~ y=\sqrt{x} \}$

		\item $R=\{(x,y)\in\mathbb{R}^2 ~|~ y=3x+1 \lor y=4x+1 \}$

		\item $R=\{(x,y)\in\mathbb{R}^2 ~|~ (y=3x+1 \land x\geq0) \lor (y=4x+1 \land x<0) \}$

	\end{enumerate}
	%\end{multicols}

	\exercise Para cada una de las siguientes definiciones, indicar si corresponde a una función o no. De las que sí son funciones, indicar dominio y codominio. De las que no son funciones, indicar si lo que no se cumple es unicidad, existencia o ambas, y justificar.
	%\begin{multicols}{2}
	\begin{enumerate} [label=(\alph*)]
		\item Cada estudiante del curso con su altura en centímetros.
		\item Cada estudiante del curso con su fecha de nacimiento.
		\item Cada estudiante del curso con cada materia que aprobó.
		\item Cada estudiante del curso con el barrio en donde vive.
		\item Cada auto con cada taller donde se hizo un service.
		\item Cada auto con el primer taller en donde se hizo un service.
		\item Cada curso con el aula en que se dicta.
		\item Cada paloma con la cantidad de plumas que tiene.
		\item Cada ciudad de la Argentina con la provincia donde está.
		\item Cada ciudad de la Argentina con la provincia de la que es capital.
		\item Cada persona con el club del que es socio/a.
	\end{enumerate}
	%\end{multicols}

	\exercise Definir una función que describa la situación, indicar el dominio, codominio e imagen y graficarla.
	%\begin{multicols}{2}
	\begin{enumerate} [label=(\alph*)]
		\item Un mayorista ofrece la siguiente oferta sobre un tipo de galletitas: hasta 5 paquetes se venden a 6 pesos el paquete; pasando los 5 paquetes hasta los 10, 5 pesos por paquete adicional; pasando los 10 paquetes, 3 pesos por paquete adicional. Por ejemplo, si una persona compra 12 paquetes, paga (5.6)+(5.5)+(2.3) = 61 pesos.

		\item Otro mayorista ofrece una oferta distinta: hasta 5 paquetes se venden a 6 pesos el paquete; entre 6 y 10 paquetes se vende a 5 pesos el paquete; a partir de 11
		paquetes, se vende a 4.5 pesos por paquete. Por ejemplo, si una persona compra 13
		paquetes, paga 13 x 4.5 = 58.5 pesos.

		\item El mismo mayorista anterior pero redondeando como no tiene monedas inferiores a un peso para dar vuelto se ve obligado a redondear para abajo el valor cobrado

		\item Otro mayorista vende las galletitas sueltas por peso y no por paquete. Ofrece lo siguiente: hasta 3kg se vende a 30 pesos el kilo; más de 3kg y hasta 6kg se vende a 25 pesos el kilo; más de 6kg se vende a 20 pesos el kilo. Por ejemplo, si una persona compra 7kg y paga 7 x 20 = 140 pesos. Debido a que el proveedor cobra digitalmente, se puede pagar el monto exacto sin redondear.
	\end{enumerate}
	%\end{multicols}

	\exercise Se definen las siguientes relaciones en $A = \{a, b, c, d, e\}$, indicar de estas cuáles son 	funciones. Para las que sí sean funciones, indicar si son inyectivas y/o suryectivas, y si se puede, definir por extensión la función inversa.
	%\begin{multicols}{2}
	\begin{enumerate} [label=(\alph*)]
		\item $R = \{(a,b), (b,c), (c,d)\}$
		\item $R = \{(a,b), (b,c), (c,d), (d,e), (e,a)\}$
		\item $R = \{(a,b), (b,c), (b,d), (d,e), (e,a)\}$
		\item $R = \{(a,a), (b,a), (c,d), (d,a), (e,a)\}$
		\item $R = \{(a,c), (b,e), (c,a), (d,b), (e,d)\}$
	\end{enumerate}
	%\end{multicols}

	\exercise De cada una de estas funciones, indicar si es inyectiva y suryectiva, justificando. Para las que sean biyectivas, decir cuál es la función inversa, indicando dominio y codominio
	%\begin{multicols}{2}
	\begin{enumerate} [label=(\alph*)]
		\item La función que indica la fecha de nacimiento de cada estudiante de la Universidad, donde el codominio son los días desde el 1ro de enero de 1800.
		\item La función del número de vagón en el que está cada pasajero de un tren que no tiene vagones vacíos.
		\item La función del número de asiento de los pasajeros de un vuelo. Pensar en dos casos: avión lleno, y avión no lleno.
		\item La función que indica el número de DNI de los residentes en Argentina, tomando como codominio los naturales.
		\item La función que relaciona de las personas que viven en un edificio e indica el
		departamento en que vive cada una.
		\item La función que relaciona los perros que se encuentran en un parque con el dueño de cada perro.
		\item La función que va de cada provincia de Argentina a su capital, tomando como
		codominio el conjunto de las ciudades capitales de provincia.
	\end{enumerate}
	%\end{multicols}

	\exercise Para cada una de las leyes de asignación indicadas, indicar el dominio más amplio para definir una función $f: D_{f} \to \mathbb{R}$. Luego graficar la función e indicar si es inyectiva y/o suryectiva. Para las funciones que resulten biyectivas, definir la inversa.
	%\begin{multicols}{2}
	\begin{enumerate} [label=(\alph*)]
		\item $3(x-\lfloor x \rfloor)$
		\item $2x-1$
		\item $7-3x$
		\item $x^2+4x-3$
		\item $\left\{\begin{matrix}2x-1 ~~~~ si ~~x\leq 1\\ ~~x^2~~ ~~~~~ si ~~x>1\end{matrix}\right.$
		\item $\left\{\begin{matrix}-x ~~~~ si ~~x\leq 0\\ ~x  ~~~~  si ~~0<x\leq 3 \\ -x ~~~~ si ~~x>3\end{matrix}\right.$
		\item $\left\{\begin{matrix}\hspace{-2mm}3x-2 ~~~ si ~~x\leq 2 ~~\\ \hspace{-2mm} ~~~~~ x^2 ~~~~~~  si ~~2<x\leq 3 \\ \hspace{-2mm}\displaystyle{\frac{x+6}{9}} ~~~ si ~~x>3 ~~\end{matrix}\right.$
		\item $\left\{\begin{matrix}x^2+7 ~~~~ si ~~x\leq 2\\ x+4 ~~~~~ si ~~x>2\end{matrix}\right.$
		\item $\sin(2x)$
		\item $\cos(x)+3$
		\item $2\arctan(x)$
		\item $\displaystyle{\frac{1}{2-\sqrt{x}}}$
		\item $\log_2(x^2)$
		\item $\exp(-x)=e^{-x}$
		\item $2^x$
	\end{enumerate}
	%\end{multicols}

	\exercise Indicar a qué función corresponde este gráfico. Observando el gráfico, indicar si la función es inyectiva, y si es suryectiva.
	%\begin{multicols}{2}
	\begin{enumerate} [label=(\alph*)]
		\item ~\\ [-15pt] \includegraphics[scale=0.5]{func1.png}
	\end{enumerate}
	%\end{multicols}

	\exercise En cada caso y de ser posible, calcular las funciones $g \circ f$ y $f \circ g$:
	%\begin{multicols}{2}
	\begin{enumerate} [label=(\alph*)]
		\item $f,g:\mathbb{R} \to \mathbb{R}$ con $f(x)=3x$ ~y~ $g(x)=x-1$
		\item $f,g:\mathbb{R} \to \mathbb{R}$ con $f(x)=\lfloor x \rfloor$ ~y~ $g(x)=x^2$
		\item $f:\mathbb{R} \to \mathbb{R} ~|~ f(x)=2x$ ~y~ $g:\left[0,+\infty\right) \to \left[0,+\infty\right) ~|~ g(x)=\sqrt{x}$
		\item $f,g:\mathbb{R} \to \mathbb{R}$ con $f(x)=\sin(x)$ ~y~ $g(x)=3x+4$
		\item $f,g:\mathbb{R} \to \mathbb{R}$ con $f(x)=3x$ ~y~ $g(x)=\left\{\begin{matrix}x+3 ~~~ si ~~x\leq 6\\ x+5 ~~~ si ~~x>6\end{matrix}\right.$
		\item $f,g:\mathbb{R} \to \mathbb{R}$ con $f(x)=|x|$ ~y~ $g(x)=x+4$
		\item $f:\left(0,+\infty\right) ~|~ f(x)=\log(x)$ ~y~ $g:\mathbb{R} \to \mathbb{R} ~|~ g(x)=x^2$
		\item $f,g:\mathbb{R} \to \mathbb{R}$ con $f(x)=e^{x}$ ~y~ $g(x)=|x|$
		\item $f:\mathbb{R}\setminus\{0\} \to \mathbb{R} ~|~ f(x)=\displaystyle{\frac{1}{x}}$ ~y~ $g:\mathbb{R} \to \mathbb{R} ~|~ g(x)=\cos(x)$

	\end{enumerate}
	%\end{multicols}

	\exercise Dadas $f,g,h: \mathbb{R} \to \mathbb{R}$. Definir $f \circ g \circ h$ y $h \circ g \circ f$. 
	%\begin{multicols}{2}
	\begin{enumerate} [label=(\alph*)]
		\item $f(x)=x^2+2x$, $g(x)=\displaystyle\frac{x}{4}$ y $h(x)=x+12$.
	\end{enumerate}
	%\end{multicols}

	\exercise Considerando la función $f(x)=x^2$ y las funciones a continuación, definir las siguientes funciones componiendo $f$ con una o dos de las funciones. \\ $h_1(x)=x+2$ \\ $h_2(x)=-x$ \\ $h_3(x)=2x$ \\ $h_4(x)=\displaystyle\frac{x}{2}$ \\ $h_5(x)=x-3$ \\ $h_6(x)=x-1$.
	%\begin{multicols}{2}
	\begin{enumerate} [label=(\alph*)]
		\item $(x+2)^2$
		\item $-x^2$
		\item $x^2+2$
		\item $2x^2$
		\item $\left(\displaystyle\frac{x}{2}\right)^2$
		\item $\displaystyle\frac{x^2}{2}$
		\item $(x-3)^2-1$
		\item $(x-1)^2-3$
		\item $(x-1)^2-1$
	\end{enumerate}
	%\end{multicols}

	\exercise Considerando la función $f(x)=|x|$ y las funciones a continuación, definir las siguientes funciones componiendo $f$ con una o dos de las funciones. \\ $h_1(x)=x+2$ \\ $h_2(x)=-x$ \\ $h_3(x)=2x$ \\ $h_4(x)=\displaystyle\frac{x}{2}$ \\ $h_5(x)=x-3$ \\ $h_6(x)=x-4$ \\ $h_7(x)=x+1$ \\ $h_8(x)=x+3$.
	%\begin{multicols}{2}
	\begin{enumerate} [label=(\alph*)]
		\item $|x|+2$
		\item $|x+1|+2$
		\item $-|x|-3$
		\item $-\displaystyle\frac{|x|}{2}$
		\item $-|x|$
		\item $-|x+1|$
		\item $2|x|$
		\item $|x|-4$
		\item $|x|-2$
		\item $-|x+3|$
		\item $-2|x|$
		\item $|x+2|+1$
		\item $-\left|\displaystyle\frac{x}{2}\right|$
		\item $\displaystyle\frac{|x|}{2}$
	\end{enumerate}
	%\end{multicols}

	\exercise Resolver las siguientes ecuaciones:
	%\begin{multicols}{2}
	\begin{enumerate} [label=(\alph*)]
		\item $2^x=10$
		\item $2\ln(x)=4$
		\item $e^{x^2+1}=\displaystyle{\frac{1}{e^2}}$
		\item $\ln(x)+\ln(x^2)=-\ln(6)$
	\end{enumerate}
	%\end{multicols}

	\exercise Graficar las siguientes funciones e indicar el dominio e imagen:
	%\begin{multicols}{2}
	\begin{enumerate} [label=(\alph*)]
		\item $2\ln(x)$
		\item $\ln(x)+1$
		\item $\ln(x-4)$
		\item $-\ln(x)$
		\item $-3\ln(x+1)-2$
		\item $\ln(-x)$
		\item $e^{2x}$
		\item $e^{-x}$
		\item $3e^{x}$
		\item $-e^{4x}$
		\item $e^{x-2}$
		\item $2^{x+1}$
	\end{enumerate}
	%\end{multicols}

	\exercise Graficar las siguientes funciones e indicar el dominio e imagen:
	%\begin{multicols}{2}
	\begin{enumerate} [label=(\alph*)]
		\item $\sin(2x)$
		\item $3\sin(x)$
		\item $-2\sin(x)$
		\item $\sin(x-\pi)$
		\item $\sin(\pi x)+5$
		\item $\sin(-x)$
		\item $-2\sin\left(x-\frac{\pi}{2}\right)+4$
		\item $2\cos(3x)$
		\item $-\cos\left(x-\frac{\pi}{2}\right)$
		\item $\cos(2x+100\pi)$
	\end{enumerate}
	%\end{multicols}

	\exercise Resolver las siguientes ecuaciones hallando todas las soluciones posibles.
	%\begin{multicols}{2}
	\begin{enumerate} [label=(\alph*)]
		\item $2\sin(x)=1$
		\item $3\sin(x)=0$
		\item $4\sin^{2}\left(x-\frac{\pi}{2}\right)=0$
		\item $\sin(x)-\frac{\sqrt{3}}{2}=0$
		\item $2\cos(-x)=\sqrt{2}$
		\item $20\cos(x)+60=-80$
	\end{enumerate}
	%\end{multicols}

	\exercise Graficar las siguientes funciones.
	%\begin{multicols}{2}
	\begin{enumerate} [label=(\alph*)]
		\item $\displaystyle\frac{2}{x}$
		\item $\displaystyle\frac{x+1}{x+1}$
		\item $\displaystyle\frac{2x+3}{4x-1}$
		\item $x^2-4x+4$
		\item $x^2-5$
		\item $-x^2+1$
		\item $3(x+1)(x-1)$
		\item $(x-1)^2-16$
		\item $4(x+1)^2-4$
		\item $\sqrt{x}$
		\item $\sqrt{-2x}$
		\item $\sqrt{x^2}$
		\item $2\sqrt{x+1}$
		\item $\sqrt{4x}+5$
		\item $-\sqrt{x}+3$
	\end{enumerate}
	%\end{multicols}

	\exercise Hallar $f \circ g$ y graficarla.
	%\begin{multicols}{2}
	\begin{enumerate} [label=(\alph*)]
		\item $f(x)=\sin(x)$ ~y~ $g(x)=2x$
		\item $f(x)=\sqrt{x}$ ~y~ $g(x)=3x-1$
		\item $f(x)=\ln(x)$ ~y~ $g(x)=x-1$
		\item $f(x)=\displaystyle{e^{-x}}$ ~y~ $g(x)=\ln(x)$
		\item $f(x)=\sqrt{x}$ ~y~ $g(x)=\ln(x-1)$
		\item $f(x)=\ln(x)$ ~y~ $g(x)=x^2+4$
		\item $f(x)=\sin(x)$ ~y~ $g(x)=\displaystyle{e^{x^2-2x+1}}$
	\end{enumerate}
	%\end{multicols}

	\exercise Hallar la función inversa cuando sea posible, indicando el dominio e imagen.
	%\begin{multicols}{2}
	\begin{enumerate} [label=(\alph*)]
		\item $f(x)=\displaystyle\frac{2x-1}{3}$
		\item $g(x)=2x^2+x-1$
		\item $h(x)=\ln(x^2-1)+5$
		\item $i(t)=\displaystyle{4+16 e^{-\frac{1}{2}t}}$
	\end{enumerate}
	%\end{multicols}

\end{enumerate}

\end{document}