\documentclass{template_practica}

\begin{document}

\practiceheader{Práctica 5: Funciones}{Comisión: Rodrigo Cossio-Pérez y Leonardo Lattenero}

\begin{enumerate}

	\exercise Determina cuáles de las siguientes relaciones son funciones. Justifica tu respuesta en cada caso y, si son funciones, representa la relación usando la notación adecuada.
	\begin{enumcols}
		\item $R=\{(x,y)\in\R^2 ~|~ |x|=|y|\}$
		\answer No es función ya que $|x|=|y| \Then y=\pm x$, por lo que no hay unicidad. Por ejemplo $(1,1)$ y $(1,-1)$ pertenecen a $R$.

		\item $R=\{(x,y)\in\R^2 ~|~ 2y+5=x^2 \}$
		\answer Es función. $\forall x\in\R ~\exists ! y \in \R: 2y+5=x^2$ con $y=\f{x^2-5}{2}\in\R$ (notar que realizar este cálculo da un valor único para $y$). 

		\item $R=\{(x,y)\in\N^2 ~|~ 3y-7=x \}$
		\answer No es función ya que algunos valores de $x$ no tienen un elemento $y$ relacionado. Si acomodamos la ley de asignación obtenemos $y=\f{x+7}{3}$ pero esta operación no siempre brindará valores naturales, por ejemplo, con $x=1$ se obtiene $y=\f{8}{3}\not\in\N$

		\item $R=\{(x,y)\in\R^2 ~|~ y=\sqrt{x} \}$
		\answer No es función ya que $\sqrt{x}$ no está definida para $x<0$, por lo que no cumple con la condición de existencia.

		\item $R=\{(x,y)\in [0,+\infty)\times\R ~|~ y=\sqrt{x} \}$
		\answer Es función. $\forall x\in [0,+\infty) ~\exists ! y \in \R: y=\sqrt{x}$ ya que la operación $\sqrt{x}$ está definida para $[0,+\infty)$ y brinda un único valor.

		\item $R=\{(x,y)\in\R^2 ~|~ y=3x+1 ~\lor~ y=4x+1 \}$
		\answer No es función ya que hay valores de $x$ relacionados a más de un valor de $y$. Por ejemplo $x=1$ está relacionado a $y=4$ y $y=5$.

		\item $R=\{(x,y)\in\R^2 ~|~ (y=3x+1 ~\land~ x\geq0) ~\lor~ (y=4x+1 ~\land~ x<0) \}$
		\answer Es función. Notar que $(x\geq0) \Eq \neg(x<0)$, por lo que las expresiones $y=3x+1$ y $y=4x+1$ no sucederán simultaneamente. Es decir, para cada valor de $x$ hay un único valor de $y$. Esta función también se puede escribir como $f: \R \to \R ~|~ f(x)=\SEL{3x+1 & si~ x\geq0\\ 4x+1 & si~ x<0}$

	\end{enumcols}

	\exercise Analiza la relación dada entre los elementos y determina si es una función. Para las que sí lo son, especifica el dominio y el codominio. Para las que no, señala qué criterio no se satisface: existencia, unicidad o ambos.
	\begin{enumcols}
		\item Estudiantes del curso con su fecha de nacimiento.
		\answer Es función, todo estudiante nació en una única fecha. $D_f$: estudiantes del curso. $\Cod_f$: fechas. \\ $f:D_f \to \Cod_f|~ y \text{ es la fecha de nacimiento de } x$.

		\item Estudiantes del curso con las materias que aprobó.
		\answer No cumple la condición de unicidad. Hay estudiantes que aprobaron más de una materia.

		\item Autos con el taller en donde se les hizo un mantenimiento.
		\answer No cumple la condición de unicidad ni de existencia. Hay autos a los que se les ha hecho mantenimiento en varios talleres y hay autos a los que nunca se les ha hecho mantenimiento.

		\item Autos con el taller en donde se les hizo el primer mantenimiento.
		\answer No cumple la condición de existencia. Hay autos a los que nunca se les ha hecho mantenimiento.

		\item Cursos de la UNQ con el aula en que se dictan.
		\answer No cumple la condición de unicidad ni la de existencia. Hay cursos que son dictados en más de un aula y hay cursos virtuales sin aula.

		\item Palomas con la cantidad de plumas que tienen.
		\answer No cumple la condición de unicidad. En distintos días una paloma puede perder o ganar plumas. Notar que si se considera la cantidad de plumas que tiene una paloma en un momento dado, sí es función.

		\item Ciudades de Argentina con la provincia donde están.
		\answer Es función, toda ciudad está en una única provincia. $D_f$: ciudades de la Argentina. $\Cod_f$: provincias. $f:D_f \to \Cod_f|~ y \text{ es la provincia en la que está }  x$.

		\item Ciudades de Argentina con la provincia de la que son capital.
		\answer No cumple la condición de existencia. Hay ciudades que no son capitales de ninguna provincia.

	\end{enumcols}

	\exercise En el conjunto $A = \{a, b, c, d, e\}$ se definen las siguientes relaciones. Identifica cuáles son funciones. Para las que lo sean, evalúa si son inyectivas, sobreyectivas y/o biyectivas. Además, cuando corresponda, proporciona la función inversa por extensión.
	\begin{enumcols}
		\item $R = \{(a,b), (b,c), (c,d)\}$
		\answer No es función en A ya que $d$ y $e$ no tienen imagen.

		\item $R = \{(a,b), (b,c), (c,d), (d,e), (e,a)\}$
		\answer Es función, por lo que la nombraremos $f$ en vez de $R$. Como $I_f=\{a,b,c,d,e\}=A=\Cod_f$, es sobreyectiva. Además, por inspección, es inyectiva. Por lo que es biyectiva. \\ La inversa es $f^{-1}=\{(b,a), (c,b), (d,c), (e,d), (a,e)\}$.

		\item $R = \{(a,b), (b,c), (b,d), (d,e), (e,a)\}$
		\answer No es función ya que $c$ no está relacionado a ningún elemento.

		\item $R = \{(a,a), (b,a), (c,d), (d,a), (e,a)\}$
		\answer Es función, por lo que la nombraremos $f$ en vez de $R$. Como $I_f=\{a,d\} \neq \Cod_f$, no es sobreyectiva. Además, $f(a)=f(b)=f(d)=f(e)$, por lo que no es inyectiva. Por lo que no es biyectiva, ni admite inversa.

		\item $R = \{(a,c), (b,e), (c,a), (d,b), (e,d)\}$
		\answer Es función, por lo que la nombraremos $f$ en vez de $R$. Como $I_f=\{a,b,c,d,e\}=A=\Cod_f$, es sobreyectiva. Además, por inspección, es inyectiva. Por lo que es biyectiva. \\ La inversa es $f^{-1}=\{(c,a), (e,b), (a,c), (b,d), (d,e)\}$.
	\end{enumcols}

	\exercise Estudia cada función dada para determinar si es inyectiva y sobreyectiva, proporcionando justificación. Para aquellas que resulten biyectivas, describe la función inversa.
	\begin{enumcols} 
		\item La función que indica la fecha de nacimiento de cada estudiante de la Universidad, donde el codominio son los días desde el 1ro de enero de 1800.
		\answer La función relaciona estudiante $\to$ fecha. La función no es sobreyectiva ya que no hay personas que hayan nacido el 1ro de enero de 1800. Además, no es inyectiva ya que hay personas que nacieron el mismo día.
		
		\item La función del número de vagón en el que está cada pasajero de un tren que no tiene vagones vacíos.
		\answer La función relaciona pasajero $\to$ vagón. Es sobreyectiva ya que no hay vagones vacíos. Sin embargo, no es inyectiva ya que hay personas que están en el mismo vagón.

		\item La función del número de asiento de los pasajeros de un vuelo. Pensar en dos casos: avión lleno, y avión no lleno.
		\answer La función relaciona pasajero $\to$ asiento. Para el caso del avíon lleno: La función es sobreyectiva ya que no hay asientos vacíos, y es biyectiva ya que dos personas no pueden ocupar el mismo asiento. Por lo tanto se puede definir la función inversa que relaciona asiento $\to$ pasajero. \\ Para el caso del avíon no lleno: La función no es sobreyectiva ya que hay asientos vacíos.

		\item La función que indica el número de DNI de residentes de Argentina, tomando como codominio los naturales.
		\answer La función relaciona residente $\to$ DNI. La función no es sobreyectiva ya que hay números naturales que no son DNI. Sin embargo, es inyectiva ya que hay personas que no hay dos personas con el mismo DNI.

		\item La función que relaciona de las personas que viven en un edificio e indica el departamento en que vive cada una.
		\answer La función relaciona persona $\to$ departamento. La función no es sobreyectiva ya que puede haber departamentos vacíos ni tampoco es inyectiva ya que puede haber dos o más personas que vivan en el mismo departamento.

		\item La función que va de cada provincia de Argentina a su capital, tomando como codominio el conjunto de las ciudades capitales de provincia.
		\answer La función relaciona provincia $\to$ capital. La función es sobreyectiva ya que no hay provincias sin capital y además es inyectiva ya que no hay dos provincias con la misma capital. Por lo tanto es biyectiva y se puede definir la función inversa que relaciona capital $\to$ provincia.

	\end{enumcols}

	\exercise Graficar las siguientes funciones $f: D_{f} \to \R$, indicando el dominio más amplio posible y su imagen. Luego, analizar si es inyectiva, sobreyectiva y/o biyectiva. En caso de que sea posible, definir la inversa.
	\begin{enumcols}[2] 

		\item $f(x)=2x-1$
		\answer $D_f=\R$. La función $f(x)$ es biyectiva (inyectiva y sobreyectiva). \\ La inversa es $f^{-1}:\R \to \R ~|~ f^{-1}(x)=\f{x+1}{2}$. 

		\item $f(x)=7-3x$
		\answer $D_f=\R$. La función $f(x)$ es biyectiva (inyectiva y sobreyectiva). \\ La inversa es $f^{-1}:\R \to \R ~|~ f^{-1}(x)=\f{-x+7}{3}$. 

		\item $f(x)=x^2+4x-3$
		\answer $D_f=\R$. La función no es sobreyectiva ya que $I_f=[-7,+\infty)\neq \Cod_f$. Tampoco es inyectiva ya que existen dos valores de $x$ para los que $f(x)=0$.

		\item $f(x)=x^2-4x+4$
		\answer $D_f=\R$. La función no es sobreyectiva ya que $I_f=[0,+\infty)\neq \Cod_f$. Tampoco es inyectiva ya que existen dos valores de $x$ para los que $f(x)=1$.

		\item $f(x)=x^2-5$
		\answer $D_f=\R$. La función no es sobreyectiva ya que $I_f=[-5,+\infty)\neq \Cod_f$. Tampoco es inyectiva ya que existen dos valores de $x$ para los que $f(x)=0$.

		\item $f(x)=-x^2+1$
		\answer $D_f=\R$. La función no es sobreyectiva ya que $I_f=(-\infty,1)\neq \Cod_f$. Tampoco es inyectiva ya que existen dos valores de $x$ para los que $f(x)=0$.

		\item $f(x)=3(x+1)(x-1)$
		\answer $D_f=\R$. La función no es sobreyectiva ya que $I_f=[-3,+\infty)\neq \Cod_f$. Tampoco es inyectiva ya que existen dos valores de $x$ para los que $f(x)=0$.

		\item $f(x)=(x-1)^2-16$
		\answer $D_f=\R$. La función no es sobreyectiva ya que $I_f=[-16,+\infty)\neq \Cod_f$. Tampoco es inyectiva ya que existen dos valores de $x$ para los que $f(x)=0$.

		\item $f(x)=4(x+1)^2-4$
		\answer $D_f=\R$. La función no es sobreyectiva ya que $I_f=[-4,+\infty)\neq \Cod_f$. Tampoco es inyectiva ya que existen dos valores de $x$ para los que $f(x)=0$.

	\end{enumcols}

	\exercise Graficar las siguientes funciones $f: \R \to \R$ e indicar su imagen.
	\begin{enumcols}[3]
		\item $f(x)=\SEL{2x-1 & si ~x\leq 1\\ x^2 & si ~x>1 }$
		\answer $I_f=\R$. \\ \img{6cm}{img/piece1.png}
		\item $f(x)=\SEL{-x & si ~~x\leq 0\\ x  & si ~0<x\leq 3 \\ -x & si ~x>3}$
		\answer $I_f=(-\infty,3)\union [0,+\infty)$. \\ \img{6cm}{img/piece2.png}
		\item $f(x)=\SEL{3x-2 & si ~~x\leq 2 \\ x^2 & si ~~2<x\leq 3 \\ \f{x+6}{9} & si ~x>3}$
		\answer $I_f=\R$. \\ \img{6cm}{img/piece3.png}
		\item $f(x)=\SEL{x^2+7 & si ~~x\leq 2\\ x+4 & si ~~x>2}$
		\answer $I_f=(6,+\infty)$. \\ \img{6cm}{img/piece4.png}
		\item $f(x)=|x|$
		\answer $I_f=[0,+\infty)$. \\ \img{6cm}{img/piece5.png}
	\end{enumcols}

	\exercise Definir las funciones correspondientes a los siguientes los gráficos, e indicar si son inyectivas, sobreyectivas y/o biyectivas.
	\begin{enumcols}[2]
		\item \img{55mm}{img/func1.png}
		\answer $f: \R \to \R ~|~ f(x)=\SEL{-x-1 & si~ x\leq0 \\x-1 & si~ 0<x\leq 3 \\ 2 & si~ 3<x\leq4 \\ \frac{1}{2}x & si~ 4<x}$

		\item \img{55mm}{img/func2.png}
		\answer $f: \R \to \R ~|~ f(x)=\SEL{1 & si~ x\leq-1\\x^2 & si~ -1<x< 1 \\ -1 & si~ 1\leq x}$

		\item \img{55mm}{img/func3.png}
		\answer $f: \R \to \R ~|~ f(x)=\SEL{x+1 & si~ x\leq1 \\(x-1)^2-1 & si~ 1<x}$

		\item \img{55mm}{img/func4.png}
		\answer $f: [0,24] \to \R ~|~ f(x)=\SEL{x & si~ x\in [0,10)\setminus\{5\} \\10 & si~ x\in\{5\}\cup[10,15) \\5 & si~ x\in[15,24]}$

	\end{enumcols}

	\exercise Definir una función que describa la situación. Indicar el dominio, codominio e imagen y graficarla.
	\begin{enumcols} 
		\item Un negocio mayorista ofrece la siguiente oferta sobre un tipo de galletitas: hasta 5 paquetes se venden a 6 pesos el paquete; pasando los 5 paquetes hasta los 10, 5 pesos por paquete adicional; pasando los 10 paquetes, 3 pesos por paquete adicional. Por ejemplo, si una persona compra 12 paquetes, paga (5.6)+(5.5)+(2.3) = 61 pesos.
		\answer $f: \N \to \N ~|~ f(x)=\SEL{6x & si~ 1\leq x\leq 5 \\30+5x & si~ 6\leq x\leq 10 \\55+3x & si~ 11\leq x}$

		\item Otro negocio mayorista ofrece una oferta distinta: hasta 5 paquetes se venden a 6 pesos el paquete; entre 6 y 10 paquetes se vende a 5 pesos el paquete; a partir de 11 		paquetes, se vende a 4.5 pesos por paquete. Por ejemplo, si una persona compra 13 paquetes, paga 13 x 4.5 = 58.5 pesos.
		\answer $f: \N \to \Q ~|~ f(x)=\SEL{6x & si~ 1\leq x\leq 5 \\5x & si~ 6\leq x\leq 10 \\ \f{9}{2}x & si~ 11\leq x}$

		\item El mismo mayorista anterior pero redondeando como no tiene monedas inferiores a un peso para dar vuelto se ve obligado a redondear para abajo el valor cobrado
		\answer $f: \N \to \N ~|~ f(x)=\SEL{6x &si~ 1\leq x\leq 5 \\5x & si~ 6\leq x\leq 10 \\ \lfloor\frac{9}{2}x\rfloor & si~ 11\leq x}$.

		\item Otro negocio mayorista vende las galletitas sueltas por peso y no por paquete. Ofrece lo siguiente: hasta 3kg se vende a 30 pesos el kilo; más de 3kg y hasta 6kg se vende a 25 pesos el kilo; más de 6kg se vende a 20 pesos el kilo. Por ejemplo, si una persona compra 7kg y paga 7 x 20 = 140 pesos. Debido a que el proveedor cobra digitalmente, se puede pagar el monto exacto sin redondear.
		\answer $f: \R^{+}_{0} \to \R^{+}_{0} ~|~ f(x)=\SEL{30x & si~ 0\leq x\leq 3 \\25x & si~ 30 < x\leq 6 \\ 20x & si~ 30 < x}$

	\end{enumcols}

	\exercise Considerando la gráfica de $f: \R^{+}_0 \to \R ~|~ f(x)=\sqrt{x}$, graficar las siguientes funciones $g: D_{f} \to \R$, indicando el dominio más amplio posible e imagen.
	\begin{enumcols}[2]
		\item $g(x)=\sqrt{-2x}$
		\answer La función es $g(x)=f(-2x)$. $D_g=\left(-\infty,0\right]$. $I_g=\left[0,+\infty\right)$ \\ \img{5cm}{img/raiz1.png}
		\item $g(x)=2\sqrt{x+1}$
		\answer La función es $g(x)=2f(x+1)$. $D_f=\left[-1,+\infty\right)$. $I_f=\left[0,+\infty\right)$ \\ \img{5cm}{img/raiz2.png}
		\item $g(x)=\sqrt{4x}+5$
		\answer La función es $g(x)=f(4x)+5$. $D_f=\left[0,+\infty\right)$. $I_f=\left[5,+\infty\right)$ \\ \img{5cm}{img/raiz3.png}
		\item $g(x)=-\sqrt{x}+3$
		\answer La función es $g(x)=-f(x)+3$. $D_f=\left[0,+\infty\right)$. $I_f=\left(-\infty,3\right]$ \\ \img{5cm}{img/raiz4.png}
	\end{enumcols}
	
	\exercise Considerando la gráfica de $f: \R\setminus\{0\} \to \R ~|~ f(x)=\f{1}{x}$, graficar las siguientes funciones $g: D_{f} \to \R$, indicando el dominio más amplio posible e imagen.
	\begin{enumcols}[2]
		\item $g(x)=\f{2}{x}$
		\answer La función es $g(x)=2f(x)$. $D_g=\R\setminus\{0\}$. $I_g=\R\setminus\{0\}$. \\ \img{5cm}{img/homografica1.png}.  
		\item $g(x)=\f{1}{3x}-2$
		\answer La función es $g(x)=f(3x)-2$. $D_g=\R\setminus\{0\}$. $I_g=\R\setminus\{-2\}$.  \\ \img{5cm}{img/homografica2.png}.
		\item $g(x)=\f{x+1}{x+2}$
		\answer Reformulamos algebraicamente: $\f{x+1}{x+2}=\f{x+2-1}{x+2}=\f{x+2}{x+2}-\f{1}{x+2}=-\f{1}{x+2}+1$. \\ La función es $g(x)=-f(x+2)+1$. $D_g=\R\setminus\{-2\}$. $I_g=\R\setminus\{1\}$. \\ \img{5cm}{img/homografica3.png}.  
		\item $g(x)=\f{2x+3}{4x-1}$
		\answer Reformulamos algebraicamente: $\f{2x+3}{4x-1}=\f{1}{2} \left(\f{4x+6}{4x-1}\right)= \f{1}{2} \left(\f{4x-1}{4x-1}+\f{7}{4x-1}\right) = \f{1}{2} + \f{7}{2(4x-1)} = \f{7}{8\left(x-\frac{1}{4}\right)} + \f{1}{2}$. \\ La función es $g(x)=7.f\left(8\left(x-\frac{1}{4}\right)\right)+ \f{1}{2}$. $D_g=\R\setminus\left\{\f{1}{4}\right\}$. $I_g=\R\setminus\left\{\f{1}{2}\right\}$. \\ \img{5cm}{img/homografica4.png}.  
	\end{enumcols}

	\exercise Considerando la gráfica de las funciones trigonométricas básicas, graficar las siguientes funciones $g: D_{f} \to \R$, indicando el dominio más amplio posible, la imagen y el período.
	\begin{enumcols}[3]
		\item $g(x)=\sin(2x)$
		\answer Considerando $f(x)=\sin(x)$, la función es $g(x)=f(2x)$. $D_g=\R$. $I_g=[-1,1]$. $P=\pi$. \\ \img{7cm}{img/sin1.png}
		\item $g(x)=3\sin(x)$
		\answer Considerando $f(x)=\sin(x)$, la función es $g(x)=3f(x)$. $D_g=\R$. $I_g=[-3,3]$. $P=2\pi$. \\ \img{7cm}{img/sin2.png}
		\item $g(x)=-2\sin(x)$
		\answer Considerando $f(x)=\sin(x)$, la función es $g(x)=-2f(x)$. $D_g=\R$. $I_g=[-2,2]$. $P=2\pi$. \\ \img{7cm}{img/sin3.png}
		\item $g(x)=\sin(x-\pi)$ 
		\answer Considerando $f(x)=\sin(x)$, la función es $g(x)=f(x-\pi)$. $D_g=\R$. $I_g=[-1,1]$. $P=2\pi$. \\ \img{7cm}{img/sin4.png}
		\item $g(x)=\sin(\pi x)+5$
		\answer Considerando $f(x)=\sin(x)$, la función es $g(x)=f(\pi x)+5$. $D_g=\R$. $I_g=[4,6]$. $P=2$. \\ \img{7cm}{img/sin5.png}
		\item $g(x)=\sin(-x)$
		\answer Considerando $f(x)=\sin(x)$, la función es $g(x)=f(-x)$. $D_g=\R$. $I_g=[-1,1]$. $P=2\pi$. \\ \img{7cm}{img/sin6.png}
		\item $g(x)=-2\sin\left(x-\f{\pi}{2}\right)+4$
		\answer Considerando $f(x)=\sin(x)$, la función es $g(x)=-2f\left(x-\f{\pi}{2}\right)+4$. $D_g=\R$. $I_g=[2,6]$. $P=2\pi$. \\ \img{7cm}{img/sin7.png}
		\item $g(x)=2\cos(3x)$
		\answer Considerando $f(x)=\cos(x)$, la función es $g(x)=2f(3x)$. $D_g=\R$. $I_g=[-2,2]$. $P=\f{2\pi}{3}$. \\ \img{7cm}{img/cos1.png}
		\item $g(x)=-\cos\left(x-\f{\pi}{2}\right)$
		\answer Considerando $f(x)=\cos(x)$, la función es $g(x)=-f\left(x-\f{\pi}{2}\right)$. $D_g=\R$. $I_g=[-1,1]$. $P=2\pi$. \\ \img{7cm}{img/cos2.png}
		\item $g(x)=\cos(2x+100\pi)$
		\answer Considerando $f(x)=\cos(x)$, la función es $g(x)=f(2(x+50\pi))$. $D_g=\R$. $I_g=[-1,1]$. $P=\pi$. \\ \img{7cm}{img/cos3.png}
		\item $g(x)=\tan(2x)-1$
		\answer Considerando $f(x)=\tan(x)$, la función es $g(x)=f(2x)-1$. $D_g=\R\setminus\{x\in\R ~|~ x=\f{k\pi}{2} con k\in\Z\}$. $I_g=\R$. $P=\f{\pi}{2}$. \\ \img{6cm}{img/tan1.png}

	\end{enumcols}
	
	\exercise Con base en los criterios dados, redefine las funciones trigonométricas para que sean biyectivas. Luego, determina y grafica la función inversa especificada, estableciendo su dominio e imagen.
	\begin{enumcols}
		\item Redefinir el dominio de $\sin(x)$ priorizando los números cercanos a cero para obtener su inversa, $\arcsin(x)$. 
		\answer Se define $f:\left[-\f{\pi}{2},\f{\pi}{2}\right] \to [-1,1] ~|~ f(x)=\sin(x)$, que es biyectiva, para que tenga inversa \\ $f^{-1}: [-1,1] \to \left[-\f{\pi}{2},\f{\pi}{2}\right] ~|~ f^{-1}(x)=\arcsin(x)$. \\ \img{5cm}{img/arcsin.png}
		\item Redefinir el dominio de $\cos(x)$ priorizando los números positivos para obtener su inversa, $\arccos(x)$.
		\answer Se define $f:[0,\pi] \to [-1,1] ~|~ f(x)=\cos(x)$, que es biyectiva, para que tenga inversa \\ $f^{-1}: [-1,1] \to [0,\pi] ~|~ f^{-1}(x)=\arccos(x)$. \\ \img{5cm}{img/arccos.png}
		\item Redefinir el dominio de $\tan(x)$ priorizando los números cercanos a cero para obtener su inversa, $\arctan(x)$. 
		\answer Se define $f:\left(-\f{\pi}{2},\f{\pi}{2}\right) \to \R ~|~ f(x)=\sin(x)$, que es biyectiva, para que tenga inversa \\ $f^{-1}: \R \to \left(-\f{\pi}{2},\f{\pi}{2}\right) ~|~ f^{-1}(x)=\arctan(x)$. \\ \img{7cm}{img/arctan.png}
	\end{enumcols}

	\exercise Resolver las siguientes ecuaciones hallando todas las soluciones posibles.
	\begin{enumcols}[3]
		\item $2\sin(x)=1$
		\answer Reformulamos la ecuación a $\sin(x)=\f{1}{2}$. En $[0,2\pi)$ uno de los valores es $x=\arcsin\left(\f{1}{2}\right)=\f{\pi}{6}=30\degs$, el otro por simetría con respecto a $90\degs$ es $x=150\degs=\f{5}{6}\pi$. Como esas soluciones se repiten peiodicamente todas el conjunto solución es $S=\left\{x\in\R ~\bigg|~ \left(x=\f{\pi}{6}+2k\pi \text{ con } k \in \Z\right) \lor \left(x=\f{5}{6}\pi+2m\pi \text{ con } m \in \Z\right)\right\}$   
		\item $3\sin(x)=0$
		\answer Es equivalente a $\sin(x)=0$ cuyas soluciones son $\{x\in \R ~|~ x=k\pi \text{ con } k \in \Z\}$
		\item $4\sin^{2}\left(x-\f{\pi}{2}\right)=0$
		\answer Reformulamos la ecuación a $\sin\left(x-\f{\pi}{2}\right)=0$ cuyas soluciones provienen de dezplazar $\f{\pi}{2}$ las soluciones de $\sin(x)=0$. Por lo tanto, el conjunto soluciones es $\left\{x\in \R ~\bigg|~ x=\f{\pi}{2}+k\pi \text{ con } k \in \Z\right\}$ 
		\item $\sin(x)-\f{\sqrt{3}}{2}=0$
		\item $2\cos(-x)=\sqrt{2}$
		\item $20\cos(x)+60=80$
		\answer Reformulamos la ecuación a $\cos(x)=1$, que tiene como conjunto solución a \\$S=\{x\in\R ~|~ x=2k\pi \text{ con } k\in\Z\}$
	\end{enumcols}

	\exercise Considerando las gráficas de las funciones exponencial y logaritmo, graficar las siguientes funciones $g: D_{f} \to \R$, indicando el dominio más amplio posible e imagen.
	\begin{enumcols}[3]
		\item $g(x)=2\ln(x)$
		\answer Considerando $f(x)=\ln(x)$, la función es $g(x)=2f(x)$. $D_g=\R^{+}$. $I_g=\R$. \\ \img{5cm}{img/log1.png}
		\item $g(x)=\ln(x)+1$
		\answer Considerando $f(x)=\ln(x)$, la función es $g(x)=f(x)+1$. $D_g=\R^{+}$. $I_g=\R$. \\ \img{5cm}{img/log2.png}
		\item $g(x)=\ln(x-4)$
		\answer Considerando $f(x)=\ln(x)$, la función es $g(x)=f(x-4)$. $D_g=(4,+\infty)$. $I_g=\R$. \\ \img{5cm}{img/log3.png}
		\item $g(x)=-\ln(x)$
		\answer Considerando $f(x)=\ln(x)$, la función es $g(x)=-f(x)$. $D_g=\R^{+}$. $I_g=\R$. \\ \img{5cm}{img/log4.png}
		\item $g(x)=-3\ln(x+1)-2$
		\answer Considerando $f(x)=\ln(x)$, la función es $g(x)=-3f(x+1)-2$. $D_g=(-1,+\infty)$. $I_g=\R$. \\ \img{5cm}{img/log5.png}
		\item $g(x)=\ln(-x)$
		\answer Considerando $f(x)=\ln(x)$, la función es $g(x)=f(-x)$. $D_g=\R^{-}$. $I_g=\R$. \\ \img{5cm}{img/log6.png}
		\item $g(x)=\log_2(x^3)$
		\answer Desarrollo algebraicamente: $\log_2(x^3)=3\log_2(x)=3\f{\ln(x)}{\ln 2}=\f{3}{\ln 2}. \ln(x)$. Considerando $f(x)=\ln(x)$, la función es $g(x)=\f{3}{\ln 2}f(x)$. $D_g=\R^{+}$. $I_g=\R$. \\ \img{5cm}{img/log7.png}
		\item $g(x)=e^{2x}$
		\answer Considerando $f(x)=e^x$, la función es $g(x)=f(2x)$. $D_g=\R$. $I_g=\R^{+}$. \\ \img{5cm}{img/exp1.png}
		\item $g(x)=e^{-x}$
		\answer Considerando $f(x)=e^x$, la función es $g(x)=f(-x)$. $D_g=\R$. $I_g=\R^{+}$. \\ \img{5cm}{img/exp2.png}
		\item $g(x)=3e^{x}$
		\answer Considerando $f(x)=e^x$, la función es $g(x)=3f(x)$. $D_g=\R$. $I_g=\R^{+}$. \\ \img{5cm}{img/exp3.png}
		\item $g(x)=-e^{4x}$
		\answer Considerando $f(x)=e^x$, la función es $g(x)=-f(4x)$. $D_g=\R$. $I_g=\R^{-}$. \\ \img{5cm}{img/exp4.png}
		\item $g(x)=e^{x-2}$
		\answer Considerando $f(x)=e^x$, la función es $g(x)=f(x-2)$. $D_g=\R$. $I_g=\R^{+}$. \\ \img{5cm}{img/exp5.png}
		\item $g(x)=2^{x+1}$
		\answer Reescribimos algebraicamente: $2^{x+1}=e^{\ln\left(2^{x+1}\right)}=e^{(x+1).\ln(2)}$. Considerando $f(x)=e^x$, la función es $g(x)=f(\ln(2)(x+1))$. $D_g=\R$. $I_g=\R^{+}$. \\ \img{5cm}{img/exp6.png}
	\end{enumcols}

	\exercise Resolver las siguientes ecuaciones.
	\begin{enumcols}[2]
		\item $2^x=10$
		\answer $x=\f{\ln 10}{\ln 2}$
		\item $2\ln(x)=4$
		\answer $x=e^2$
		\item $e^{x^2-3}=\f{1}{e^2}$
		\answer $x=\pm1$
		\item $\ln(x)+\ln(x^2)=-\ln(8)$
		\answer $x=\f{1}{2}$
	\end{enumcols}

	\exercise Elige un dominio y codominio para las leyes de asignación dadas, de modo que la función tenga inversa. Luego, encuentra la ley de asignación de esa función inversa.
	\begin{enumcols}[2]
		\item $f(x)=\f{2x-1}{3}$
		\answer $D_f=\R$. $Cod_f=\R$. $f^{-1}(x)=\f{3x+1}{2}$
		\item $g(x)=(x-3)^2-5$
		\answer $D_g=[3,+\infty)$. $Cod_g=[-5,+\infty)$. $g^{-1}(x)=\sqrt{x+5}+3$
		\item $h(x)=\ln(x^2-1)+5$
		\answer $D_h=(1,+\infty)$. $Cod_h=\R$. $h^{-1}(x)=\sqrt{e^{x-5}+1}$
		\item $i(x)=\displaystyle{4+16 e^{-\frac{1}{2}x}}$
		\answer $D_i=\R$. $Cod_i=(4,+\infty)$. $h^{-1}(x)=-2\ln\left(\f{x-4}{16}\right)$
	\end{enumcols}

	\exercise Utilizando las funciones parte entera techo/piso y la parte fraccionaria, resolver las siguientes operaciones.
	\begin{enumcols}[4]
		\item $\floor{2}$
		\answer $\floor{2}=2$
		\item $\floor{2.3}$
		\answer $\floor{2.3}=2$
		\item $\floor{2.9}$
		\answer $\floor{2.9}=2$
		\item $\floor{-6}$
		\answer $\floor{-6}=-6$
		\item $\floor{-6.8}$
		\answer $\floor{-6.8}=-7$
		\item $\floor{-6.1}$
		\answer $\floor{-6.1}=-7$
		\item $\ceil{1.7}$
		\answer $\ceil{1.7}=2$
		\item $\ceil{1.2}$
		\answer $\ceil{1.2}=2$
		\item $\ceil{1}$
		\answer $\ceil{1}=1$
		\item $\ceil{-10.5}$
		\answer $\ceil{-10.5}=-10$
		\item $\ceil{-10}$
		\answer $\ceil{-10}=-10$
		\item $\text{frac}(4.8)$
		\answer $\text{frac}(4.8)=0.8$
		\item $\text{frac}(4)$
		\answer $\text{frac}(4)=0$
		\item $\text{frac}(-5.2)$
		\answer $\text{frac}(-5.2)=0.8$
		\item $\text{frac}(-5)$
		\answer $\text{frac}(-5)=0$
	\end{enumcols}

	\exercise Utilizando la función módulo-$n$, resolver las siguientes operaciones.
	\begin{enumcols}[3]
		\item $5 \text{ mod } 3$
		\answer $5 \text{ mod } 3 = 2$

		\item $18 \text{ mod } 3$
		\answer $18 \text{ mod } 3 = 0$

		\item $-2 \text{ mod } 3$
		\answer $-2 \text{ mod } 3 = 1$

		\item $100 \text{ mod } 7$
		\answer $100 \text{ mod } 7 = 2$

		\item $18 \text{ mod } -3$
		\answer $18 \text{ mod } -3 = 0$

		\item $-2 \text{ mod } -3$
		\answer $-2 \text{ mod } -3 = -2$
	\end{enumcols}

	\exercise Graficar las siguientes funciones, indicar su imagen y estudiar si son inyectivas, sobreyectivas y/o biyectivas.
	\begin{enumcols}[2]

		\item $f: \R \to \Z ~|~ f(x)=\floor{x}$
		\answer $I_f=\Z$. No es inyectiva pues $f(0.5)=f(0.9)=0$. Es sobreyectiva ya que $I_f = Cod_f$. \\ \img{5.5cm}{img/floor.png}

		\item $f: \R \to \Z ~|~ f(x)=\ceil{x}$
		\answer $I_f=\Z$. . No es inyectiva pues $f(0.5)=f(0.9)=1$. Es sobreyectiva ya que $I_f = Cod_f$. \\ \img{5.5cm}{img/ceil.png}

		\item $f: \R \to \R ~|~ f(x)=\text{frac}(x)$
		\answer $I_f=[0,1)$. No es inyectiva pues $f(0.7)=f(1.7)=0.7$. No es sobreyectiva ya que $I_f \neq Cod_f$. \\ \img{6cm}{img/frac.png}

		\item $f: \R \to \R ~|~ f(x)=3(x-\floor{x})$
		\answer Considerando que $f(x)=3 \text{frac}(x)$ realizamos el gráfico. $I_f=[0,3)$. No es inyectiva pues $f(0.9)=f(1.9)=2.7$. No es sobreyectiva ya que $I_f \neq Cod_f$. \\ \img{6cm}{img/frac2.png}

		\item $f: \Z \to \Z ~|~ f(x)= x \text{ mod } 5$
		\answer $I_f=\{0,1,2,3,4\}$. No es inyectiva ya que $f(3)=f(8)=3$. No es sobreyectiva pues $I_f\neq Cod_f$.  \\ \img{5cm}{img/mod1.png}

		\item $f: \Z \to \Z ~|~ f(x)= x \text{ mod } -4$
		\answer $I_f=\{-3,-2,-1,0\}$. No es inyectiva ya que $f(3)=f(7)=-1$. No es sobreyectiva pues $I_f\neq Cod_f$. \\ \img{5cm}{img/mod2.png}
	\end{enumcols}

	\exercise Definir las funciones $g \circ f$ y $f \circ g$ indicando su dominio más amplio y codominio. Evidenciar que las mismas son diferentes comparación de pares ordenados, su gráfico u otro método.
	\begin{enumcols}[2]
		\item $f(x)=3x$ ~y~ $g(x)=x-1$
		\answer $f\circ g: \R \to \R ~|~ (f\circ g) (x)=3(x-1)$ \\ $g\circ f: \R \to \R ~|~ (g\circ f) (x)=3x-1$ \\ Las funciones son distintas pues $(f\circ g) (3)=6$ mientras que $(g\circ f) (3)=8$ 

		\item $f(x)=\floor{x}$ ~y~ $g(x)=x^2$
		\answer $f\circ g: \R \to \Z ~|~ (f\circ g) (x)=\floor{x^2}$ \\ $g\circ f: \R \to \Z ~|~ (g\circ f) (x)=\floor{x}^2$ \\ Las funciones son distintas pues $(f\circ g) (4.2)=17$ mientras que $(g\circ f) (4.2)=16$ 

		\item $f(x)=2x$ ~y~ $g(x)=\sqrt{x}$
		\answer $f\circ g: [0,+\infty) \to \R ~|~ (f\circ g) (x)=2\sqrt{x}$ \\ $g\circ f: [0,+\infty) \to \R ~|~ (g\circ f) (x)=\sqrt{2x}$ \\ Las funciones son distintas pues $(f\circ g) (2)=2\sqrt{2}$ mientras que $(g\circ f) (2)=2$ 

		\item $f(x)=\sin(x)$ ~y~ $g(x)=3x+4$
		\answer $f\circ g: \R \to \R ~|~ (f\circ g) (x)=2\sin(3x+4)$ \\ $g\circ f: \R \to \R ~|~ (g\circ f) (x)=3\sin(x)+4$ \\ Las funciones son distintas pues $(f\circ g) (0)=0$ mientras que $(g\circ f) (0)=\sin(4)$ 

		\item $f(x)=3x$ ~y~ $g(x)=\SEL{x+3 & si ~~x\leq 6\\ x+5 & si ~~x>6}$
		\answer $f\circ g: \R \to \R ~|~ (f\circ g) (x)=\SEL{3x+9 & si ~~x\leq 6\\ 3x+15 & si ~~x>6}$. \\ $g\circ f: \R \to \R ~|~ (g\circ f) (x)=\SEL{3x+3 & si ~~x\leq 2\\ 3x+5 & si ~~x>2}$ \\ Las funciones son distintas pues $(f\circ g) (1)=12$ mientras que $(g\circ f) (1)=6$ 

		\item $f(x)=|x|$ ~y~ $g(x)=x+4$
		\answer $f\circ g: \R \to \R ~|~ (f\circ g) (x)=|x+4|$ \\ $g\circ f: \R \to \R ~|~ (g\circ f) (x)=|x|+4$ \\ Las funciones son distintas pues $(f\circ g) (-1)=3$ mientras que $(g\circ f) (-1)=5$ 

		\item $f(x)=\log(x)$ ~y~ $g(x)=x^2$
		\answer $f\circ g: \R\setminus\{0\} \to \R ~|~ (f\circ g) (x)=\log(x^2)$ \\ $g\circ f: (0,+\infty) \to \R ~|~ (g\circ f) (x)=\log^2(x)$ \\ Las funciones son distintas pues no tienen el mismo dominio.

		\item $f(x)=e^{x}$ ~y~ $g(x)=|x|$
		\answer $f\circ g: \R \to \R ~|~ (f\circ g) (x)=e^{|x|}$ \\ $g\circ f: \R \to \R ~|~ (g\circ f) (x)=|e^x|$ \\ Las funciones son distintas pues $(f\circ g) (-1)=\frac{1}{e}$ mientras que $(g\circ f) (-1)=e$ 

		\item $f(x)=\f{1}{x}$ ~y~ $g(x)=\cos(x)$
		\answer $f\circ g: \R\setminus\{x |~ x=\frac{\pi}{2}+k\pi \text{ con } k\in\Z\} \to \R ~|~ (f\circ g) (x)=\f{1}{\cos(x)}$ \\ $g\circ f: \R\setminus\{0\} \to \R ~|~ (g\circ f) (x)=\cos\left(\f{1}{x}\right)$ \\ Las funciones son distintas pues no tienen el mismo dominio

		\item $f(x)=\sin(x)$ ~y~ $g(x)=2x$
		\answer $f\circ g: \R \to \R ~|~ (f\circ g) (x)=\sin(2x)$ \\ $g\circ f: \R \to \R ~|~ (g\circ f) (x)=2\sin(x)$ \\ Las funciones son distintas pues $(f\circ g) \left(\frac{\pi}{2}\right)=0$ mientras que $(g\circ f) \left(\frac{\pi}{2}\right)=2$ 

		\item $f(x)=\sqrt{x}$ ~y~ $g(x)=3x-1$
		\answer $f\circ g: \left[\frac{1}{3},+\infty\right) \to \R ~|~ (f\circ g) (x)=\sqrt{3x-1}$ \\ $g\circ f: [0,+\infty) \to \R ~|~ (g\circ f) (x)=3\sqrt{x}-1$ \\ Las funciones son distintas pues no tienen el mismo dominio.

		\item $f(x)=\ln(x)$ ~y~ $g(x)=x-1$
		\answer $f\circ g: (1,+\infty) \to \R ~|~ (f\circ g) (x)=\ln(x-1)$ \\ $g\circ f: (0,+\infty) \to \R ~|~ (g\circ f) (x)=\ln(x)-1$ \\ Las funciones son distintas pues no tienen el mismo dominio

		\item $f(x)=\displaystyle{e^{-x}}$ ~y~ $g(x)=\ln(x)$
		\item $f(x)=\sqrt{x}$ ~y~ $g(x)=\ln(x-1)$
		\item $f(x)=\ln(x)$ ~y~ $g(x)=x^2+4$

		\item $f(x)=\sin(x)$ ~y~ $g(x)=x^2-2x+1$
		\answer $f\circ g: \R \to \R ~|~ (f\circ g) (x)=\sin(x^2-2x+1)$ \\ $g\circ f: \R \to \R ~|~ (g\circ f) (x)=\sin^2(x)-2\sin(x)+1$ \\ Las funciones son distintas pues $(f\circ g) (1)=0$ mientras que $(g\circ f) (1)=\sin^2(1)-2\sin(1)+1\simeq0.0251$ 

	\end{enumcols}

	\exercise Dadas las funciones definidas de $\R$ a $\R$: $f(x)=x^2$, ~~$g(x)=\f{x}{4}$ ~~ y ~~ $h(x)=x+3$. Calcular las composiciones indicadas, pudiendo restringir los dominios de las funciones dominios para obtenerlas.
	\begin{enumcols}[3]
		\item $f \circ g \circ h$. 
		\answer $(f \circ g \circ h)(x)=\left(\f{x+3}{4}\right)^2$

		\item $h \circ g \circ f$
		\answer $(h \circ g \circ f)(x)=\f{x^2}{4}+3$

		\item $f \circ h \circ h$
		\answer $(f \circ h \circ h)(x)=(x+6)^2$

		\item $f \circ g \circ g$
		\item $f \circ f$
		\item $f \circ h \circ g^{-1}$
		\answer Considerando que $g^{-1}(x)=4x$, entonces $(f \circ h \circ g^{-1})(x)=(4x+3)^2$ 

		\item $g \circ h^{-1} \circ h^{-1}$
		\answer Considerando que $h^{-1}(x)=x-3$, entonces $(g \circ h^{-1} \circ h^{-1})(x)=\f{x-6}{4}$

		\item $g^{-1} \circ f \circ g^{-1}$
		\item $f \circ f^{-1} \circ f$
		\answer Redefinimos $f: [0,+\infty) \to [0,+\infty)$ para que tenga inversa. Luego $(f \circ f^{-1} \circ f)(x)=f(x)=x^2$
	\end{enumcols}

	\exercise Definir la función $f$ para que tenga la ley de asignación indicada a través de componer las siguientes funciones reales o sus inversas: ~~ $g(x)=x^2$, ~~ $h(x)=|x|$, ~~ $i(x)=x+1$, ~~ $j(x)=-x$  ~~y ~~ $k(x)=2x$.
	\begin{enumcols}[3]
		\item $f(x)=-x^2$
		\answer $j \circ g (x)$
		\item $f(x)=2x^2$
		\answer $k\circ g(x)$
		\item $f(x)=2|x|$
		\answer $k\circ h(x)$
		\item $f(x)=\left(\f{x}{2}\right)^2$
		\answer $g\circ k^{-1}(x)$
		\item $f(x)=-|x|$
		\answer $j\circ h(x)$
		\item $f(x)=\f{|x|}{2}$
		\answer $k^{-1}\circ h(x)$
		\item $f(x)=|x|+2$
		\answer $i\circ i\circ h(x)$
		\item $f(x)=(4x)^2$
		\answer $g\circ k\circ k(x)$
		\item $f(x)=-|x+1|$
		\answer $j\circ h\circ i(x)$
		\item $f(x)=\f{x^2}{2}$
		\answer $k^{-1}\circ g(x)$
		\item $f(x)=-2|x|$
		\answer $j\circ k\circ h(x)$
		\item $f(x)=x^2+2$
		\answer $i\circ i\circ g(x)$
		\item $f(x)=|x+2|$
		\answer $h\circ i\circ i(x)$
		\item $f(x)=(x+1)^2-1$
		\answer $i^{-1}\circ g\circ i(x)$
		\item $f(x)=|x+1|+2$
		\answer $i\circ i\circ h\circ i(x)$
		\item $f(x)=-|x|-2$
		\answer $j\circ i\circ i\circ h(x)$
		\item $f(x)=(x-1)^2-1$
		\answer $i^{-1}\circ g\circ i^{-1} (x)$
		\item $f(x)=-\f{|x|}{2}$
		\answer $j\circ k^{-1}\circ h(x)$
		\item $f(x)=|x|-2$
		\answer $i^{-1}\circ i^{-1}\circ h(x)$
		\item $f(x)=-\left|\f{x}{2}\right|$
		\answer $j^{-1}\circ h\circ k^{-1}(x)$

	\end{enumcols}

	\exercise Resolver los siguientes ejercicios sobre funciones de orden superior:
	\begin{enumcols}
		\item Se denomina $A$ al conjunto de todas las funciones cuadráticas $f: \R \to \R ~|~ f(x)=ax^2+bx+c$ y se define la función $F: A \to \R$ de la forma $\Delta=F(f) \Eq $ "$\Delta$ es el discriminante de $f$". Calcular $F(x^2+x-2)$, $F(x^2+5x+4)$ y $F(3x^2-x)$. Realizar un diagrama de Venn enlazado para representar las imágenes calculadas. Analizar si la función es inyectiva.
		\answer $F(x^2+x-2)=1^2-4.1.(-2)=9$ \\ $F(x^2+5x+4)=5^2-4.1.4=9$ \\ $F(3x^2-x)=(-1)^2-4.3.0=1$\\ La función no es inyectiva pues $F(x^2+x-2)=F(x^2+5x+4)=9$.

		\item Sea $F$ una función que toma un par ordenado $(a,b)$ y una función real $f$, y devuelve otro par ordenado: se define $F$ de forma que $F((x,y),f)=(f(x),f(y))$. Por ejemplo, usando la función $f(x)=x^2$ obtenemos $F((1,-4),f)=(1,16)$. Calcular $F((4,100),g)$ con $g(x)=\sqrt{x}$. Por otra parte, definir una función $h$ para que $F((2,5),h)=(3,6)$.
		\answer $F((4,100),g)=(\sqrt{4},\sqrt{100})=(2,10)$ \\ Se define $h(x)=x+1$ para que $F((2,5),h)=(2+1,5+1)=(3,6)$

		\item La función $F$ toma funciones reales $f$ y devuelve otra función $g$ de la forma $g=F(f)=f \circ f$. Considerando $h(x)=x+5$, ~ $i(x)=3x$ ~ y ~ $j(x)=x^2+x$, calcular $F(h)$, $F(i)$, $F(j)$ y representar lo en un diagrama de Venn enlazado.
		\answer $F(h)=h \circ h$, es decir, $g: \R \to \R ~|~ g(x)=x+10$. \\ $F(i)=i \circ i$, es decir, $g: \R \to \R ~|~ g(x)=9x$. \\ $F(j)=j \circ j$, es decir, $g: \R \to \R ~|~ g(x)=x^4+2x^3+2x^2+x$. 

		\item Dadas las funciones $f_m: \R \to \R ~|~ f_m(x)=mx+5$ y la función $F: \R \to B ~|~ F(m)= f_{m}$. Obtener $F(2)$ y graficarla. Describir qué elementos contiene el conjunto $B$ (el codominio de $F$).
		\answer $F(2)$ es la función $f: \R \to \R ~|~ f(x)=2x+5$. \\ $B$ es el conjunto $B$ de todas las funciones lineales (representadas por rectas oblicuas).  
	\end{enumcols}

	\exercise La \textit{cota superior asintótica del uso de recursos de un algoritmo} (también llamada \textit{Big O}) indica cuánto aumenta el uso de recursos en función del tamaño de los datos de entrada $n$, medidos como un numero natural. Por ejemplo, asumiendo valores grandes de $n$, un algoritmo caracterizado por $O(n)$ duplicará su tiempo de ejecución cuando se duplique $n$, ya que se comportará como la función $f(n)=n$, que es lineal. Por otra parte, un algoritmo $O(n^2)$ tardará 4 veces más al duplicar $n$ ya que se comportará como la función $f(n)=n^2$, que es cuadrática. Si se tiene tres algoritmos distintos para realizar una tarea, con las siguientes \textit{Big O}: $O(n^2)$, $O(\log n)$ y $O(2^n)$. ¿Qué algoritmo es el idóneo para trabajar con una gran cantidad de datos de entrada? Para responder esto, realizar un gráfico de las funciones $n^2$, $\log n$ y $2^n$. La mejor será aquella que crece con menor rapidez en función de $n$.

\end{enumerate}

\end{document}