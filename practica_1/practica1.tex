\documentclass{template_practica}

\begin{document}

\practiceheader{Práctica 1: Lógica proposicional y de primer orden}{Comisión: Rodrigo Cossio-Pérez y Leonardo Lattenero}

\textbf{EJERCICIOS JR.}

\begin{enumerate}

	\exercise Reconocer las funciones del lenguaje e indicar qué casos son proposiciones. Si son proposiciones, indicar su valor de verdad.
	\begin{enumcols}[2]
		\item Hoy es martes 
		\answer Función informativa, es proposición. El valor de verdad dependerá del día.

		\item ¡Auch!
		\answer Función exclamativa, no es una proposición. Notar que gritar "Me acabo de lastimar" sí sería es una proposición.

		\item Fuí a visitar las playas de la costa de Buenos Aires 
		\answer Función informativa, es proposición. El valor de verdad depende de la persona que lo lea.

		\item Vaya a visitar las playas de la costa de Buenos Aires
		\answer Función imperativa, no es proposición.

		\item ¿Venís hoy a clase?
		\answer Función interrogativa, no es proposición.

		\item 5 es múltiplo de 2 
		\answer Función informativa, es proposición. Es falsa.

		\item Si estudiás continuamente, siempre vas a aprender cosas nuevas 
		\answer Función informativa, es proposición. La proposición es verdadera pero tiene detalles discutibles. Para afirmar que para alguna persona es falsa deberiamos pensar en una persona que estudie continuamente pero que no aprenda cosas nuevas.

		\item $x=6$ y $x+4=10$ 
		\answer Función informativa, es proposición. Aunque sean simbolos matemáticos, esta proposición se puede leer como "El número $x$ es seis y el número $x$ sumado a 4 es igual a 10". Lo que hace que esto sea una proposición es que estamos afirmando las igualdades, que el número es igual a 6 y que sumado a 4 es igual a 10. La proposición es V.

		\item $x+4$ 
		\answer No es proposición. Se puede leer como "El número $x$ más 4" y no se afirma nada al respecto. Es similar al inciso de "En las aulas de la UNQ".

		\item $x^2-4$ no tiene raíces reales 
		\answer Función afirmativa, es una proposición. Para averiguar su valor de verdad debemos buscar las raíces de la parábola, es decir, $x^2-4=0$. Mediante la fórmula de Bhaskara (la resolvente cuadrática) podemos obtener que $x=2$ o $x=-2$. $2$ y $-2$ son números reales. Por lo que la proposición es falsa. 

	\end{enumcols}


	\exercise Simbolizar las siguientes proposiciones, indicar el diccionario de lenguaje cuando sea necesario y cuando amerite hallar una proposición equivalente considerando las propiedades y definiciones de los operadores.
	\begin{enumcols}[2]

		\item La manzana es una fruta y la lechuga una verdura. 
		\answer $p \land q$ con $p$:"\textit{La manzana es una fruta}" y $q$:"\textit{La lechuga una verdura}". Un equivalente es $q \land p$:"\textit{La lechuga es una verdura y la manzana es una fruta}".

		\item No está lloviendo. 
		\answer $\neg p$ con $p$:"\textit{Está lloviendo}".

		\item Si estás en la estación Bernal, estás cerca de la UNQ 
		\answer $p\then q$ con $p$:"\textit{Estás en la estación Bernal}" y $q$:"\textit{Estás cerca de la UNQ}". Un equivalente es $\neg q \then \neg p$: "\textit{Si no estás cerca de la UNQ, no estás en la estación de Bernal}".

		\item No es buen deportista pero sus notas son excelentes.
		\answer $\neg p \land q$ con $p$:"\textit{Es buen deportista}" y $q$:"\textit{Sus notas son excelentes}". Un equivalente es $q \land \neg p$:"\textit{Sus notas son excelentes pero no es buen deportista}". Notar que linguísticamente las frases son distintas porque se le da distinta importancia relevancia al orden de las proposiciones, en la primera se rescata que es buen estudiante, en la segunda se condena que es mal deportista. En este caso la lógica no logra mostrar estas diferencias linguísticas. \href{https://youtu.be/HXzyX5XGPp8?t=503}{Resolución por Tu Profe en Linea}.

		\item El caballo está galopando, o se detuvo y relinchó.
		\answer $p \lor (q \land r)$ con $p$:"\textit{El caballo está galopando}", $q$:"\textit{El caballo de detuvo}" y $r$:"\textit{El caballo relinchó}". \href{https://youtu.be/TgwraosKUuY?t=70}{Resolución por Christian Omar Arias López}.

		\item En la UNQ hay wi-fi si y sólo si hay luz.
		\answer $p\eq q$ con $p$:"\textit{En la UNQ hay wi-fi}" y $q$:"\textit{Hay luz}". Un equivalente es $(p \then q) \land (q \then p)$:"\textit{Si en la UNQ hay wi-fi, hay luz. Si hay luz, en la UNQ hay wi-fi}".

		\item Una de dos: o mañana lloverá o estará soleado. 
		\answer $p \xor q$ con $p$:"\textit{Mañana lloverá}" y $q$:"\textit{Mañana estará soleado}". Un equivalente es $(p \land \neg q) \land (\neg p \land q)$:"\textit{Mañana lloverá y no estará soleado, o bien, mañana estará soleado y no lloverá}". 

		\item La materia se aprueba promocionando los parciales, aprobando el final o dando exitosamente el exámen libre. 
		\answer $p \lor q \lor r$ con $p$:"\textit{La materia se aprueba promocionando los parciales}", $q$:"\textit{La materia se aprueba aprobando el final}" y $r$:"\textit{La materia se aprueba dando exitosamente el exámen libre}". También se acepta $p \lor q \lor r \then s$ con el diccionario $p$:"\textit{Promociono los parciales}", $q$:"\textit{Apruebo el final}", $r$:"\textit{Doy exitosamente el exámen libre}" y $s$:"\textit{Apruebo la materia}". Esta simbología captura mejor la situación, pero por otro lado tuvimos que llevarla a la primera persona (yo) y podria haber sido cualquier estudiante. Para mejorar esto se puede utilizar lógica de predicados, que se verá al final de la unidad. Dejo la respuesta para que se revise más adelante. $\forall x: p(x) \lor q(x) \lor r(x) \then s(x)$ con el diccionario $p(x)$:"$x$ \textit{promociona los parciales}", $q(x)$:"$x$ \textit{aprueba el final}", $r(x)$:"$x$ \textit{da exitosamente el exámen libre}" y $s(x)$:"$x$ \textit{aprueba la materia}".

		\item El pan no levará si le ponés mucha sal. Tampoco levará si lo dejás en un lugar frío. 
		\answer $(q\then \neg p)  \land  (r\then \neg p)$ con $p$:"\textit{El pan levará}", $q$:"\textit{Le pones mucha sal}" y $r$:"\textit{Lo dejas en un lugar frío}". Un equivalente es $(q \lor r) \then \neg p$:"\textit{Si le pones mucha sal o lo dejas en un lugar frío, el pan no levará}".

		\item Si te tomás el 324 te deja cerca de la UNQ, si te tomás el 65 no. 
		\answer $( p\then q ) \land ( r\then \neg q )$ con $p$:"\textit{Te tomas el 324}", $q$:"\textit{Te deja cerca de la UNQ}" y $p$:"\textit{Te tomás el 65}".  

	\end{enumcols}

	\exercise Determinar si las siguientes proposiciones son tautologías, contradicciones o contingencias mediante la realización de su tabla de verdad. 
	\begin{enumcols}[2]
		\item $(\neg p \lor  q) \land  \neg q$
		\answer Contingencia. \href{https://www.wolframalpha.com/input?i=%28not+p+or+q%29+and+not+q}{Resolución}.

		\item $p \land  \neg p$
		\answer Contradicción. \href{https://www.wolframalpha.com/input?i=truth+table+of%3A+p+and+not+p}{Resolución}.

		\item $p \land  F_0 \lor  q$
		\answer Contingencia.\href{https://www.wolframalpha.com/input?i=p+and+r+or++q}{Resolución, ver solo donde $r$ es F}.

		\item $(p \then  q) \then  r$
		\answer Contingencia. \href{https://www.wolframalpha.com/input?i=%28p+%3D%3E+q%29+%3D%3E+r}{Resolución}.

		\item $p \eq (q \xor  r)$
		\answer Contingencia. \href{https://www.wolframalpha.com/input?i=p+%3C%3D%3E+%28q+xor++r%29}{Resolución}.

		\item $(p \nand q) \nor r$
		\answer Contingencia. \href{https://www.wolframalpha.com/input?i=%28p+nand+q%29+nor+r}{Resolución}

		\item $((\neg p \lor q) \then (r \lor p)) \lor (p \then q)$
		\answer Es tautología. \href{https://youtu.be/k-amMQR3oMc}{Resolución por ProfeGuille}.

	\end{enumcols}

	\exercise Averiguar si las siguientes proposiciones son equivalentes mediante tablas de verdad y/o reglas de equivalencia.  Dar un ejemplo en lenguaje natural que lo evidencie. Nota: no siempre el mismo método.
	\begin{enumcols}[2]

		\item $\neg (\neg p) \Eq  p$
		\answer La \href{https://www.wolframalpha.com/input?i=truth+table%3A+not+%28not+p%29+%3C%3D%3E+p}{tabla de verdad} revelea que es una tautología. Por lo tanto, son equivalentes. Esta regla de equivalencia se llama Doble Negación.

		\item $p\then q \Eq \neg q\then \neg p$
		\answer La \href{https://www.wolframalpha.com/input?i=%28p+%3D%3E+q%29+%3C%3D%3E+not+q+%3D%3E+not+p}{tabla de verdad} revela que es una tautología. Por lo tanto, son equivalentes. Esta regla de equivalencia se llama Definición de la Implicación.

		\item $p\then q \Eq \neg p\then \neg q$
		\answer La \href{https://www.wolframalpha.com/input?i=%28p+%3D%3E+q%29+%3C%3D%3E+%28not+p+%3D%3E+not+q%29}{tabla de verdad} revela que es una contingencia. Por lo tanto, NO son equivalentes.

		\item $p \land  (p \lor  q) \Eq  p$
		\answer La \href{https://www.wolframalpha.com/input?i=%28p+and+%28p+or++q%29%29+%3C%3D%3E++p}{tabla de verdad} revela que es una tautología. Esta regla de equivalencia se llama Absorción Total.

		\item $p\then q	\Eq (p \land  \neg q)\then F_0$
		\answer La \href{https://www.wolframalpha.com/input?i=%28p+%3D%3E+q%29+%3C%3D%3E+%28%28p+and+not+q%29+%3D%3E+r%29}{tabla de verdad (ver dónde r es F)} revela que es una tautología. Por lo tanto, son equivalentes. Esta regla de equivalencia se llama Reducción al absurdo.

		\item $(p\land q)\then r \Eq p\then (q\then r) $
		\answer La \href{https://www.wolframalpha.com/input?i=%28%28p+and+q%29+%3D%3E+r%29%3C%3D%3E+%28p+%3D%3E+%28q+%3D%3Er%29%29}{tabla de verdad} revela que es una tautología. Por lo tanto, son equivalentes. Esta regla de equivalencia se llama Exportación.

		\item $p\lor q \then  r	\Eq (p\then r) \land  (q\then r)$
		\answer La \href{https://www.wolframalpha.com/input?i=%28%28p+or+q+%29+%3D%3E++r%29+%3C%3D%3E+%28+%28p+%3D%3E+r%29+and+%28q+%3D%3E+r%29+%29}{tabla de verdad} revela que es una tautología. Por lo tanto, son equivalentes. Esta regla de equivalencia se llama Demostración por casos.

	\end{enumcols}

	\exercise Simplificar las siguientes expresiones
	\begin{enumcols}[2]

		\item $\neg ((p \land \neg q)\then p) \lor q$
		\answer La expresion más simple es $q$. \href{https://youtu.be/BOydu7cpv70}{Resolución por Tu Profe en Linea}.

		\item $(\neg q \then \neg p) \land \neg (\neg p \then \neg q)$
		\answer $\neg p \land q$. \href{https://youtu.be/p005yi28rgk?t=737}{Resolución por ProfesorTriquero}.

		\item $(\neg q \lor p) \lor (p \land \neg q)$
		\answer Puede ser $q \then p$, o bien $p \lor \neg q$. \href{https://youtu.be/p005yi28rgk?t=995}{Resolución por ProfesorTriquero}.

		\item ($p \land ( p \then q)) \then q$
		\answer Es equivalente a $T_0$ (es una tautología). \href{https://youtu.be/BOydu7cpv70?t=586}{Resolución por Tu Profe en Linea}.

		\item $\neg ((\neg p \land q) \then p) \lor q$
		\answer La expresión mas simple es $q$. \href{https://youtu.be/KyIdCTWZuJ8}{Resolución por ProfeGuille}.

		\item $(\neg p \then q) \land \neg (q \then \neg p)$
		\answer $p \land q$. \href{https://youtu.be/shOOoVRqKcA}{Resolución por ProfeGuille}.

	\end{enumcols}

	\exercise Escribir en forma normal disyuntiva (DNF) y en forma normal conjuntiva (CNF) las siguientes proposiciones.
	\begin{enumcols}[2]

		\item $p\xor q$
		\answer DNF: $(p\land \neg q) \lor  (\neg p\land q)$ \\ CNF: $(p\lor q) \land  (\neg p\lor \neg q)$

		\item $\neg (p \nand q) \lor \neg p$
		\answer DNF: $(p\land \neg q)$  \\ CNF: $(\neg p\lor \neg q) \land  (p\lor q) \land  (p\lor \neg q)$
		
		\item $p\then  (q\land r)$
		\answer DNF: $(p\land q\land r) \lor  (\neg p\land q\land r) \lor  (\neg p\land q\land \neg r) \lor  (\neg p\land \neg q\land r) \lor  (\neg p\land \neg q\land \neg r)$  \\ CNF: $(\neg p\lor \neg q\lor r) \land  (\neg p\lor q\lor \neg r) \land  (\neg p\lor q\lor r)$

	\end{enumcols}


	\exercise Verificar mediante una tabla de verdad si las siguientes proposiciones son reglas de inferencia. Dar un ejemplo en lenguaje natural que lo evidencie.
	\begin{enumcols}[2]

		\item $p \land q \Then p$
		\answer Es tautología, por lo que es una regla de inferencia. Se la llama \textit{Simplificación}. Ver la \href{https://www.wolframalpha.com/input?i=%28p+and+q%29+%3D%3E+p}{tabla de verdad}.

		\item $p\lor q \Then  q$
		\answer Es contingencia, no es regla de inferencia. Ver la \href{https://www.wolframalpha.com/input?i=%28p+or+q%29+%3D%3E+q}{tabla de verdad}.

		\item $p \land  T_0 \Then  q$
		\answer Contingencia.

		\item $(\neg p \land q) \Then (\neg p \lor q)$
		\answer Es tautología, por lo que es una regla de inferencia que no tiene nombre. \href{https://youtu.be/NZSuHeymu4M?t=382}{Resolución por Tu Profe en Linea}.

		\item $(p\eq q) \Then  (p\then q)$
		\answer Es tautología, por lo que es una regla de inferencia. Se la llama \textit{Elimiaciónn del bicondicional}. Ver la \href{https://www.wolframalpha.com/input?i=%28p%3C%3D%3Eq%29+%3D%3E+%28p%3D%3Eq%29}{tabla de verdad}.

		\item $((p \eq q) \land r ) \Then \neg (q \land r)$
		\answer Es contingencia, no es regla de inferencia. \href{https://youtu.be/NZSuHeymu4M?t=829}{Resolución por Tu Profe en Linea}.

	\end{enumcols}

	\exercise Sin utilizar las tablas de verdad, demostrar si los siguientes razonamientos son válidos o no. 
	\begin{enumcols}[3]

		\item \Reasoning{$\neg p\then p$}{$p$}
		\answer \DReasoning{$\neg p\then p$}{$\neg \neg p \lor p$ ~~~(def. $\then$); $p \lor p$ ~~~(doble $\neg$)}{$p$ ~~~(idempotencia)}

		\item \Reasoning{$p\then q$ ; $\neg q$}{$\neg p$}
		\answer \Reasoning{$p\then q$ ; $\neg q$}{$\neg p$ ~~~(modus tollens)}

		\item \Reasoning{$p\then q$ ; $q\then r$ ; $p$}{$r$}
		\answer \DReasoning{$p\then q$ ; $q\then r$ ; $p$}{$p \then r$ ~~~(tran. del $\then$)}{$r$ ~~~(modus ponens)}

		\item \Reasoning{$\neg p$}{$\neg (p\land q)$}
		\answer \DReasoning{$\neg p$}{$\neg p \lor \neg q$ ~~~(adición)}{$\neg (p\land q)$ ~~~(De Morgan)}

		\item \Reasoning{$p\then r$ ; $q\then r$ ; $r\then s\land t$ ; $p\lor q$}{$s$}
		\answer \DReasoning{$p\then r$ ; $q\then r$ ; $r\then s\land t$ ; $p\lor q$}{$r$ ~~~(elim. disy.);$s \land t$ ~~~(modus ponens)}{$s$ ~~~(simplificación)}

		\item \Reasoning{$p\then q$ ; $q\land r\then s$}{$p\land r\then  s$}
		\answer Aplicamos exportación: \\ \DReasoning{$p\then q$ ; $q\land r\then s$; $p$; $r$}{$q$ ~~~(modus ponens)}{$s$ ~~~(conj. y mod. pon.)}

		\item \Reasoning{$p\then  q$ ; $q\eq r$ ; $\neg r$}{$\neg p$}
		\answer \DReasoning{$p\then  q$ ; $q\eq r$ ; $\neg r$}{$q \then r$ ~~~(elim. $\eq$); $\neg q$ ~~~(mod. tollens)}{$\neg p$ ~~~(mod. tollens)}

		\item \Reasoning{$p \then q$ ; $p \lor s$ ; $\neg q$}{$s$}
		\answer \DReasoning{$p \then q$ ; $p \lor s$ ; $\neg q$}{$\neg p$ ~~~(mod. toll.)}{$s$ ~~~(mod. toll. pon.)}

	\end{enumcols}

	\exercise Averiguar si los siguientes razonamientos son válidos y evaluar su solidez.
	\begin{enumcols}
		\item Por el pronóstico semanal, sabemos que si es martes, llueve y hace frío. Sabemos que es martes. Demostrar que hace frío.
		\answer Razonamiento válido. Usamos $m$:"\textit{es martes}", $l$:"\textit{llueve}", $f$:"\textit{hace frio}". Por lo que el razonamiento será \\ \DReasoning{$m \then l \land f$; $m$}{$l \land f$ ~~~(mod. pon.)}{$f$ ~~~(simpl.)}

		\item El ladrón tenía llave de la puerta o entró por la ventana. Si entró por la ventana, pisoteó las macetas. Las macetas no están pisoteadas. Por lo tanto, el ladrón tenía llave de la puerta.
		\answer Razonamiento válido. Utilizando $l$:"\textit{el ladron tenía llave}", $v$:"\textit{el ladron entró por la ventana}" y $p$:"\textit{el ladron pisoteó las macetas}", se demuestra: \\ \DReasoning{$l \lor v$; $v \then p$; $\neg p$}{$\neg v$ ~~~(mod. tollens)}{$l$ ~~~(mod. toll. ponens)} \\  \href{https://youtu.be/DD5EleyOl-0?t=74}{Planteo por Maria Alicia Piñeiro}.

		\item El producto de dos enteros impares es impar
		\answer Razonamiento válido. Utilizamos los números enteros $x,y \in \Z$: \\ \DReasoning{$x$ es impar; $y$ es impar}{$x=2k+1$ con $k \in \Z$; $y=2m+1$ con $m \in \Z$;$x.y=(2k+1).(2m+1)=2(2km+k+m)+1$ con $(2km+k+m) \in \Z$}{$x.y$ es impar}

		\item La suma de dos números naturales pares es un número natural impar
		\answer Razonamiento inválido. Contraejemplo: $2+4=6$. $2$ y $4$ son pares, pero $6$ no es impar.

		\item Si el cuadrado de un número entero es impar, dicho número es impar. 
		\answer Razonamiento válido. Demostraremos el contrarrecíproco: "\textit{Si un número entero no es impar, el cuadrado de dicho número no es impar}". Utilizamos el numero entero $n \in \Z$: \\ \DReasoning{$n$ no es impar}{$n$ es par; $n=2k$ con $k \in \Z$;$n^2=(2k).(2k)=2(2.k.k)$ con $(2.k.k) \in \Z$;$n^2$ es par}{$n^2$ no es impar}

		\item Si un número entero es múltiplo de 6, entonces dicho número es múltiplo de 2 y de 3.
		\answer Razonamiento válido. Utilizamos el numero entero $n \in \N$: \\ \DReasoning{$n$ es múltiplo de $6$}{$n=k.6$ con $k \in \N$; $n=(k.2).3$ con $2k \in \N$;$n=(k.3).2$ con $3k \in \N$}{$n$ es multiplo de 3 y de 2}

		\item Si un número entero es menor que otro, el primer número es menor o igual que el sucesor del segundo
		\answer Razonamiento válido. \\ \DReasoning{$x<y$}{$y<y+1$ ~~~(prop. suma en $\R$); $x<y+1$ ~~~(transit. del $<$); $(x<y+1) \lor ($x=y+1$)$ ~~~(adición)}{$x \leq y+1$}

		\item Si se cumple que $x^2-x-6>0$, entonces se cumple que $|x| > 1$.
		\answer Razonamiento válido. \\ \DReasoning{$x^2-x-6>0$}{$(x+2)(x-3)>0$ ~~~(factorizo); $(x>3) \lor (x<-2)$ ~~~(prop. desigualdad); $(x>3) \then |x|>3$ ~~~(prop. val. abs.); $(x<-2) \then |x|>2$ ~~~(prop. val. abs.);$(|x|>3) \lor (|x|>2)$   ~~~(dilema constr.); $|x|>3 \then |x|>1$; $|x|>3 \then |x|>1$}{$|x|>1$ ~~~(elim. disy.)}

		\item Si $|x| = |y|$, entonces  $y=x$ o $y=-x$.
		\answer Razonamiento válido. \\ \DReasoning{$|x| = |y|$}{$(x = |y|) \lor (x = -|y|)$ ~~~(prop. val. abs.); $(x = |y|) \lor (-x = |y|)$ ~~~(opero algebr.); $(x = y) \lor (-x = y) \lor (-x = y) \lor (-(-x) = y)$ ~~~(prop. val. abs.)}{$(y=x) \lor (y=-x)$ ~~~(idempot.)}

		\item Si $a.b = 0$, la recta $y=mx+b$ pasa por $P(0,0)$ o bien la recta $y=ax+3$ es horizontal.
		\answer Razonamiento válido. \\ \DReasoning{$a.b = 0$}{$(a=0) \lor (b=0)$; $(a=0) \then y=ax+3$ es horizontal; $(b=0) \then y=mx+b$ pasa por $P(0,0)$}{$y=mx+b$ pasa por $P(0,0) \lor y=ax+3$ es horizontal ~~~(dilema construc.)}

	\end{enumcols}

	\exercise Simbolizar las siguientes proposiciones, indicando el diccionario de lenguaje y el conjunto universal. Cuando se indique, negar la proposición.
	\begin{enumcols} 

		\item Alguien canta o toca la guitarra. (Adicionalmente, negar esta expresión) 
		\answer Dado el universo $U = \{x ~|~ x$ es una persona$\}$, y los predicados  $C(x)$:\textit{"la persona $x$ canta"} y  $G(x)$:\textit{"la persona $x$ toca la guitarra"}. Se simboliza $\exists x: C(x) \lor G(x)$ con su negación $\neg \left(\exists x: C(x) \lor G(x)\right) \Eq \forall x: \neg C(x) \land \neg G(x)$. \href{https://youtu.be/rnaCiSpVtP4}{Resolución por MathLogic}.

		\item Todos los números naturales son pares o negativos. (Adicionalmente, negar esta expresión) 
		\answer Dado el universo $U = \N$ y los predicados $P(x)$:\textit{"$x$ es par"} y $N(x)$:\textit{"$x$ es negativo"}. Se simboliza $\forall x: P(x) \lor N(x)$ con su negación $\neg \left(\forall x: P(x) \lor N(x)\right) \Eq \exists x: \neg P(x) \land \neg N(x)$.

		\item Si un automóvil es cómodo, entonces es caro
		\answer ado el universo $U = \{x ~|~ x$ es un automóvil$\}$ y los predicados $P(x)$:\textit{"$x$ es cómodo"} y $Q(x)$:\textit{"$x$ es caro"}. Se simboliza $\forall x: P(x) \then  Q(x)$

		\item Hay gente que cuando tiene frío toma mate, y cuando no lo tiene no
		\answer Dado el universo $U = \{x ~|~ x$ es una persona$\}$, y los predicados $F(x)$:\textit{"$x$ tiene frío"}  y $M(x)$:\textit{"$x$ toma mate"}. Se simboliza $\exists x: F(x) \eq  M(x)$.

		\item Ningún pájaro es un anfibio. Si un animal no es un anfibio, tampoco será un pez. Por lo tanto, ningun pájaro es pez.
		\answer Dado el universo $U = \{x ~|~ x$ es un animal$\}$, y los predicados $P(x)$:\textit{"$x$ es un pájaro"}, $A(x)$:\textit{"$x$ es un anfibio"} y $Z(x):$\textit{"$x$ es un pez"}. Se simboliza \\ \Reasoning{$\forall x: P(x) \then \neg A(x)$; $\forall x: \neg A(x) \then \neg Z(x)$}{$\forall x: P(x) \then Z(x)$} \\ \href{https://youtu.be/M71MEB3TkVw?t=32}{Resolución por MathLogic}.

	\end{enumcols}


	\exercise Indicar el valor de verdad de las siguientes proposiciones cuantificadas. 
	\begin{enumcols}

		\item $\exists x: P(x)$:\textit{"Según el pronóstico semanal, el día x llueve"} con $U = \{$Lunes, Martes, Miércoles, Jueves, Viernes, Sábado, Domingo$\}$
		\answer Depende del pronóstico del clima de esa semana

		\item $\forall x \in U: x^2 < 26$, con $U=\{1,3,5\}$
		\answer Verdadera, ya que $1 < 26 ~~\land~~ 9 < 26 ~~\land~~ 25 < 26$. \href{https://youtu.be/ChfOh0xG7Ok?t=121}{Resolución por Tu Profe en Linea}.

		\item $\exists x \in U: x^2 < 26$, con $U=\{1,3,5\}$
		\answer Verdadera, ya que $1 < 26$ (y con eso basta). \href{https://youtu.be/ChfOh0xG7Ok?t=246}{Resolución por Tu Profe en Linea}.

		\item $\forall x \in \N: x^2 -9 =0$
		\answer Falso, ya que $5^2-9=25-9=16 \neq 0$. \href{https://youtu.be/ChfOh0xG7Ok?t=385}{Resolución por Tu Profe en Linea}.

		\item $\forall x \in \R: x+3 < 6$
		\answer Falso, ya que $10 + 3 = 13 \nless 6$. \href{https://youtu.be/ChfOh0xG7Ok?t=506}{Resolución por Tu Profe en Linea}.

		\item $\exists x \in \R: x^2 =1$ 
		\answer La proposición es verdadera ya que $x=-1$ cumple

		\item Dado $U = \N$, $\left(\forall x: x < 5\right) \lor \left( \exists x: x^2 = 4 \right)$ 
		\answer La proposición es  verdadera ya que $\left(\forall x: x < 5\right)$ es falsa pero $\left( \exists x: x^2 = 4 \right)$ es verdadera en $x=2$

		\item $\forall x \exists y:  x < y$, donde $x$ e $y$ pertenecen al conjunto $\{1,2,3\}$.
		\answer La proposiciòn es falsa ya que para $x=3$ no existe un $y$ que cumpla.
	
	\end{enumcols}

	\exercise Averiguar si los siguientes razonamientos son válidos y, en el caso de que tengan un contexto, evaluar su solidez.
	\begin{enumcols}[2]

		\item Dada la variable $x$ y la constante $a$, se plantea: \\ \Reasoning{$\forall x: P(x)\then  Q(x) \land  R(x)$; $P(a)$}{$\exists x: R(x)$}
		\answer Razonamiento válido. \\ \DReasoning{$\forall x: P(x)\then  Q(x) \land  R(x)$; $P(a)$}{$P(a)\then  Q(a) \land  R(a)$ ~~~(part. univ. en $a$); $Q(a) \land  R(a)$ ~~~(mod. ponens); $R(a)$ ~~~(simplif.)}{$\exists x: R(x)$ ~~~(general. exist.)}

		\item \Reasoning{$Q(a) \then R(a)$; $Q(a) \then \neg R(a)$}{$\exists x: \neg Q(x)$} 
		\answer Razonamiento válido. \href{https://youtu.be/jf1nYwJEZMs?t=591}{Resolución por MathLogic}.

		\item Considerando $U=\{3,4,6\}$ y los predicados $A(x):$\textit{"$x$ es entero"}, $B(x):$\textit{"$x$ es par"} y $C(x):$\textit{"$x$ es múltiplo de 4"} se plantea: \\ \Reasoning{$\exists x: A(x) \then \neg B(x)$; $C(6) \lor B(6)$; $\neg C(6)$}{$\exists x: \neg A(x)$} 
		\answer Razonamiento inválido. \href{https://youtu.be/PtjWjkc5txI?t=535}{Resolución por Maria Alicia Piñeiro}.

		\item \Reasoning{$\neg \forall x: P(x) \then  Q(x)$}{$\exists x: P(x)$}
		\answer Razonamiento válido. \\ \DReasoning{$\neg (\forall x: P(x) \then Q(x))$; $\exists x: \neg (P(x) \then Q(x))$}{$\neg (P(a) \then Q(a))$ ~~~(part. exist.); $\neg (\neg P(a) \lor Q(a))$ ~~~(def. $\then$); $P(a) \land \neg Q(a)$ ~~~(De. Morgan); $P(a)$ ~~~(simpl.)}{$\exists x: P(a)$ ~~~(general. exist.)}

		\item \Reasoning{$\neg \exists x: \neg P(x) \land  \neg Q(x)$; $\neg P(a)$}{$Q(a)$}
		\answer Razonamiento válido. \\ \DReasoning{$\neg \exists x: \neg P(x) \land  \neg Q(x)$; $\neg P(a)$}{$\forall x: \neg (\neg P(x) \land \neg Q(x))$ ~~~(neg. cuantif.); $\neg (\neg P(a) \land \neg Q(a))$ ~~~(part. en $a$); $P(a) \lor Q(a)$ ~~~(De Morgan)}{$Q(a)$ ~~~(mod. toll. pon.)} 

		\item Todos los hombres son mortales. Sócrates es un hombre. Por lo tanto, Sócrates es mortal.
		\answer Razonamiento válido. \href{https://youtu.be/PtjWjkc5txI}{Resolución por Maria Alicia Piñeiro}.

		\item Hay productos tienen buen precio o buena calidad. Todos los productos de buena calidad son resistentes. Entonces, hay productos que tienen buen precio o que son resistentes. 
		\answer Razonamiento válido. \href{https://youtu.be/PtjWjkc5txI?t=328}{Ejercicio parecido por Maria Alicia Piñeiro}.

		\item Todos los dulces contienen carbohidratos. Existen comidas sin carbohidratos que tienen proteínas. Por lo tanto, todas las comidas tienen carbohidratos o proteinas.
		\answer Razonamiento inválido. Se puede demostrar por contraejemplo: asumiendo que todos los dulces tienen carbohidratos y que hay comidas sin carbohidratos y proteínas, aun puede suceder que no todas las comidas tengan carbohidratos o proteínas, como es el caso de la manteca. \href{https://youtu.be/AMDgepP_N_A?t=892}{Ejercicio parecido por Jonathan Castro}.

	\end{enumcols}

\end{enumerate}

\textbf{EJERCICIOS SR.}
\begin{enumerate}[resume]

	\exercise Reconocer las funciones del lenguaje e indicar qué casos son proposiciones. Si son proposiciones, indicar su valor de verdad.
	\begin{enumcols}[2]

		\item Tal vez llueve
		\answer Función informativa, es proposición. Podemos indicar si es verdadero en momentos y lugares donde es posible que llueva, mientras que será falsa si en lugares donde no es posible que llueva.

		\item ¿Está lloviendo?
		\answer Función interrogativa, no es proposición.

		\item Dadas las rectas $R_1: y=x+1$ y $R_2: y=-x+7$, $R_1 \perp R_2$.
		\answer Función afirmativa, es proposición. Se puede leer como "$R_1$ es perpendicular a $R_2$". Para ver si son perpendiculares debemos revisar la relacion entre sus pendientes. Podemos ver que $m_1 = -\f{1}{m_2}$ se cumple ya que $1 = -\f{1}{-1}$. Por lo que la proposición es verdadera.

		\item $ [ 0, \infty )$
		\answer No es una proposición. Se puede leer como "El intervalo que va desde cero a infinito". Pero carece de afirmación, simplemente enuncia un objeto, es similar al inciso x+4.

		\item Gracias
		\answer Función exclamativa, no es proposición. Notar que "Estoy agradecido con vos" sí sería una proposición.

		\item Me parece que sí 
		\answer Función informativa, es proposición. El valor de verdad no depende de si llueve o no, depende de si quien lo dice le parece que llueve o no. Si el narrador está engañando será falsa, si realmente cree que sí será verdadera.

		\item ¡No me digas!
		\answer No está clara la función pero es exclamativa o imperativa, en cualqueir caso, no es proposición.

		\item $x=9$ y $2 < x < 7$
		\answer Función afirmativa, es una proposición. Puede leerse como "$x$ es igual a 9 y 2 es menor que $x$ y $x$ es menor que 7". La proposición es falsa.

		\item En las aulas de la UNQ
		\answer No es proposición ya que se está enunciando un objeto sin afirmar nada sobre el mismo. Gramaticalmente podemos decir que es una oración sin verbo. Por otra parte, la función del lenguaje no es clara ya que carecemos de contexto. Notar que si esta oración fuese la respuesta a la pregunta "¿En dónde se cursa?", entonces podríamos interpretar que la respuesta es "(Se cursa) en las aulas de la UNQ" y sí sería una proposición.

		\item $2 < x < 7$ (donde $x$ no está definido)
		\answer No es una proposición. Como $x$ no tiene un valor el enunciado no tendrá un valor de verdad. Este tipo de enunciados los veremos al final de la unidad y se llaman esquemas proposicionales, predicados o enunciados abiertos. Notar que si $x$ fuera un número definido pero desconocido para nosotros sí sería una proposicion con valor de verdad definido (aunque desconocido).

		\item Si tan solo me hubiese acordado de regar el potus…
		\answer Función exclamativa, no es proposición.

		\item A caballo regalado no se le miran los dientes 
		\answer Es complicado. Debemos preguntarnos si hay una afirmación en esta frase. Primero deberíamos reformularla para entender si hay un verbo implicito. "Si alguien te regala un caballo, no deberías mirarle los dientes". Aquí se observa que hay una afirmación sobre si deberíamos mirarle los dientes al caballo o no. Esta afirmación sí es una proposición pero puede ser verdadera o falsa. Si nuestra interpretación de la frase es una órden como "Si te regalan un caballo, no le mires los dientes" esta oración no será una proposición, notar que se pierde el verbo \textit{deberías}. El hecho de que no haya una respuesta clara y correcta sobre este ejercicio muestra que la lógica esta limitada a la interpretación previa del lenguaje.

	\end{enumcols}

	\exercise Simbolizar las siguientes proposiciones, indicar el diccionario de lenguaje cuando sea necesario y cuando amerite hallar una proposición equivalente considerando las propiedades y definiciones de los operadores.
	\begin{enumcols}[2]

		\item Luis es feliz, si escribe poemas.
		\answer $p \then q$ con $q$:"\textit{Luis es feliz}" y $p$:"\textit{Luis escribe poemas}". Un equivalente es $\neg p \lor q$: "Luis no escribe poemas o luis es feliz". \href{https://youtu.be/HXzyX5XGPp8?t=374}{Resolución por Tu Profe en Linea}.

		\item No es cierto que estudiamos y no aprobamos.
		\answer $\neg (p \land \neg q)$ con $p$:"\textit{Estudiamos}" y $q$:"\textit{Aprobamos}". Un equivalente es $\neg p \lor q$:"\textit{No estudiamos o aprobamos}". Otro equivalente es $p \then q$:"\textit{Si estudiamos, aprobamos}". \href{https://youtu.be/HXzyX5XGPp8}{Resolución por Tu Profe en Linea}.

		\item No me gusta la pizza ni las empanadas. 
		\answer $\neg p  \land  \neg q$ con $p$:"\textit{Me gusta la pizza}" y $q$:"\textit{Me gustan las empanadas}". Un equivalente es $p \downarrow  q$, que se lee igual:"\textit{Ni me gusta la pizza, ni me gustan las empanadas}".

		\item Hoy no es jueves, ya que ayer no fue miércoles.
		\answer $\neg q \land (\neg q\then \neg p)$ con $p$:"\textit{Hoy es jueves}" y $q$:"\textit{Ayer fue miércoles}". Notar que no solo se debe indicar la relación $\neg q\then \neg p$, sino que también debemos afirmar que $\neg q$.

		\item Si llueve traigo el impermeable y las botas. 
		\answer $p \then  ( q  \land  r )$ con $p$:"\textit{Llueve}", $q$:"\textit{Traigo el impermeable}" y $r$:"\textit{Traigo las botas}". Un equivalente es $(p \then q) \land (p \then r )$:"\textit{Si llueve traigo el impermeable y si llueve traigo las botas}".

		\item Si has amado sabés lo que se siente amar, sino no. 
		\answer $( p \then  q )  \land  ( \neg p \then  \neg q )$ con $p$:"\textit{Has amado}" y $p$:"\textit{Sabés lo que se siente amar}". Un equivalente es $p \eq  q$: "\textit{Has amado si y sólo si sabes lo que se siente amar}".

		\item El anciano ingresó a la cabaña y tomo asiento, o permaneció afuera; si y solo si regresó de viaje.
		\answer $((p \land q) \lor r) \eq s$ con $p$:"\textit{El anciano ingresó a la cabaña}", $q$:"\textit{El anciano tomó asiento}", $r$:"\textit{El anciano permaneció afuera}" y $s$:"\textit{El anciano regresó de viaje}". \href{https://youtu.be/TgwraosKUuY?t=331}{Resolución por Christian Omar Arias López}.

		\item Caminamos sin prisa pero sin pausa. 
		\answer $\neg p  \land  \neg q$ con $p$:"\textit{Caminamos con prisa}" y $q$:"\textit{Caminamos con pausa}". Un equivalente es $p \downarrow  q$:"\textit{Ni caminamos con prisa ni caminamos con pausa}"

		\item Python es un lenguaje de programación interpretado, de propósito general y de alto nivel, por lo tanto resulta útil para procesar datos o automatizar tareas simples. 
		\answer $( p  \land  q  \land  r ) \land ( p  \land  q  \land  r  \then   s \lor  t )$ con $p$:"\textit{Python es un lenguaje de programación interpretado}", $q$:"\textit{Python es un lenguaje de programación  de propósito general}", $r$:"\textit{Python es un lenguaje de programación de alto nivel}", $s$:"\textit{Python resulta útil para procesar datos}" y $t$:"\textit{Python resulta útil para automatizar tareas simples}".

		\item Ir rápido equivale a ir despacio pero sin pausas. 
		\answer Hay varias simbolizaciones posibles. Por ejemplo, si reformulamos el enunciado para obtener proposiciones: "\textit{Voy rápido sí y solo si voy despacio y sin pausas}" con las proposiciones $p$:"\textit{Voy rápido}", $q$:"\textit{Voy despacio}" y $r$:"\textit{Voy con pausas}" obtenemos en símbolos $p \eq ( q \land \neg r)$.

		\item O termino las tareas en la semana, con lo que tendría el fin de semana libre, o las termino durante el fin de semana, con lo que tendría que abastecerme de café o mate.  
		\answer $(p \xor q) \land ( p \then  q ) \land  ( r \then  ( s \lor  t) )$ con $p$:"\textit{Termino las tareas en la semana}", $q$:"\textit{Tendré el fin de semana libre}", $r$:"\textit{Termino las tareas durante el fin de semana}", $s$:"\textit{Tendría que abastecerme de café}" y $t$:"\textit{Tendría que abastecerme de mate}".

		\item Para aprobar, basta dedicar tiempo y atención a la materia.  
		\answer Primero reformulamos el enunciado:"\textit{Si dedico tiempo y atención a la materia, entonces apruebo}". Entonces tenemos la proposición $( q  \land  r ) \then  p$ con $p$:"\textit{Apruebo}", $q$:"\textit{Dedico tiempo a la materia}" y $r$:"\textit{Dedico atención a la materia}".

		\item Calavera no chilla y piola se la banca.
		\answer Una opcion simple es $\neg p  \land  r$ con $p$:"\textit{Calavera chilla}" y $q$:"\textit{Piola se la banca}". Pero podríamos tener $( p\then q )  \land  ( r\then s )$ considerando $p$:"\textit{Alguien es calavera}", $q$:"\textit{Alguien chilla}", $q$:"\textit{Alguien es piola}" y $q$:"\textit{Alguien se la banca}"

		\item Vine en tren y suspendieron el servicio, así que tengo que volver en colectivo o caminando.
		\answer $( p  \land  q ) \land ( p  \land  q  \then  r \lor  s )$ con $p$:"\textit{Vine en tren}", $q$:"\textit{Suspendieron el servicio}", $r$:"\textit{Tengo que volver en colectivo}" y $s$:"\textit{Tengo que volver caminando}". Notar que además de enunciar la relacion de implicación entre las proposiciones se debe afirmar que se cumple el antecedente.

		\item En un juego son importantes las mecánicas y la historia. Las mecánicas mantienen a la persona jugadora activa. La historia la mantiene interesada. 
		\answer $( p  \land  q )  \land  r  \land  s $. $p$:"\textit{En un juego son importantes las mecánicas}", $q$:"\textit{En un juego es importantes la historia}", $r$:"\textit{Las mecánicas mantienen a la persona jugadora activa}" y $s$:"\textit{La historia la mantiene interesada}".

		\item Decir que los fantasmas se ponen azules cuando el pacman come la fruta, es lo mismo que decir que los fantasmas están azules o el pacman no se comió la fruta. 
		\answer $( q \then  p ) \eq  ( p \lor  \neg q)$ con $p$:"\textit{Los fantasmas se ponen azules}" y $q$:"\textit{El pacman come la fruta}".

	\end{enumcols}

	\exercise Determinar si las siguientes proposiciones son tautologías, contradicciones o contingencias mediante la realización de su tabla de verdad. 
	\begin{enumcols}[2]
		\item $p \xor  (\neg q \xor  r)$
		\answer Contingencia. \href{https://www.wolframalpha.com/input?i=p+or++%28not+q+xor++r%29}{Resolución}

		\item $p \land  q \lor  r$
		\answer Contingencia. \href{https://www.wolframalpha.com/input?i=p+and++q+or++r}{Resolución}.

		\item $p \nimply (q \land  r)$
		\answer Contingencia. \href{https://www.wolframalpha.com/input?i=not+%28p+%3D%3E+%28q+and++r%29%29}{Resolución}.

		\item $(T_0 \nor q) \nimply r$
		\answer Contradicción. \href{https://www.wolframalpha.com/input?i=not+%28%28p+nor+q%29+%3D%3E+r%29}{Resolución, ver sólo donde $p$ es V}.

		\item $(\neg p \eq q) \land \neg (q \then \neg p)$
		\answer Es contradicción. \href{https://youtu.be/n_t1f0xa3D0?t=413}{Resolución por ProfeGuille}.

		\item $((p \land \neg q) \then \neg r) \lor (p \xor q)$
		\answer Es tautología. \href{https://youtu.be/n_t1f0xa3D0?t=643}{Resolución por ProfeGuille}.

		\item $\neg ((r \then p) \land (\neg q \lor p)) \land ( p \land (p \then r))$
		\answer Es contradicción. \href{https://youtu.be/rlJmcBGdOoY}{Resolución por ProfeGuille}.

		\item $\neg ((\neg p \land q) \then p ) \lor q$
		\answer Es contingencia. \href{https://youtu.be/n_t1f0xa3D0?t=40}{Resolución por ProfeGuille}.

	\end{enumcols}

	\exercise Averiguar si las siguientes proposiciones son equivalentes mediante tablas de verdad y/o reglas de equivalencia.  Dar un ejemplo en lenguaje natural que lo evidencie. Nota: no siempre el mismo método.
	\begin{enumcols}[2]
		\item $\neg p\then p \Eq p$
		\answer La \href{https://www.wolframalpha.com/input?i=truth+table%3A%28not+p+%3D%3E+p%29+%3C%3D%3E+p}{tabla de verdad} revela que es una tautología. Por lo tanto, son equivalentes. Esta regla de equivalencia no tiene nombre (ni es muy útil). 

		\item $\neg p\lor \neg q \Eq (p\land q)$
		\answer La \href{https://www.wolframalpha.com/input?i=%28not+p+or+not+q%29+%3C%3D%3E+%28p+and+q%29}{tabla de verdad} revela que es una contradicción. Por lo tanto, las proposiciones no son equivalentes.

		\item $(p \land \neg q) \Eq (\neg p \lor q)$
		\answer La \href{https://www.wolframalpha.com/input?i=%28p+and+not+q%29+%3C%3D%3E+%28not+p+or+q%29}{tabla de verdad} revela que es una contradicción. Por lo tanto, las proposiciones no son equivalentes. También pueden ver la \href{https://youtu.be/NZSuHeymu4M?t=639}{resolución por Tu Profe en Linea}.

		\item $(p \then (q \lor r)) \Eq ((p \land r) \then \neg q)$
		\answer La \href{https://www.wolframalpha.com/input?i=%28p+%3D%3E+%28q+or+r%29%29+%3C%3D%3E+%28%28p+and+r%29+%3D%3E+%5Cneg+q%29}{tabla de verdad} revela que es una contingencia. Por lo tanto, las proposiciones no son equivalentes. Para más detalles ver la \href{https://www.youtube.com/live/-yHPDgU-lfE?feature=share&t=578}{resolución por Julio Profe}.

		\item $\neg (p\land q\then \neg r)	\Eq 	p\land q\lor r$
		\answer La \href{https://www.wolframalpha.com/input?i=not+%28%28p+and+q%29+%3D%3E+not+r%29+%3C%3D%3E++%28%28p+and+q%29+or+r%29}{tabla de verdad} revela que es una contingencia. Por lo tanto, no son equivalentes. 

		\item $p\then (q\land r\land s) \Eq (p\then q)\land (p\then r)\land (p\then s)$
		\answer La \href{https://www.wolframalpha.com/input?i=truth+table%3A+%28p+%3D%3E+%28q+and+r+and+s%29%29+%3C%3D%3E+%28%28p+%3D%3E+q%29+and+%28p+%3D%3E+r%29+and+%28p+%3D%3E+s%29%29}{tabla de verdad} revela que es una tautología. Por lo tanto, son equivalentes. Esta regla de equivalencia no tiene nombre.

	\end{enumcols}

	\exercise Simplificar las siguientes expresiones
	\begin{enumcols}[2]
		\item $(p \xor q) \then (q \then p)$
		\answer Puede ser $q \then p$, o bien $p \lor \neg q$. \href{https://youtu.be/5r8S-wMJq7I}{Resolución por ProfeGuille}.

		\item $((p \lor \neg q) \then \neg p) \land (\neg p \eq q)$
		\answer $\neg p \land q$. \href{https://youtu.be/Ayk4qXcoiOM}{Resolución por ProfeGuille}.

		\item $((\neg q \then p) \then (p \lor \neg q)) \land \neg (p \land q)$
		\answer La expresión más simple es $\neg q$. \href{https://youtu.be/BOydu7cpv70?t=884}{Resolución por Tu Profe en Linea}.

		\item $( p \then \neg (q \then p) ) \then \neg q$
		\answer Puede ser $q \then p$, o bien $p \lor \neg q$. \href{https://youtu.be/BOydu7cpv70?t=334}{Resolución por Tu Profe en Linea}.

		\item $(q \then \neg p) \land ((p \land q)\then(p \eq q))$
		\answer Puede ser $p \then \neg q$, o bien $\neg ( p \land q )$, o también $\neg p \lor \neg q$. \href{https://youtu.be/p005yi28rgk?t=392}{Resolución por ProfesorTriquero}.

		\item $(p \lor (\neg q \land r)) \then (p \then (\neg p \land q))$
		\answer La expresión mas simple es $\neg p$. \href{https://youtu.be/UZDME4cZxNc}{Resolución por ProfeGuille}.

		\item $((p \then q) \lor \neg (\neg q \lor \neg p)) \then \neg q$
		\answer La expresión mas simple es $\neg q$. \href{https://youtu.be/DbjwP3w5yTE}{Resolución por ProfeGuille}.
	\end{enumcols}

	\exercise Escribir en forma normal disyuntiva (DNF) y en forma normal conjuntiva (CNF) las siguientes proposiciones.
	\begin{enumcols}[2]
		\item $\neg (p\then q)$
		\answer DNF: $(p\land \neg q)$  \\ CNF: $(\neg p\lor \neg q) \land  (p\lor q) \land  (p\lor \neg q)$

		\item $(p \eq  q) \eq r$
		\answer DNF: $(p\land q\land r) \lor  (p\land \neg q\land \neg r) \lor  (\neg p\land q\land \neg r) \lor  (\neg p\land \neg q\land r)$  \\ CNF: $(\neg p\lor \neg q\lor r) \land  (\neg p\lor q\lor \neg r) \land  (p\lor \neg q\lor \neg r) \land  (p\lor q\lor r)$

		\item $p \land (q \then r)$
		\answer DNF: $p \land q \land \neg r$ \\ CNF: $(\neg p \lor \neg q \lor \neg r)\land (\neg p \lor q \lor \neg r)\land (\neg p \lor q \lor r)\land (p \lor \neg q \lor \neg r)\land (p \lor \neg q \lor r)\land (p \lor q \lor \neg r)\land (p \lor q \lor r)$

	\end{enumcols}

	\exercise Sin utilizar las tablas de verdad, demostrar si los siguientes razonamientos son válidos o no. 
	\begin{enumcols}[3]
		
		\item \Reasoning{$p\then q$ ; $r\land p$ ; $q\eq s$}{$q\land s$}
		\answer \DReasoning{$p\then q$ ; $r\land p$ ; $q\eq s$}{$q \then s$ ~~~(elim. $\eq$); $p$ ~~~(simplif.); $q$ ~~~(mod. ponens); $s$ ~~~(mod. ponens)}{$q\land s$ ~~~(conjunción)}

		\item \Reasoning{$p\then q$ ; $q\lor r\then s$}{$p\lor r\then s$}
		\answer Aplicamos exportación y casos. Caso 1: \\ \DReasoning{$p\then q$ ; $q\lor r\then s$;$p$}{$q$ ~~~(modus ponens); $q \lor r$ ~~~(adición)}{$s$ ~~~(modus ponens)} \\ Caso2: \\ \DReasoning{$p\then q$ ; $q\lor r\then s$;$r$}{$q \lor r$ ~~~(adic. y conmut.)}{$s$ ~~~(modus ponens)}

		\item \Reasoning{$p\then q$ ; $r\then s$}{$(p\land r)\then (q\land s)$}
		\answer Aplicamos exportación: \\ \DReasoning{$p\then q$ ; $r\then s$;$p$;$r$}{$q$ ~~~(mod. ponens); $s$ ~~~(mod. ponens)}{$q\land s$ ~~~(conjunción)}

		\item \Reasoning{$p \then q$ ; $s \lor \neg q$ ; $\neg s$}{$\neg p$}
		\answer \DReasoning{$p \then q$ ; $s \lor \neg q$ ; $\neg s$}{$\neg q$ ~~~(mod. toll. pon.)}{$\neg p$ ~~~(mod. toll.)}

		\item \Reasoning{$p \land q$ ; $(p \lor r) \then t$}{$p\land t$}
		\answer \DReasoning{$p \land q$ ; $(p \lor r) \then t$}{$p$ ~~~(simpl.); $p \lor r$ ~~~(adición); $t$ ~~~(mod. ponens)}{$p\land t$ ~~~(conjunción)}

		\item \Reasoning{$p \lor q$ ; $r \lor s$ ; $p \then r$ ; $q \then s ;\neg r$}{$s$}
		\answer Recordar que no hace falta usar todas las premisas: \\ \Reasoning{$p \lor q$ ; $r \lor s$ ; $p \then r$ ; $q \then s ;\neg r$}{$s$ ~~~(m.t.p. premisas 2 y 5)}

		\item \Reasoning{$p \then q$ ; $q \then r$  ; $s \then t$ ; $p \lor  s$}{$r\lor t$}
		\answer \DReasoning{$p \then q$ ; $q \then r$  ; $s \then t$ ; $p \lor  s$}{$p \then r$ ~~~(transit. de $\then$)}{$r\lor t$ ~~~(dilema construc.)}

		\item \Reasoning{$p \then q$ ; $(p \land  q) \then r$ ; $\neg(p \land  q)$}{$\neg p$}
		\answer Por contradicción (asumo $p$): \\ \DReasoning{$p \then q$ ; $(p \land  q) \then r$ ; $\neg(p \land  q)$;$p$}{$q$ ~~~(mod pon.); $(p \land q) \land \neg (p \land q)$ ~~~(conj.)}{$F_0$ ~~~(binarismo excl.)}

		\item \Reasoning{$q \then r$ ; $\neg s \then (t \then u)$ ; $s \lor (q \lor t)$ ; $\neg s$}{$r\lor u$}
		\answer \DReasoning{$q \then r$ ; $\neg s \then (t \then u)$ ; $s \lor (q \lor t)$ ; $\neg s$}{$t \then u$ ~~~(mod. pon.); $q \lor t$ ~~~(mod. tol. pon)}{$r\lor u$ ~~~(dilema construc.)}

		\item \Reasoning{$p \lor q \then r$; $r \then p \land s$}{$p \eq r$}
		\answer Demostramos $p \then r$ usando exportación: \\ \DReasoning{$p \lor q \then r$; $r \then p \land s$;$p$}{$p \lor q$ ~~~(adición)}{$r$ ~~~(mod. ponens)} \\ Demostramos $r \then p$ usando exportación: \\ \DReasoning{$p \lor q \then r$; $r \then p \land s$;$r$}{$p \land s$ ~~~(mod. ponens)}{$p$ ~~~(simpl.)} \\ Por lo tanto, $p \eq r$.

	\end{enumcols}

	\exercise Dado el valor de verdad de algunas propociciones, averiguar el valor de otras utilizando tablas de verdad, reglas lógicas, razonamientos, etc. Se sugiere no siempre usar el mismo método.
	\begin{enumcols}

		\item Dado que la proposición $(r\land \neg p)\then \neg q$ es V, $p$ es F y $q$ es V, hallar el valor de $r$. 
		\answer $r$ es F 

		\item Dado que la proposición $(p\lor \neg q) \eq  (r\then s)$ es F  y $r$ es F, hallar el valor de $p$ y de $q$. 
		\answer $p$ es F y $q$ es V

		\item Dado que la proposición $\neg (p\land q) \xor  (r\lor s)$ es V  y $r$ es V, hallar el valor de $p$ y de $q$. 
		\answer $p$ es V y $q$ es V

		\item Dado que la proposición $(p\xor q) \land  \neg (r\lor s)$ es V  y $p$ es V, hallar el valor de $q$, $r$ y $s$. 
		\answer $q$ es F, $r$ es F y $s$ es F

		\item Dado que la proposición $(p\lor r) \xor  (q\land p \eq r)$ es F  y $r$ es V, hallar el valor de $p$ y de $q$. 
		\answer $p$ es V y $q$ es V

		\item $r$ es F. Averiguar el valor de $((r \then q) \xor \neg r) \land p$.
		\answer $((r \then q) \xor \neg r) \land p$ es F. \href{https://youtu.be/bitBrw1NvNk?t=663}{Resolución por Tu Profe en Linea}.

		\item $p$ es V. Averiguar el valor de $((r \land \neg p) \land (q \lor p)) \then r$.
		\answer $((r \land \neg p) \land (q \lor p)) \then r$ es V. \href{https://youtu.be/bitBrw1NvNk?t=793}{Resolución por Tu Profe en Linea}.

		\item $(p \then q) \lor \neg r$ es F. Averiguar el valor de $p$, $q$ y $r$.
		\answer $p$ es V, $q$ es F y $r$ es V. \href{https://youtu.be/bitBrw1NvNk}{Resolución por Tu Profe en Linea}.

		\item $(p \land \neg q) \then \neg (r \then \neg s)$ es F. Averiguar el valor de $p$, $q$, $r$ y $s$.
		\answer $p$ es V, $q$ es F, $r$ es V y $s$ es V. \href{https://youtu.be/bitBrw1NvNk?t=140}{Resolución por Tu Profe en Linea}.

		\item $((p \then q) \eq t) \lor (p \then t)$ es F. Averiguar el valor de $p$, $q$ y $t$.
		\answer $p$ es V, $q$ es V y $t$ es F. \href{https://youtu.be/bitBrw1NvNk?t=273}{Resolución por Tu Profe en Linea}.

		\item $((s \then p) \then (p \eq q)) \lor (p \land r)$ es F y $s$ es V. Averiguar el valor de $p$, $q$ y $r$.
		\answer $p$ es V, $q$ es F y $r$ es F. \href{https://youtu.be/bitBrw1NvNk?t=465}{Resolución por Tu Profe en Linea}.

		\item $\neg \left( (p \then q) \to s(s \then r)\right)$ es V. Averiguar el valor de $(\neg q \then \neg p) \xor (r \then s)$ y de $(\neg p \lor q) \land (s \land \neg r)$
		\answer $(\neg q \then \neg p) \xor (r \then s)$ es F y $(\neg p \lor q) \land (s \land \neg r)$ es V. \href{https://youtu.be/p005yi28rgk}{Resolución por ProfesorTriquero}.

		\item $\left(((p \xor q) \land r) \then (s \eq r) \right) \lor \left((q \then p) \then (\neg s \land t)\right)$ es F. Averiguar el valor de $p \then q$, de $r \xor s$ y de $(p \lor q) \then (\neg s \xor t)$
		\answer $p \then q$ es F, $r \xor s$ es V y $(p \lor q) \then (\neg s \xor t)$ es V. \href{https://youtu.be/-VaC5y-rrMI}{Resolución por Tu Profe en Linea}.

	\end{enumcols}


	\exercise Averiguar si los siguientes razonamientos son válidos y evaluar su solidez.
	\begin{enumcols} 
		\item Por el pronóstico semanal, sabemos que si es martes, llueve y hace frío. También sabemos que no llueve. Demostrar que no es martes.
		\answer Razonamiento válido. Usamos $m$:"\textit{es martes}", $l$:"\textit{llueve}", $f$:"\textit{hace frio}". Por lo que el razonamiento será \\ \DReasoning{$m \then l \land f$; $\neg l$}{$\neg l \lor \neg f$ ~~~(adición); $\neg (l \land f)$ ~~~(De Morgan)}{$\neg m$ ~~~(mod. tol.)}

		\item Para bajar de peso debo hacer dieta o ir al gimnasio. Si voy al gimnasio gasto dinero. Quiero bajar de peso sin gastar dinero. Entonces voy a hacer dieta. 
		\answer Razonamiento válido. Utilizando $b$:"\textit{bajo de peso}", $d$:"\textit{hago dieta}", $g$:"\textit{voy al gimnasio}" y $s$:"\textit{gasto dinero}", se demuestra: \\ \DReasoning{$b \then d \lor g$; $g \then s$; $b; \neg s$}{$d \lor g$ ~~~(mod. pon.); $\neg g$ ~~~(mod. tol.)}{$d$ ~~~(mod. tol. pon.)} \\ \href{https://youtu.be/DD5EleyOl-0?t=255}{Planteo por Maria Alicia Piñeiro}.

		\item Este disco está roto o fue borrado. Si estuviera roto, tendría marcas en las pistas. Si el disco está protegido, no se puede borrar. Por lo tanto, este disco tiene marcas en las pistas o no está proteigo.
		\answer Razonamiento válido. \href{https://youtu.be/rEB0nEdSFXo}{Resolución por Maria Alicia Piñeiro}.

		\item Si $n$ es un número entero impar, entonces $n^2$ es impar.
		\answer Razonamiento válido. \href{https://youtu.be/yAGoR6FUyi8}{Resolución por Aula de Matemática}.

		\item Si $3n+2$ es un número entero impar, entonces $n$ es impar.
		\answer Razonamiento válido. \href{https://youtu.be/yAGoR6FUyi8?t=159}{Resolución por Aula de Matemática}.

		\item Si $n^2$ es un número entero par, entonces $n$ es par.
		\answer Razonamiento válido. \href{https://youtu.be/yAGoR6FUyi8?t=310}{Resolución por Aula de Matemática}.

		\item Si el resultado de multiplicar un número entero por 5 y sumarle 3 es un número par, el número inicial es impar
		\answer Razonamiento válido. Demostraremos el contrarrecíproco: "\textit{Si un nùmero no es impar, el resultado de multiplicarlo por 5 y sumarle 3 no será par}": Utilizamos el numero entero $n \in \Z$\\ \DReasoning{$n$ es par}{$n=2k$ con $k\in \Z$; $5n+3=10k+3=2(5k+1)+1$ con $(5k+1) \in \Z$;$5n+3$ es impar}{$5n+3$ no es par}
		
		\item La suma de dos múltiplos de 3 es un número par
		\answer Razonamiento inválido. Contraejemplo: $3+6=9$. $3$ y $6$ son múltiplos de 3, pero $9$ no es par

		\item La suma de un múltiplo de 3 y de un múltiplo de 2 nunca es múltiplo de 3
		\answer Razonamiento inválido. Contraejemplo: $3+6=9$. $3$ es múltiplo de 3 y $6$ es múltiplo de 2, pero $9$ es múltiplo de 3.
		
		\item Si un número natural es múltiplo de 10, entonces es múltiplo de 100
		\answer Razonamiento inválido. Contraejemplo: $30$ es múltiplo de 10, pero no es múltiplo de 100.

		\item Dados dos números racionales, $r$ y $s$. La suma de $r$ y $s$ es un número racional. Tambien se podría leer como: La suma de dos números racionales es racional. 
		\answer Razonamiento válido. \href{https://youtu.be/wbGe5hIjFUo}{Resolución por Aula de Matemática}.

		\item El producto de dos números números racionales es racional. 
		\answer Razonamiento válido. \href{https://youtu.be/kLjDAzcKWMI}{Resolución por Aula de Matemática}.

		\item Para cualquier número natural n se verifica que $n^2 - (n-1)^2 < 20$  o que  $n^2 - (n-1)^2 >50$.

		\item Dadas tres rectas no verticales $R_1$, $R_2$ y $R_3$. Si $R_1 \perp  R_2$ y $R_2 \perp  R_3$, entonces $R_1 \parallel  R_3$ o bien $R_1 \sim R_3$.
		\answer Razonamiento válido. Utilizamos las rectas $R_1$, $R_2$ y $R_3$ a partir de la ecuación $y=mx+b$. \\ \DReasoning{$R_1 \perp  R_2$; $R_2 \perp  R_3$}{$m_2 = -\f{1}{m_1}$;$m_2 = -\f{1}{m_3}$; $m_1 = m_3$ ~~~(despeje); $(m_1 = m_3) \land T_0$ ~~~(identidad); $(m_1 = m_3) \land \left[(b_1=b_3) \lor (b_1 \neq b_3)\right]$ ~~~(binar. excl.); $\left[(m_1 = m_3) \land (b_1=b_3)\right] \lor \left[(m_1 = m_3) \land  (b_1 \neq b_3)\right]$ ~~~(distribut.)}{$(R_1 \sim R_3) \lor (R_1 \parallel  R_3)$ ~~~(props. $\sim$ y $\parallel$)}

		\item Si $k>2$, la parábola $x^2+k.x+1$ no tiene raíces.
		\answer Razonamiento inválido. Se puede refutar mediante: \\ \DReasoning{$k>2$}{$k^2>4$;$k^2-4>0$; $b^2-4ac=k^2-4$ ~~~(calc. discrim.); $b^2-4ac>0$}{$x^2+k.x+1$ tiene 2 raíces reales} \\ También se puede buscar un contraejemplo: $k=3$ y $x^2+3.x+1$ tiene las raíces reales $x_1=\f{-3-\sqrt{5}}{2}$ y $x_2=\f{-3+\sqrt{5}}{2}$.

		\item Si $a.b < 0$ y $|a|>b>0$, la parábola $x^2+a$ tiene dos raíces reales.
		\answer Razonamiento válido. \\ \DReasoning{$a.b < 0$;$|a|>b>0$}{$(a>0 \land b<0) \lor (a<0 \land b>0)$ ~~~(condición desigualdad); $b>0$ ~~~(simplif.); $a>0$ ~~~(distr., ident. y simplif.); $-4a>0$ ~~~(calc. discrim.)}{$x^2+a$ tiene dos raíces reales.}

	\end{enumcols}

	\exercise Simbolizar las siguientes proposiciones, indicando el diccionario de lenguaje y el conjunto universal.
	\begin{enumcols} 

		\item Hay automóviles veloces y cómodos
		\answer Dado el universo $U = \{x ~|~ x$ es un automóvil$\}$ y los predicados $V(x)$:\textit{"$x$ es veloz"}, $C(x)$:\textit{"$x$ es cómodo"}. Se simboliza $\exists x: V(x) \land  Q(x)$

		\item No todos los número enteros múltiplos de 5 son impares
		\answer Dado el universo $U = \Z$ y los predicados $P(x)$:\textit{"$x$ es múltiplo de 5"} y $Q(x)$:\textit{"$x$ es impar"}. Se simboliza $\neg \left(\forall x: P(x) \then  Q(x) \right)$

		\item Existen números naturales que son múltiplos de 2 pero no de 3
		\answer Dado el universo $U = \N$ y los predicados  $P(x)$:\textit{"$x$ es múltiplo de 2"} y $Q(x)$:\textit{"$x$ es múltiplo de 3"}. Se simboliza $\exists x: P(x) \land  \neg Q(x)$

		\item Considerando las tazas en mi cocina, escribir las siguientes variantes de proposiciones: Las tazas que me regaló mi abuela son de porcelana. Algunas de las tazas que me regaló mi abuela son de porcelana. Todas mis tazas me las regaló mi abuela y son de porcelana.
		\answer Dado el universo $U = \{x ~|~ x$ es una taza de mi cocina $\}$ y los predicados $P(x)$: \textit{"mi abuela me regaló la taza $x$"} y $Q(x)$: \textit{"la taza $x$ es de porcelana"}. Se simboliza $\forall x: P(x) \then Q(x)$, $\exists x: P(x) \land Q(x)$, $\forall x: P(x) \land Q(x)$. \href{https://youtu.be/aI-G5V-IGfA}{Resolución por Maria Alicia Piñeiro}.

		\item Si todo es fácil y agradable entonces Marta no estudiará. No hay cosas desagradables. Además, todo es fácil. Entonces, Marta no estudiará.
		\answer Dado el universo $U = \{x ~|~ x$ es una cosa $\}$ y los predicados $F(x)$:\textit{"$x$ es fácil"}, $A(x)$:\textit{"$x$ es agradable"}, $E(y)$:\textit{"$y$ estudiará"} y la constante $m=$Marta. Se simboliza \\  \Reasoning{$\left(\forall x: F(x) \land A(x) \right) \then \neg E(m)$;$\neg \left( \exists x: \neg A(x) \right)$;$\forall x: F(x)$}{$\neg E(m)$}. \\\href{https://youtu.be/M71MEB3TkVw?t=736}{Resolución por MathLogic}.

		\item Todo ejecutivo que sea poeta no es un hombre imaginativo. Todo hombre imaginativo es amante del riesgo. Por consiguiente, si todo hombre imaginativo no es poeta, ningún ejecutivo es poeta.
		\answer \href{https://youtu.be/M71MEB3TkVw?t=1191}{Resolución por MathLogic}.

		\item Ana no duerme. Todos los que tienen sueño, duermen. Hay una persona joven que no tiene sueño. Todos los 	arquitectos que no tienen sueño escuchan la radio. En todos los casos los jóvenes tienen sueño o usan la computadora. Marcos es un arquitecto que usa la computadora. No hay nadie que escuche la radio y use la computadora. Hay alguien que no es arquitecto, escucha la radio y tiene sueño.
		\answer Dado el universo $U = \{x ~|~ x$ es una persona$\}$, y los predicados $D(x)$:\textit{"$x$ duerme"}, $S(x)$:\textit{"$x$ tiene sueño"}, $J(x)$:\textit{"$x$ es joven"}, $A(x)$:\textit{"$x$ es arquitecto"}, $R(x)$:\textit{"$x$ escucha la radio"} y $C(x)$:\textit{"$x$ usa la computadora"}. Se simboliza $\neg D(Ana)$, $\forall x:S(x) \then  D(x)$, $\exists x: J(x) \land  S(x)$, $\forall x: A(x) \land  \neg S(x) \then  R(x)$, $\forall x: J(x) \then  S(x) \lor  C(x) $, $A(Marcos) \land  C(Marcos)$, $\neg \left(\exists x: R(x) \land  C(x) \right)$ y finalmente $\exists x: \neg A(x) \land  R(x) \land  S(x) $.

	\end{enumcols}


	\exercise Indicar el valor de verdad de las siguientes proposiciones cuantificadas. 
	\begin{enumcols}

		\item $\exists x \in U: x^2 < 26$, con $U=\{7,8\}$
		\answer Falso. \href{https://youtu.be/ChfOh0xG7Ok?t=305}{Resolución por Tu Profe en Linea}.

		\item $\exists x \in \N: x^2 -9 =0$
		\answer Verdadero, ya que $3^2-9=9-9=0$. \href{https://youtu.be/ChfOh0xG7Ok?t=460}{Resolución por Tu Profe en Linea}.

		\item $\exists x \in \R: x+3 < 6$
		\answer Verdadero, ya que $2 + 3 = 5 \nless 6$. \href{https://youtu.be/ChfOh0xG7Ok?t=557}{Resolución por Tu Profe en Linea}.

		\item $\forall x \in \Z: x-5 < 8$ 
		\answer La proposición es falsa, ya que $20 \in \Z \land 20-5 = 13 \nless 8$. \href{https://youtu.be/rnaCiSpVtP4?t=303}{Resolución por MathLogic}.

		\item Dado $U = \{1,2,3,4\}$ y los predicados $P(x)$:\textit{"$x$ es múltiplo de 2"} y $Q(x)$:\textit{$x \leq 3$}, determinar el valor de $\left(\exists x: P(x) \land  Q(x) \right) \land \left(\forall x: P(x) \lor  Q(x) \right) \land  \left(\forall x: P(x) \then  Q(x) \right)$
		\answer Como $\left(\forall x: P(x) \then  Q(x) \right)$ es falsa, toda la proposición es falsa

		\item Dadas dos variables $x,y \in \{1,2,3\}$. Hallar el valor de verdad de $\left(\forall x ~\exists y: x+y = 2 \right) \lor \neg \left(\forall x \exists y: x+y = 4 \right) \lor \left(\forall x \forall y: x+y \geq  4 \right)$
		\answer La proposiciòn es falsa. $\left(\forall x \exists y: x+y = 2 \right)$ es falsa ya que para $x=3$ no existe $y$ que cumpla, $\neg \left(\forall x \exists y: x+y = 4 \right)$ es falsa ya que para $x=1$ tenemos $y=3$, para $x=2$ tenemos $y=2$, para $x=3$ tenemos $y=1$ y finalmente $\left(\forall x \forall y: x+y \geq  4 \right)$ es falsa ya que para $x=2$ e $y=1$ no se cumple. 

		\item $\forall x \in \R ~\exists y \in \R:  x+y = 1 $ 
		\answer La proposición es verdadera. Dado un $x$ genérico puedo elegir $y=1-x$ que cumple la proposición. 

		\item $\exists x \in \N ~ \forall y \in \N:  x > y  \land   x^2 < y $ 
		\answer La proposición es falsa. Para un determinado $y$ debo elegir un $x$ que cumpla $x^2 < y < x$, es decir, $x^2 < x$. Ningún $x \in \N$ cumple esa condición.

		\item $\forall x \in \R ~\forall y \in \R:  x-y \leq  0  \then   x < y+2 $ 
		\answer La proposición es verdadera. Como $x \leq y < y+2$, si $x$ cumple el antecedente $x \leq  y$, seguro cumple el consecuente $x < y+2$. 

		\item Dados $x,y,z \in \R$, hallar el valor de verdad de $\forall x ~\forall y ~\exists z: x < y  \then   x < z < y $ 
		\answer Verdadero, si elegimos $z=\f{x+y}{2}$ (el promedio), cumplirá la condición. F, por las propiedades de la igualdad, que se cumplen para todos los número reales.

		\item Dados $x,y,z \in \R$, hallar el valor de $\exists x ~\exists y ~\exists z: x=y \land  y=z \then  x\neq  z $ 
		\answer Falso, por las propiedades de la igualdad, que se cumplen para todos los número reales.

		\item Dado $U = \{1,2,3\}$ y el predicado $P(x): x-2=0$, determinar el valor de $\left(\exists x: P(x)\right)\land\left(\forall y \forall z: P(y) \land P(z) \then y=z \right)$
		\answer Verdadero. $\left(\exists x: P(x)\right)$ es verdadero y se pueden probar las 9 combinaciones de valores de $y$ y $z$ para demostrar  que $\left(\forall y ~\forall z: P(y) \land P(z) \then y=z \right)$ también.

		\item Dado $U = \{1,2,3\}$ indicar el valor de verdad de $\exists ! x: x^2 < 3$
		\answer Verdadero ya que sólo $P(1)$ es verdadero

		\item Dado $U = \{1,2,3\}$ indicar el valor de verdad de $\exists ! x: x^2 > x$
		\answer Falso ya que $P(2)$ y $P(3)$ son verdaderas

	\end{enumcols}


	\exercise Averiguar si los siguientes razonamientos son válidos y, en el caso de que tengan un contexto, evaluar su solidez.
	\begin{enumcols}[2]

		\item \Reasoning{$\neg \exists x: \neg P(x) \land  Q(x)$; $\forall x:  P(x) \then  R(x)$; $Q(a)$}{$R(a)$}
		\answer Razonamiento válido. \\ \DReasoning{$\neg \exists x: \neg P(x) \land  Q(x)$; $\forall x:  P(x) \then  R(x)$; $Q(a)$}{$\forall x: \neg( \neg P(x) \land Q(x))$ ~~~(neg. cuantif.); $\neg( \neg P(a) \land Q(a))$ ~~~(part. en $a$); $P(a) \lor \neg Q(a)$ ~~~(De Morgan); $P(a)$ ~~~(mod. toll. pon.); $P(a) \then  R(a)$ ~~~(part. en $a$)}{$R(a)$ ~~~(mod. pon.)}

		\item \Reasoning{$\forall x: P(x) \then  Q(x) \land  R(x)$; $\forall x: R(x) \lor  S(x) \then  T(x)$; $P(a)$}{$T(a)$}
		\answer \DReasoning{$\forall x: P(x) \then  Q(x) \land  R(x)$; $\forall x: R(x) \lor  S(x) \then  T(x)$; $P(a)$}{$P(a) \then  Q(a) \land  R(a)$ ~~~(part. en $a$); $Q(a) \land  R(a)$ ~~~(mod. pon.); $R(a)$ ~~~(simplif.); $R(a) \lor  S(a)$ ~~~(adición); $R(a) \lor  S(a) \then  T(a)$ ~~~(part. en $a$)}{$T(a)$ ~~~(mod. pon.)}

		\item \Reasoning{$\forall x\forall y\forall z:  P(x,y)\land P(y,x) \then  (P(x,z)\eq P(y,z)) $; $P(a,b)$; $P(b,c)$; $P(b,a)$}{$P(a,c)$}
		\answer \DReasoning{$\forall x\forall y\forall z:  P(x,y)\land P(y,x) \then  (P(x,z)\eq P(y,z)) $; $P(a,b)$; $P(b,c)$; $P(b,a)$}{$P(a,b)\land P(b,a) \then  (P(a,c)\eq P(b,c))$ ~~~(part. $x=a$, $y=b$, $z=c$);$P(a,c)\eq P(b,c)$ ~~~(mod. ponens)}{$P(a,c)$ ~~~(mod. pon. del $\eq$)}

		\item \Reasoning{$\forall x \forall y: \neg (R(x) \then \neg S(x,y))$; $\forall x \exists y: P(x) \then Q(x,y)$; $\exists x \forall y: R(x) \land Q(x,y) \then \neg S(x,y)$}{$\forall x: \neg P(x)$ }
		\answer Razonamiento válido. \href{https://youtu.be/DD5EleyOl-0?t=996}{Resolución por MathLogic}.

		\item \Reasoning{$\forall x \forall y: \neg (R(x) \then \neg S(x,y))$; $\forall x \exists y: P(x) \then Q(x,y)$; $\exists x \forall y: R(x) \land Q(x,y) \then \neg S(x,y)$}{$\exists x: \neg P(x)$}
		\answer Razonamiento válido. \href{https://youtu.be/0Dcg9nDzkys?t=1194}{Resolución por Maria Alicia Piñeiro}.

		\item Todas las personas invitadas a la cena estudiaron abogacía o ingeniería. Quienes estudiaron ingeniería estudiaron en la UNQ. Ariel, uno de los invitados, no estudió en la UNQ. Por lo tanto, al menos una persona invitada es abogada.  
		\answer Razonamiento válido. \href{https://youtu.be/AMDgepP_N_A}{Resolución por Jonathan Castro}.

		\item Algunos sillones están tapizados. Algunos sillones son blancos. Todos los sillones blancos tienen almohadones. Por lo tanto, algunos sillones están tapizados y tienen almohadones.
		\answer Razonamiento inválido. Se puede demostrar por contraejemplo, definiendo un universo de sillones a elección. \href{https://youtu.be/kszEIGQ1XOA}{Resolución por Maria Alicia Piñeiro}.

		\item Todos los sillones están tapizados. Algunos sillones son blancos. Todos los sillones blancos tienen almohadones. Por lo tanto, algunos sillones están tapizados y tienen almohadones.
		\answer Razonamiento válido. Se puede demostrar por contraejemplo, definiendo un universo de sillones a elección. \href{https://youtu.be/kszEIGQ1XOA?t=285}{Resolución por Maria Alicia Piñeiro}.

		\item Todos los bebes de terapia estaban en incubadora o con respirador. Los que estaban en incubadora eran prematuros y de bajo peso. Lucio, uno de los bebes de terapia, tenía buen peso. Por lo tanto, al menos un bebé de terapia estaba con respirador.
		\answer Razonamiento válido. \href{https://youtu.be/6QGZCc6QbVE}{Resolución por Maria Alicia Piñeiro} y su \href{https://youtu.be/C-BJgBFUIPA}{continuación}.

	\end{enumcols}

	\exercise Resuelve los siguientes ejercicios variados
	\begin{enumcols}

		\item Dado el esquema proposicional $F(x): -x<1 \then x>5$ donde $x \in \Z$, encontrar dos constantes para $x$ tales que se conviertan en una proposicion verdadera y otros dos para que sea falsa.
		\answer Por ejemplo, 10 y 12 para que sea V, y -6 y -7 para que sea F. \href{https://youtu.be/DLsV_L097gE?t=687}{Resolución por MathLogic}.

		\item Analizar si las proposiciones $\left(\exists x: P(x)\right) \land \left( \forall x: P(x) \then Q(x) \right)$ es equivalente a $\exists x: Q(x)$.
		\answer Las expresiones no son equivalentes. \href{https://youtu.be/WC7P8FMIHFw}{Resolución por Maria Alicia Piñeiro}.

		\item Utilizando el predicado $P(x): 2x+1 >2$ con el conjunto $U=\{1,2,3\}$, encontrar por lo menos cinco proposiciones verdaderas.
		\answer Algunos ejemplos son $P(1)$, $P(2)$, $P(3)$, $\forall x: 2x+1>2$, $P(1) \land P(2)$, $\exists x: 2x+1 \leq 2$. \href{https://youtu.be/rnaCiSpVtP4?t=633}{Resolución por MathLogic}.

	\end{enumcols}

	\iffalse
	\exercise A partir del webinar \href{https://youtu.be/kU3_XLfn4jo}{La lógica y los circuitos}, responder:
	\begin{enumcols} 
		\Item ¿Qué elementos físicos se han usado para representar los dos estados lógicos (0/1 o F/V)? 
		\answer Historicamente se ha utilizado la corriente electrica para para tener dichos estados. Cuando los transistores tienen un voltaje \textit{alto} se activa el paso de corriente emulando el estado V, mientras que si tienen un voltaje \textit{bajo} no pasará corriente y estarán en estado F. Sin embargo para establecer los dos estados se puede utilizar cualquier tipo de corriente, Steve Mould utilizó \href{https://youtu.be/IxXaizglscw}{agua} y varias personas han experimentado con dominos, entre ellos, \href{https://youtu.be/SudixyugiX4}{Neil Fraser}, \href{https://youtu.be/OpLU__bhu2w}{Stand-up Maths} y \href{https://youtu.be/lNuPy-r1GuQ}{Numberphile}.  

		\item ¿Cómo se podría armar físicamente una lógica de 3 valores? 
		\answer Una opción es utilizar tres corrientes: baja, uno mediana y uno alta. La dificultad a partir de allí es armar maquinas físicas que logren representar las compuertas lógicas a partir de dichos voltajes. Análogamente cualquier cosa que represente una corriente (como agua o dominós) podría ser empleada. 

		\item ¿Qué dificultades existen al utilizar lógica en los circuitos? 
		\answer La dificultad principal para representar los estados lógicos es pasar de procesos físicos analógicos, con posibilidad de tener infinitos valores, a valores digitales con dos estados bien definidos. 

		\Item ¿Qué es el álgebra de Boole y que relación tiene con la lógica?
		\answer El álgebra de Boole es un sistema matemático para representar variables lógicas. Para constituir el sistema se definen operaciones complemento, suma, producto, etc), valores que pueden tomar las variables (0,1) y otras notaciones y reglas que permiten modelar circuitos lógicos.
		
		\Item ¿Cómo se representan los operadores lógicos típicos en el álgebra de Boole? 
		\answer Negación $\neg p$ se escribe $\bar{p}$, conjunción $p\land q$ es producto $p.q$ y disyunción $p \lor q$ es suma $p+q$. Estas operaciones se definen para ser utilizadas con los valores $0$ y $1$ a partir de reglas internas, por ejemplo los casos que definen a la siyunción son $0+0=0$, $1+0=1$, $0+1=1$, $1+1=1$, que son análogos a la tabla de verdad.

		\item ¿Cómo se conoce a las formas normales disyuntiva (DNF) y conjuntiva (CNF) en el álgebra de Boole? 
		\answer A la DNF se le suele llamar \textit{suma de productos} y a la CNF \textit{producto de sumas}.

		\item A partir de las tablas de verdad mostradas en el webinar, indicar cuál es la expresión lógica que representa la salida de los circuitos lógicos: \\ - salida del Multiplexor,\\ - Sum y Carry en Half Adder,\\ - Sum y CarryOut en Full Adder,\\ - Difference y Borrow en Half Subtractor,\\ - salidas del Decodificador,\\ - salidas del Demultiplexor. 

		\item Encontrar algún error en el webinar o tema sobre el cuál se debe aclarar algo 

		\item Buscar algún aspecto de la relación entre matemática discreta y teoría de circuitos que no haya sido tratado en el webinar

	\end{enumcols}
	\fi
	

\end{enumerate}

\end{document}