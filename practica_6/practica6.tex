\documentclass[a4paper]{article}
\usepackage[margin=1.5cm]{geometry}

%\documentclass[11pt]{article}
%\usepackage[paperwidth=9cm,paperheight=60cm,margin=0.4cm]{geometry}

\usepackage{multicol}
\usepackage{enumitem}
\usepackage{graphicx}

%Links
\usepackage[colorlinks = true,
            linkcolor = blue,
            urlcolor  = blue,
            citecolor = blue,
            anchorcolor = blue]{hyperref}

%Simbolos matemáticos
\usepackage{amsmath}
\usepackage{amssymb}

%Enumeracion
\usepackage{enumitem}

%Páginas sin numeración
\pagestyle{empty}

%Interlineado
\renewcommand{\baselinestretch}{1.5}

%Arreglar comillas
\usepackage [autostyle]{csquotes}
\MakeOuterQuote{"}

%Macros
\newcommand{\Item}{\item[\stepcounter{enumii}$\blacktriangleright$\textbf{(\alph{enumii})}]} %Negrita en algunos items
\newcommand{\answer}{\item[**]}
\newcommand{\exercise}{\item}

%Logic macros
\newcommand{\then}{\to}
\newcommand{\eq}{\leftrightarrow}
\newcommand{\xor}{\veebar}
\newcommand{\nor}{\downarrow}
\newcommand{\nimply}{\nrightarrow}
\newcommand{\nand}{\uparrow}
\newcommand{\Then}{\Rightarrow}
\newcommand{\Eq}{\Leftrightarrow}


\begin{document}

\noindent \hrulefill 
\vspace{-7pt}
\begin{center} 
	\textbf{ Práctica 6: Técnicas de conteo y número combinatorio} \\
	Comisión: Rodrigo Cossio-Pérez y Leonardo Lattenero
\end{center}
\vspace{-10pt}
\hrulefill


\begin{enumerate}

	\exercise Un turista debe trasladarse de una ciudad a otra y puede optar por viajar en avión, ómnibus o tren y en cada uno de esos medios puede elegir viajar en primera clase o en clase turista:
	%\begin{multicols}{2}
	\begin{enumerate} [label=(\alph*)]
		\item ¿De cuántas maneras distintas puede realizar el viaje?
		\answer 6. Por principio de multiplicación: $3 . 2 = 6$. 

		\item Si el turista decide visitar Buenos Aires, Rosario, Córdoba, Mendoza y Salta (en ese orden) contando con las mismas opciones de transporte, ¿de cuántas maneras puede realizar el itinerario?
		\answer 1296. Realiza 4 viajes y ante cada viaje tiene 6 opciones, por lo que las opciones totales serán: $6.6.6.6 = 6^4 = 1296$. Alternativamente se puede pensar que se seleccionan 6 opciones con repetición en 4 viajes, por lo que es $P^r(6,4)=6^4=1296$.
		
		\item Como el ítem anterior, pero suponiendo que en ómnibus hay sólo una clase
		\answer 625. Por principio de la suma, en cada viaje tiene 5 opciones: $ 2 + 2 + 1 = 5$. Como realiza 4 viajes, las opciones totales serán: $5.5.5.5 = 5^4 = 625$.  Alternativamente se puede pensar que se seleccionan 5 opciones con repetición en 4 viajes, por lo que es $P^r(5,4)=5^4=625$.
	\end{enumerate}
	%\end{multicols}

	\exercise Resolver los siguientes problemas random
	%\begin{multicols}{2}
	\begin{enumerate} [label=(\alph*)]
		\item La cerradura de una caja de caudales se compone de tres anillos, cada uno de los cuales está marcado con 20 letras distintas. ¿Cuál es el máximo número de intentos para abrirla que resulten infructuosos?
		\answer 7999. Hay 20 opciones para cada anillo, por lo que las opciones totales son: $20.20.20 = 20^3 = 8000$. Sin embargo, en el peor de los casos el intento 8000 cuenta como exitoso, por lo que los intentos infructuosos son 7999. Alternativamente se puede pensar que se seleccionan 20 opciones con repetición en 3 anillos, por lo que es $P^r(20,3)=20^3=8000$. 

		\item Para confeccionar un examen, se dispone de 3 problemas de geometría, 4 de combinatoria y 2 de álgebra. De cuántas maneras pueden ordenarse los problemas si los que corresponden a un mismo tema deben aparecer en forma consecutiva?
		\answer 1728. Hay 3 temas, por lo que se pueden ordenar de $3! = 6$ formas. En el tema de geometría $3! = 6$ formas de ordenar los problemas, en el de combinatoria hay $4!=24$ formas de ordenarlos, en el de álgebra hay $2!=2$ formas. Por lo que el total de opciones es $6.6.24.2 = 1728$

	\end{enumerate}
	%\end{multicols}

	\exercise Calcular de cuántas maneras pueden sentarse 6 niños y 4 niñas en el cine en 10 asientos consecutivos, si:
	%\begin{multicols}{2}
	\begin{enumerate} [label=(\alph*)]
		\item Todas las niñas desean sentarse juntas y lo mismo sucede con los niños.
		\answer 34560. Hay dos grupos (niñas y niños), por lo que hay $2!=2$ formas de ubicarlos. En el grupo de niñas hay $4!=24$ formas de ubicarlas, y en el grupo de niños hay $6!=720$ formas. Por lo que las opciones totales son $2.24.720=34560$.

		\item Las niñas desean estar juntas y a los varones les da igual.
		\answer 120960. El grupo de niñas puede considerarse como un bloque a distribuir entre los niños. Los 6 niños más el bloque de niñas se pueden distribuir de $P(7,7)=7!=5040$ formas. Además, dentro del bloque las niñas pueden distribuirse de $P(4,4)=4!=24$ formas. Por lo que las opciones totales son $5040.24=120960$.

		\item Daniela y Pedro no quieren estar juntos

	\end{enumerate}
	%\end{multicols}

	\exercise Resolver las siguientes situaciones sobre dígitos:
	%\begin{multicols}{2}
	\begin{enumerate} [label=(\alph*)]
		\item ¿Cuántos números de tres cifras distintas puede formarse con los dígitos impares?
		\answer 125. Hay 5 dígitos impares (1,3,5,7,9). Por principio de multiplicación, en la primera cifra hay 5 opciones, en la segunda 4 y en la tercera 3, por lo que las opciones totales son: $5.4.3 = 60$. Alternativamente se puede pensar que se seleccionan 3 opciones sin repetición en 5 posibles, por lo que es $P(5,3)=\frac{5!}{(5-3)!}=\frac{5!}{2!}=60$.

		\item ¿Cuántos números capicúas de cinco cifras y que no comienzan con cero hay?
		\answer 900. Hay 9 opciones para la primera cifra (del 1 al 9), 10 para la segunda (del 0 al 9), 10 para la tercera (del 0 al 9), 1 para la cuarta (la misma que la segunda) y 1 para la quinta (la misma que la primera). Por lo tanto, las opciones totales son: $9.10.10.1.1=900$.

		\item Considerando las cifras 0-5, ¿cuántos números hay de 4 cifras distintas que no arrancan con 0?
		\answer 300. Hay 5 opciones para la primera cifra (del 1 al 5) y luego 5 opciones para las restantes (del 0 al 5, excepto el número ya elegido) que deben elegirse sin repetir en 3 posiciones (las restantes para formar al número). Por lo tanto, el número total de opciones es: $5.P(5,3)= 5. \frac{5!}{(5-3)!}= 5. \frac{5!}{2!} = 5.60=300$.

		\item Considerando las cifras 0-5, ¿cuántos de 4 cifras distintas que no inician con 0 son menores que 3000?
		\answer 120. Para la primer cifra hay 2 opciones (1 o 2). Luego 5 opciones para las restantes (del 0 al 5, excepto el número ya elegido) que deben elegirse sin repetir en 3 posiciones (las restantes para formar al número). Por lo tanto, el número total de opciones es: $2.P(5,3)= 2. \frac{5!}{(5-3)!}= 2. \frac{5!}{2!} = 2.60=120$.

		\item Considerando las cifras 0-5, ¿cuántos de 4 cifras distintas son divisibles por 5 y no comienzan con 0?
		\answer 108. Hay que utilizar el principio de la adición. Para que el número sea divisible por 5, la última cifra debe ser 0 o 5. Si la última cifra es 0, hay 5 opciones para la primera cifra (del 1 al 5) y luego 4 opciones para las restantes en 2 posiciones sin repetir, es decir, $5.P(4,2)=5.\frac{4!}{(4-2)!}=5.12=60$. Si la última cifra es 5, hay 4 opciones para la primera cifra (del 1 al 4) y luego 4 opciones para las restantes en 2 posiciones sin repetir, es decir, $4.P(4,2)=4.\frac{4!}{(4-2)!}=4.12=48$. Por lo tanto, el número total de opciones es: $60+48=108$.

		\item Considerando los números de 5 cifras que se obtienen permutando los dígitos de 17283, ¿cuántos de ellos son impares?
		\answer 72. Para la ultima cifra hay 3 opciones (1,3,7) y luego 4 opciones para las restantes en 4 posiciones sin repetir, es decir, $3.P(4,4)=3.4!=3.24=72$.

		\item Considerando los números de 5 cifras que se obtienen permutando los dígitos de 17283, de menor a mayor, ¿qué lugar ocupa el 23178?
		\answer Posición 31. Contamos las opciones de que el numero sea menor o igual. Para que el numero sea menor igual debería ser $1----$ que tiene $P(4,4)=4!=24$ opciones. O bien $21---$ que tiene $P(3,3)=3!=6$ opciones. O bien el propio $23178$ que tiene $1$ opción. Por lo que el numero total de opciones, que también es la posición del 23178, es $24+6+1=31$.

		\item Considerando los números de 5 cifras que se obtienen permutando los dígitos de 17283, de mayor a menor , ¿qué lugar ocupa el 83712?
		\answer Posición 8. Contamos las opciones de que el número sea mayor o igual. Una opción es que sea $87---$ que tiene $P(3,3)=3!=6$ opciones. Otra opción es que sea $837--$ con $P(2,2)=2!=2$ opciones (esta opción incluye al 83712). El número total de opciones, que es igual a la posición del número, será $6+2=8$.


	\end{enumerate}
	%\end{multicols}


\end{enumerate}

\end{document}