\documentclass[a4paper]{article}
\usepackage[margin=1.5cm]{geometry}

%\documentclass[11pt]{article}
%\usepackage[paperwidth=9cm,paperheight=60cm,margin=0.4cm]{geometry}

\usepackage{multicol}
\usepackage{enumitem}
\usepackage{graphicx}

%Links
\usepackage[colorlinks = true,
            linkcolor = blue,
            urlcolor  = blue,
            citecolor = blue,
            anchorcolor = blue]{hyperref}

%Simbolos matemáticos
\usepackage{amsmath}
\usepackage{amssymb}

%Enumeracion
\usepackage{enumitem}

%Páginas sin numeración
\pagestyle{empty}

%Interlineado
\renewcommand{\baselinestretch}{1.5}

%Arreglar comillas
\usepackage [autostyle]{csquotes}
\MakeOuterQuote{"}

%Macros
\newcommand{\Item}{\item[\stepcounter{enumii}$\blacktriangleright$\textbf{(\alph{enumii})}]} %Negrita en algunos items
\newcommand{\answer}{\item[**]}
\newcommand{\exercise}{\item}

%Logic macros
\newcommand{\then}{\to}
\newcommand{\eq}{\leftrightarrow}
\newcommand{\xor}{\veebar}
\newcommand{\nor}{\downarrow}
\newcommand{\nimply}{\nrightarrow}
\newcommand{\nand}{\uparrow}
\newcommand{\Then}{\Rightarrow}
\newcommand{\Eq}{\Leftrightarrow}


\begin{document}

\noindent \hrulefill 
\vspace{-7pt}
\begin{center} 
	\textbf{ Práctica 0: Propiedades algebraicas} \\
	Comisión: Rodrigo Cossio-Pérez y Leonardo Lattenero
\end{center}
\vspace{-10pt}
\hrulefill


\begin{enumerate}

	\exercise Hallar todos los valores de $x$ que responden a la ecuación
	%\begin{multicols}{2}
	\begin{enumerate} [label=(\alph*)]
		\item $2x-7x-5=0$
		
		\item $(3x-1)^2-1=9x^2+12$

		\item $\displaystyle\frac{2-x}{x-1}=3$

		\item $\displaystyle\frac{-2x+1}{x+1}=\displaystyle\frac{4x-7}{-2(x+6)}$

		\item $6(x+9)=2(3x+\displaystyle\frac{37}{2})+17$
	
	\end{enumerate}
	%\end{multicols}

	\exercise Simplificar las siguientes expresiones algebraicas
	%\begin{multicols}{2}
	\begin{enumerate} [label=(\alph*)]
		\item $4.\left(3^{x}\right)^{2}-3^{2x+1}$
		\answer $3^{2x}$
	
	\end{enumerate}
	%\end{multicols}


	\exercise Decidir si las siguientes expresiones son equivalentes
	%\begin{multicols}{2}
	\begin{enumerate} [label=(\alph*)]
		\item $(2x+5)(x+3)$ y $2x^2+11x+15$
		\answer Son equivalentes

		\item $5.2^{x+1}$ $\displaystyle\frac{5.2^{x}}{2}$
		\answer No son equivalentes
		
		\item $2x^2+4x-6$ y $(x-1)(x+3)$
		\answer No son equivalentes

	\end{enumerate}
	%\end{multicols}

	\exercise Hallar todos los valores de $x$ que responden a la inecuación
	%\begin{multicols}{2}
	\begin{enumerate} [label=(\alph*)]
		\item $2x+1>0$

		\item $\displaystyle\frac{3x-5}{x-1}<0$

		\item $\displaystyle\frac{-4x-2}{x+1}>1$

		\item $(x-2)(x+1)>0$

		\item $-8(x-2)(2x+7)<0$
	
	\end{enumerate}
	%\end{multicols}


	\exercise Analizar si que valen las siguientes propiedades y justificarlo
	\begin{multicols}{2}
	\begin{enumerate} [label=(\alph*)]
		\item $x < \sqrt{2} x +1$ para $x<0$
		\answer Válido. Como $x>0$ y $\sqrt{2}>1$: $x < \sqrt{2} < \sqrt{2} x +1$
		
		\item $\displaystyle\frac{x-1}{2} < x$ para $x>0$
		\answer Válido. Como $x>0$: $x > \displaystyle\frac{x}{2} > \displaystyle\frac{x}{2}-\displaystyle\frac{1}{2}$ 

		\item Si se tienen $a>b>c$, esto implica que $a+1>c+1$
		\answer	Valido. $a>b>c$ implica $a>c$ por transitividad. Si sumamos 1, se tiene $a+1>c+1$

		\item $a^2 > a$
		\answer Inválido. Si $a=0.1$ se tiene que $a^2=0.01$. \textit{Nota: La propiedad vale para $a>1$y también para $a<0$.}

		\item $3^{x} < 3^{x+1}$ para $x \in \mathbb{N}$
		\answer Válido. Como $1 < 3$, multiplicamos por $3^{x}$ (que es positivo) y obtenemos $3^{x} < 3.3^{x}$. Finalmente, por propiedades de la potenciación, $3^{x} < 3^{x+1}$.

		\item Si $a>b$, entonces $a^x > b^x$
		\answer Inválido. Si $x=0$ se tiene que $a^0=b^0=1$. \textit{Nota, a propiedad es válidad para $x>0$.}

	\end{enumerate}
	\end{multicols}

	\exercise Hallar el conjunto de valores del parámetro $k \in \mathbb{R}$ que cumplen la condición
	%\begin{multicols}{2}
	\begin{enumerate} [label=(\alph*)]

		\item La parábola $x^2+kx+4$ tiene una única raíz.

		\item La parábola $kx^2+4x+2$ tiene dos raíces reales.

		\item La parábola $\displaystyle\frac{1}{2}x^2-3x+2k$ no tiene raíces reales.
	\end{enumerate}
	%\end{multicols}

	\exercise Indicar a qué conjunto numérico ($\mathbb{N}$, $\mathbb{Z}$, $\mathbb{Q}$ o $\mathbb{R}$) pertenecen los siguientes números y dar ejemplos que justifiquen
	\begin{multicols}{2}
	\begin{enumerate} [label=(\alph*)]
		\item $3x+5$ con $x \in \mathbb{N}$
		\answer Pertenece a $\mathbb{N}$, por ser la suma de dos números naturales.

		\item $4x^2$ con $x \in \mathbb{N}$
		\answer Pertenece a $\mathbb{N}$, por ser el producto de números naturales.

		\item $\displaystyle\frac{x^2}{3}+1$ con $x \in \mathbb{N}$
		\answer Pertenece a $\mathbb{Q}$, por ser el cociente de dos números naturales. 

		\item $-6x+1$ con $x \in \mathbb{N}$
		\answer Pertenece a $\mathbb{Z}$, por ser la suma de un número entero negativo y un número natural.

		\item $x^2+x+1$ con $x \in \mathbb{Z}$
		\answer Pertenece a $\mathbb{Z}$, por ser la suma de tres números enteros.

		\item $x+\displaystyle\frac{1}{2}$ con $x \in \mathbb{Z}$
		\answer Pertenece a $\mathbb{Q}$, por ser la suma de un número entero y un número racional.

		\item $\displaystyle\frac{1}{x-1}$ con $x \in \mathbb{Z}$ y $x \neq 1$
		\answer Pertenece a $\mathbb{Q}$, por ser el cociente de un número entero y un número entero distinto de cero.

		\item $3 \sqrt{x}$ con $x \in \mathbb{N}$
		\answer Pertenece a $\mathbb{R}$, por ser la raíz cuadrada de un número natural.

		\item $\displaystyle\frac{x^2}{x-4}$ con $x \in \mathbb{Z}$ y $x \neq 4$
		\answer Pertenece a $\mathbb{Q}$, por ser el cociente de un número entero y un número entero distinto de cero.

		\item $\displaystyle\frac{\sqrt{3}x-3}{2}$ con $x \in \mathbb{Z}$
		\answer Pertenece a $\mathbb{R}$, por ser el cociente de un número irracional y un número entero. 

		\item $x+3$ con $x \in \mathbb{Q}$
		\answer Pertenece a $\mathbb{Q}$, por ser la suma de dos números racionales.

		\item $\displaystyle\frac{1}{x}$ con $x \in \mathbb{Q}$ y $x \neq 0$
		\answer Pertenece a $\mathbb{Q}$, por ser el cociente de dos números racionales.

		\item $\sqrt{x}$ con $x \in \mathbb{Q}$
		\answer Pertenece a $\mathbb{R}$, por ser la raíz cuadrada de un número racional.
	\end{enumerate}
	\end{multicols}

	\exercise Graficar las siguientes funciones, indicando sus elementos notables (ordenada/abscisas al origen, vértice, etc.)
	%\begin{multicols}{2}
	\begin{enumerate} [label=(\alph*)]
		\item $y=-4x+2$

		\item $y=\displaystyle\frac{2}{3}x-1$
		
		\item $y=x^2+4x+4$

		\item $y=-(x-1)^2+3$
	
	\end{enumerate}
	%\end{multicols}

	\exercise Analizar las siguientes situaciones geométricas
	%\begin{multicols}{2}
	\begin{enumerate} [label=(\alph*)]
		\item Averiguar si la recta $y=2x+1$ y la recta $y=2x-5$ son paralelas

		\item Averiguar si la recta $y=2x+1$ y la recta $y=3x+1$ son perpendiculares

		\item Hallar una recta perpendicular a la recta $y=2x+1$ que pase por el punto $(1,2)$

		\item Hallar una recta paralela a la recta $y=\frac{1}{3}x+1$ que pase por el punto $(1,1)$

		\item Calcular la intersección de las rectas $y=2x+1$ y $y=3x-1$

		\item Calcular la intersección de la recta $y=2x+1$ y la parábola $y=x^2+1$

		\item Averiguar si a recta $y=-x+3$ se intersecta con la parábola $y=x^2+2x+5$
	
		\item Dar una recta perpendicular a la recta $x=2$ que pase por el punto $(1,5)$
	\end{enumerate}
	%\end{multicols}

\end{enumerate}

\end{document}