\documentclass{template_practica}

\begin{document}

\practiceheader{Práctica 0: Propiedades algebraicas}{Comisión: Rodrigo Cossio-Pérez y Leonardo Lattenero}

\begin{enumerate}

	\exercise Hallar todos los valores de $x$ que responden a la ecuación
	\begin{enumcols}
		\item $2x-7x-5=0$
		
		\item $(3x-1)^2-1=9x^2+12$

		\item $\f{2-x}{x-1}=3$

		\item $\f{-2x+1}{x+1}=\f{4x-7}{-2(x+6)}$

		\item $6(x+9)=2\left(3x+\f{37}{2}\right)+17$
	
	\end{enumcols}

	\exercise Decidir si las siguientes expresiones son equivalentes
	\begin{enumcols}
		\item $(2k+5)(k+3)$ ~y~ $2k^2+11k+15$
		\answer Son equivalentes

		\item $5.2^{x+1}$ ~y~ $\f{5.2^{x}}{2}$
		\answer No son equivalentes
		
		\item $2x^2+4x-6$ ~y~ $(x-1)(x+3)$
		\answer No son equivalentes

		\item $2n^3+3n^2+n$ ~y~ $n(n+1)(2n+1)$
		\answer Son equivalentes

		\item $4.\left(3^{x}\right)^{2}-3^{2x+1}$ y $3^x$
		\answer No son equivalentes

	\end{enumcols}

	\exercise Hallar todos los valores de $x$ que responden a la inecuación
	\begin{enumcols}
		\item $2x+1>0$

		\item $\f{3x-5}{x-1}<0$

		\item $\f{-4x-2}{x+1}>1$

		\item $(x-2)(x+1)>0$

		\item $-8(x-2)(2x+7)<0$
	
	\end{enumcols}

	\exercise Analizar si las siguientes propiedades son correctas. Justificar
	\begin{enumcols}[2]
		\item $x < \sqrt{2} x +1$ para $x>0$
		\answer Correcto. Como $x>0$ y $\sqrt{2}>1$: $x < \sqrt{2} < \sqrt{2} x +1$
		
		\item $\f{x-1}{2} < x$ para $x>0$
		\answer Correcto. Como $x>0$: $x > \f{x}{2} > \f{x}{2}-\f{1}{2}$ 

		\item Si se tienen $a>b>c$, esto implica que $a+1>c+1$
		\answer	Correcto. $a>b>c$ implica $a>c$ por transitividad. Si sumamos 1, se tiene $a+1>c+1$

		\item $a^2 > a$
		\answer Incorrecto. Si $a=0.1$ se tiene que $a^2=0.01$. \textit{Nota: La propiedad vale para $a>1$y también para $a<0$.}

		\item $3^{x} < 3^{x+1}$ para $x \in \N$
		\answer Correcto. Como $1 < 3$, multiplicamos por $3^{x}$ (que es positivo) y obtenemos $3^{x} < 3.3^{x}$. Finalmente, por propiedades de la potenciación, $3^{x} < 3^{x+1}$.

		\item Si $a>b$, entonces $a^x > b^x$
		\answer Incorrecto. Si $x=0$ se tiene que $a^0=b^0=1$. \textit{Nota, a propiedad es válidad para $x>0$.}

	\end{enumcols}

	\exercise Analizar si las siguientes afirmaciones son correctas. Justificar
	\begin{enumcols}[2]
		\item $13$ es un número impar
		\answer Correcto ya que $13$ puede escribirse como $2k+1$ con $k=6 \in \Z$

		\item $68$ es un número impar
		\answer Incorrecto. $68$ es un número par ya que puede escribirse como $2k$ con $k=34 \in \Z$

		\item $-12$ es un número par
		\answer Correcto ya que $-12$ puede escribirse como $2k$ con $k=-6 \in \Z$

		\item $0$ es un número par
		\answer Correcto ya que $0$ puede escribirse como $2k$ con $k=0 \in \Z$

		\item $30$ es múltiplo de 5
		\answer Correcto ya que $30$ puede escribirse como $5k$ con $k=6 \in \Z$

		\item $17$ es múltiplo de 3
		\answer Incorrecto. $17$ no es múltiplo de 3 ya que no puede escribirse como $3k$ con $k \in \Z$

		\item $-12$ es múltiplo de 4
		\answer Correcto ya que $-12$ puede escribirse como $4k$ con $k=-3 \in \Z$

		\item $2$ divide a $-12$
		\answer Correcto ya que $-12$ puede escribirse como $2k$ con $k=-6 \in \Z$

		\item $-3$ divide a $11$
		\answer Incorrecto. $11$ no es múltiplo de $-3$ ya que no puede escribirse como $-3k$ con $k \in \Z$

	\end{enumcols}

	\exercise Hallar el conjunto de valores del parámetro $k \in \R$ que cumplen la condición
	\begin{enumcols}

		\item La parábola $x^2+kx+4$ tiene una única raíz.

		\item La parábola $kx^2+4x+2$ tiene dos raíces reales.

		\item La parábola $\f{1}{2}x^2-3x+2k$ no tiene raíces reales.
	\end{enumcols}

	\exercise Indicar a qué conjunto numérico ($\N$, $\Z$, $\Q$ o $\R$) pertenecen los siguientes números y dar ejemplos que justifiquen
	\begin{enumcols}[2]
		\item $3x+5$ con $x \in \N$
		\answer Pertenece a $\N$, por ser la suma de dos números naturales.

		\item $4x^2$ con $x \in \N$
		\answer Pertenece a $\N$, por ser el producto de números naturales.

		\item $\f{x^2}{3}+1$ con $x \in \N$
		\answer Pertenece a $\Q$, por ser el cociente de dos números naturales. 

		\item $-6x+1$ con $x \in \N$
		\answer Pertenece a $\Z$, por ser la suma de un número entero negativo y un número natural.

		\item $x^2+x+1$ con $x \in \Z$
		\answer Pertenece a $\Z$, por ser la suma de tres números enteros.

		\item $x+\f{1}{2}$ con $x \in \Z$
		\answer Pertenece a $\Q$, por ser la suma de un número entero y un número racional.

		\item $\f{1}{x-1}$ con $x \in \Z$ y $x \neq 1$
		\answer Pertenece a $\Q$, por ser el cociente de un número entero y un número entero distinto de cero.

		\item $3 \sqrt{x}$ con $x \in \N$
		\answer Pertenece a $\R$, por ser la raíz cuadrada de un número natural.

		\item $\f{x^2}{x-4}$ con $x \in \Z$ y $x \neq 4$
		\answer Pertenece a $\Q$, por ser el cociente de un número entero y un número entero distinto de cero.

		\item $\f{\sqrt{3}x-3}{2}$ con $x \in \Z$
		\answer Pertenece a $\R$, por ser el cociente de un número irracional y un número entero. 

		\item $x+3$ con $x \in \Q$
		\answer Pertenece a $\Q$, por ser la suma de dos números racionales.

		\item $\f{1}{x}$ con $x \in \Q$ y $x \neq 0$
		\answer Pertenece a $\Q$, por ser el cociente de dos números racionales.

		\item $\sqrt{x}$ con $x \in \Q$
		\answer Pertenece a $\R$, por ser la raíz cuadrada de un número racional.
	\end{enumcols}

	\exercise Graficar las siguientes funciones, indicando sus elementos notables (ordenada/abscisas al origen, vértice, etc.)
	\begin{enumcols}
		\item $y=-4x+2$

		\item $y=\f{2}{3}x-1$
		
		\item $y=x^2+4x+4$

		\item $y=-(x-1)^2+3$
	
	\end{enumcols}

	\exercise Analizar las siguientes situaciones geométricas
	\begin{enumcols}
		\item Averiguar si la recta $y=2x+1$ y la recta $y=2x-5$ son paralelas

		\item Averiguar si la recta $y=2x+1$ y la recta $y=3x+1$ son perpendiculares

		\item Hallar una recta perpendicular a la recta $y=2x+1$ que pase por el punto $(1,2)$

		\item Hallar una recta paralela a la recta $y=\f{1}{3}x+1$ que pase por el punto $(1,1)$

		\item Calcular la intersección de las rectas $y=2x+1$ y $y=3x-1$

		\item Calcular la intersección de la recta $y=2x+1$ y la parábola $y=x^2+1$

		\item Averiguar si a recta $y=-x+3$ se intersecta con la parábola $y=x^2+2x+5$
	
		\item Dar una recta perpendicular a la recta $x=2$ que pase por el punto $(1,5)$
	\end{enumcols}

\end{enumerate}

\end{document}